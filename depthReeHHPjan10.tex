\documentclass[onecolumnreqno]{amsart}
\usepackage{t1enc}
\usepackage[latin2]{inputenc}
\usepackage{lmodern}
\usepackage[mathscr]{eucal}
\usepackage{amsmath,dsfont}
\usepackage{amssymb}
\usepackage{amsthm}
\usepackage{enumerate}
\usepackage{multirow}
%\usepackage{color}
\usepackage[all]{xy}
%\usepackage{currfile}
%\usepackage{ulem}
\newtheorem{tetel}{Theorem}[section]
\newtheorem{defi}[tetel]{Definition}
\newtheorem{cor}[tetel]{Corollary}
\newtheorem{lemma}[tetel]{Lemma}
\newtheorem{prop}[tetel]{Proposition}
\newtheorem{rem}[tetel]{Remark}
\newtheorem{notation}[tetel]{Notation}
\addtolength{\textwidth}{24mm}
\addtolength{\textheight}{4cm}
\setlength{\hoffset}{-12mm}
\setlength{\voffset}{-23mm}
\setlength{\parindent}{0pt}
%\setlength{\parskip}{\baselineskip}
\newcommand{\cupdot}{\mathbin{\mathaccent\cdot\cup}}
\def\Irr{\operatorname{Irr}}
\def\C{{\mathbb C}}
\begin{document}
%\today\hfill \currfilename
\title{The  depth of the maximal subgroups of Ree groups}
\author{L.~H\'ethelyi}
\address{Department of Algebra, Budapest University of Technology and Economics, H-1521 Budapest, M\H uegyetem rkp. 3--9.}
\email{fobaba@t-online.hu}

\author{E.~Horv\'ath}
\address{Department of Algebra, Budapest University of Technology and Economics, H-1521 Budapest, M\H uegyetem rkp. 3--9.}
\email{he@math.bme.hu}

\author{F.~Pet\'enyi}
\address{Department of Algebra, Budapest University of Technology and Economics,
 H-1521 Budapest, M\H uegyetem kp. 3--9.}
\email{petenyi.franciska@gmail.com}
%%%%%%%%%%%%%%%%%%%%%%%%%%%%%%%%%%%%%%%%%%%%%%%%%%%%%%%%%%%%%%%%%%%%%%%%%%%%%%%%
\dedicatory{To the memory of Professor K. Corr\'adi}
\abstract{%
We determine the combinatorial  and the ordinary depth of the maximal subgroups of the simple Ree
 groups ${}^2G_2(q)$. As an application of these calculations, we determine the subdegrees of primitive actions of the groups ${}^2G_2(q)$. We also improve an earlier estimate of Burness, Liebeck and Shalev
 on the base size of "non-standard" primitive actions of ${}^2G_2(q)$ on the coset space of maximal subgroups from $3$ to $2$. \\
Key words: Ree groups, combinatorial depth, ordinary depth, maximal subgroup, primitive action, base size, subdegree.\\
AMS MSC2000 classification: 20D06, 20C15, 20B20, 20C33, 20B15 
}
}
\maketitle

\section{Introduction}
$\club$
Similarly to \cite{F}, \cite{F2}, \cite{FKR} and \cite{HHP}, we will study the  combinatorial  and ordinary depth of some subgroups of a certain group class.
We will determine the depth of maximal subgroups of Ree groups ${}^2G_2(q)$, $q\geqslant 27$, which we will denote by $R(q)$.

These two notions are closely connected to intersections of conjugates of subgroups of the group. This is central in the classical problem in determining the base sizes of permutation representations of a group. With the methods we used to
 calculate the depth of maximal subgroups of Ree groups we could also determine
the base sizes of primitive permutation representations of these groups.
This improves  the base size bound of \cite[Theorem 4]{BLSh} from $3$ to $2$, which is sharp, see Theorem \ref{1.4}. Additionally we could also determine the subdegrees of these permutation representations, see Theorem \ref{tsubd}.\\
 
 Originally depth was defined for von-Neumann algebras, see \cite{GHJ}.
Later it was also defined for Hopf algebras, see \cite{KN}. For some recent results in this direction, see
\cite{K} and \cite{KHSZ}. In \cite{KK} and later in \cite{BKK} the
 depth  of semisimple algebra inclusions was studied. The ordinary depth of a group inclusion $H\leqslant G$ (denoted by $d(H,G)$) is defined as the depth of  the group algebra inclusion $\C H\subseteq \C G$, studied in \cite{BDK}. It is proved in \cite{BDK} that the ordinary depth of a subgroup $H$ of a finite group  $G$ is  bounded  from above by the combinatorial depth $d_c(H,G)$.
 In particular, it is always finite. 
For the  original definitions of $d_c(H,G)$ and of $d(H,G)$, see \cite{BDK}.  Here  we will use some equivalent forms of the original definitions.\\


Let us introduce the following notations:
  $H^x:=x^{-1}Hx$ and $H^{(x_1,\dots,x_n)}:=H\cap H^{x_1}\cap\dots \cap H^{x_n}$, for $x,x_1,\dots,x_n\in G$.

  We will also use the notations:  $\mathcal{U}_{\,0}(H):=\{H\}$,
 $\mathcal{U}_{\,1}(H):=\{ H\cap H^x \mid x\in G\}$, and in general
let $ \mathcal{U}_{\,n}(H):=\{
 H^{(x_1,\dots,x_n)} \mid x_1,\dots,x_n\in G\}$\\



\begin{defi}\label{jellemzes}\cite[Thm. 3.9]{BDK}
Let $H$ be a subgroup of finite group $G$. The {\em combinatorial depth} $d_c(H,G)$ is the minimal possible positive integer  which can be determined from the
following upper bounds: 

\begin{enumerate}
\item{Let $i\geqslant 1$. The combinatorial depth $d_c(H,G)\leqslant 2i$ if and only if for
 any $x_1,\dots,x_i\in G$, there exist $y_1,\dots, y_{i-1}\in G$
 with $H^{(x_1,\dots,x_i)}=H^{(y_1,\dots ,y_{i-1})}$.
(In other words, $d_c(H,G)\leqslant 2i$ if and only if
$\mathcal{U}_{\;i}(H)=\mathcal{U}_{\,i-1}(H)$ )}
\item{Let $i> 1$. The  combinatorial depth $d_c(H,G)\leqslant 2i-1$ if and only if for any $x_1,\dots,x_i\in G$, there exist $y_1,\dots, y_{i-1}\in G$ with $H^{(x_1,\dots,x_i)}=H^{(y_1,\dots ,y_{i-1})}$ and $x_1hx_1^{-1}=y_1hy_1^{-1}$ for any $h\in H^{(x_1,\dots, x_i)}$.}
\item{ The combinatorial depth $d_c(H,G)=1$ if and only if  for every $x\in G$ there exists  an element
$y\in H$  such that $xhx^{-1}=yhy^{-1}$, for every $h\in H$.(In other words:
$G=HC_G(H)$.)}
\end{enumerate}
\end{defi}

It is easy to see that $d_c(H,G)\leqslant 2 $ if and only if $H\triangleleft G$.
\\

We define the ordinary depth using an equivalent form of it, more details are in \cite{BKK}.\\
Two irreducible characters $\alpha,\beta \in \Irr(H)$ are {\em related},
$\alpha \sim \beta$, if they are constituents of  $\chi|_H$, for some
 $\chi \in \Irr(G)$. The {\em distance} $d(\alpha,\beta)=m$
is the smallest integer $m$ such that there is a chain of irreducible
 characters of $H$ such that $\alpha=\psi_0 \sim \psi_1 \sim \ldots \sim
\psi_m=\beta $. If there is no such chain then
$d(\alpha,\beta)=-\infty $ and if $\alpha=\beta $ then the distance is zero.
 If $X$ is the set of irreducible constituents of $\chi|_H$ then
{\em $m(\chi):=\max_{\alpha \in \Irr(H)}\min_{\psi \in  X} d(\alpha,\psi)$.}

\begin{defi}\cite[Thm~3.9, Thm~ 3.13]{BKK}\label{distance} Let $H$ be a subgroup of a finite group $G$.
The {\em ordinary depth} $d(H,G)$ is the minimal possible  positive integer  which can be determined from the following upper bounds: 
\begin{itemize}
\item[(i)] Let $m\geqslant 1$. The  ordinary depth $d(H,G)$ $\leqslant 2m+1$ 
if and only if the distance between two irreducible characters of $H$ is at most $m$.
\item[ (ii)] Let $m\geqslant 2$. The   ordinary depth $d(H,G)\leqslant 2m$ 
if and only if $m(\chi)\leqslant m-1$ for all $\chi \in Irr(G)$.
\item[ (iii)]   $d(H,G)\leqslant 2$ if and only if $H$ is normal in $G$,  $d(H,G)=1$
if and only if $G=HC_G(x)$ for all $x\in H$.
\end{itemize}
\end{defi}

\begin{tetel}\cite[Thm~6.9]{BKK}\label{inter}
Suppose that  $H$ is a subgroup of a finite group $G$ and $N=Core_G(H)$
is the intersection  of $m$ conjugates of $H$. Then $d(H,G)
\leqslant 2m$ in $G$. If $N\leqslant Z(G)$  also holds, then $d(H,G)\leqslant 2m-1$.
\end{tetel}

A trivial consequence of  Theorem \ref{inter} is that if $G$ is simple and $H$ is a proper subgroup having a disjoint conjugate, 
 then  $d(H,G)=3$.\\

The paper is organized as follows. In Section 2 first we introduce the main information about Ree groups, after that in each section we determine the 
 combinatorial and ordinary depth of a fixed maximal subgroup, more precisely:
for  $N_G(P)$ ($P\in Syl_3(G)$), $N_G(M^{\pm1})$ ($M^{\pm1}\in Hall_{q\pm 3m+1}(G)$), $N_G(M)$ ($M\in Hall_{\frac{q+1}{4}}(G)$), $C_G(i)$ ($o(i)=2$), $G_0\simeq R(q_0),q_0>3.$ \\
In Section \ref{subd} we  prove Theorem \ref{tsubd}.
\\

Our main results are the following.


\begin{tetel}\label{1.4} Using the notations of  Theorem \ref{ree},
 the  combinatorial and the ordinary depths of the maximal subgroups $\mathcal M$ in $G=R(q)$ for $q\geqslant 27$ are the following:

\[
  \begin{array}{c|c|c}
   \mathcal{M} & d_c(\mathcal{M},G) & d(\mathcal{M},G) \\
   \hline
   N_G(P) & 5 & 5 \\
   N_G(M^{\pm 1}) & 4 & 3 \\
   N_G(M) & 6 & 3 \\
   C_G(i) & 6 & 3 \\
   G_0    & 4 & 3 
  \end{array}
\]
Moreover, apart from the parabolic subgroups $N_G(P)$, $P\in Syl_3(G)$, for every maximal subgroup $\mathcal M$ there is an element $x\in G$ with $\mathcal{M}\cap \mathcal{M}^x=\{1\}.$
In particular, every "non-standard" primitive action of $G$ on the coset
space $G/\mathcal {M}$ has base size $2$. (The standard primitive action on the coset space $G/N_G(P)$ has base size $3$.) 
\end{tetel} 

The above result about base sizes improves an ealier result, see   \cite[Theorem 4]{BLSh}, which
bounded the   base sizes for  "non-standard" primitive actions of $R(q)$   by $3$.


\eject
As another application of our calculated data,  we  prove in Section \ref{subd} the following:



\begin{tetel}\label{tsubd}
The subdegrees  $|{\mathcal M}:{\mathcal M}\cap {\mathcal M}^x|$ of primitive actions of Ree groups $R(q)$ for
$q\geqslant 27$ on the coset spaces $G/\mathcal {M}$ of maximal subgroups $\mathcal {M}$ are the following:

\begin{center}
\begin{tabular}{c|p{12cm}}
 $\mathcal {M}$ & subdegrees \\
\hline
$N_G(P)$ &  $1$,   $q^3$\\
\hline
$N_G(M^{\pm 1})$ & $1$, $6(q+1\pm 3^n)$, $3(q+1\pm 3^n)$, $2(q+1\pm 3^n)$, $(q+1\pm 3^n)$\\
\hline
$N_G(M)$ & $1$, $6(q+1)$, $3(q+1)$, $2(q+1)$, $(q+1)$\\
\hline
$C_G(i)$ & $1$, $(q-1)(q+1)$, $\frac{q(q-1)}{2}$, $\frac{q(q-1)(q+1)}{4}$,
$\frac{q(q-1)(q+1)}{2}$, $q(q-1)(q+1)$\\
\hline
$G_0$ & $1$, $(q_0^3+1)q_0^3(q_0-1)$, $(q_0^3+1)(q_0-1)$,
        $(q_0^3+1)q_0(q_0-1)$,  $\frac{(q_0^3+1)q_0(q_0-1)}{2}$, \\
      & $(q_0^3+1)q_0^2(q_0-1)$, $*\frac{(q_0^3+1)q_0^2(q_0-1)}{3}$,
        $\frac{(q_0^3+1)q_0^3(q_0-1)}{2}$,
        $(q_0^3+1)q_0^3$, \\
      & $(q_0+1-3^{n_0+1})q_0^3(q_0-1)$,
        $**\frac{(q_0+1-3^{n_0+1})q_0^3(q_0-1)}{2}$, \\
      & $(q_0+1+3^{n_0+1})q_0^3(q_0-1)$, $***\frac{(q_0+1+3^{n_0+1})q_0^3(q_0-1)}{2}$, $(q_0^2-q_0+1)q_0^3(q_0-1)$\\
\hline
$H\simeq R(3)$ & $1$,  $1512$,  $56$,   $168$,   $84$,   $504$,  $****168$,
    $216$,  $*****128$,   $756$ 
\end{tabular}
\end{center}

 Here the value at  $*$ can occur only if $2n+1=3(2n_0+1)$ and certain equations  over  $GF(q_0)$ have solutions in $GF(q)\setminus GF(q_0)$.
The value at $**$ occurs  if and only if $q_0+1+3^{n_0+1}|q+1$ and the value at $***$ occurs  if and only if  $q_0+1-3^{n_0+1}|q+1$.\\
The value at $****$ occurs only if $q=27$ and  the value at $*****$ occurs if and only if $7|q+1$.\\

\bigskip


\end{tetel}

%$d_c(N_G(P),G)=d(N_G(P),G)=5$, $d_c(N_G(M^1),G)=d_c(N_G(M^{-1}),G)=4$, $d(N_G(M^1),G)=d(N_G(M^{-1}),G)=3$, $d_c(N_G(M),G)=d_c(C_G(i),G)=6$, $d(N_G(M),G)=d(C_G(i),G)=3$, and $d_c(G_0, G)=4$, $d(G_0, G)=3$.
%\end{tetel}
\section{General information on Ree groups}
First let us recall some facts about Ree groups, $ ^2G_2(q)=R(q)$.
\begin{tetel}\label{ree}\cite[Ch XI. Thm. 13.2, p. 262.]{Hup}, \cite{W}, \cite{LN},
\cite[L. 2, L. 3]{zll}
Let $q=3^{2n+1}$ ($n\geqslant 1$), then there exist groups $G=R(q)$ of order $(q^3+1)q^3(q-1)$ with the following properties:
 \begin{enumerate}[(a)]
 \item{$G$ is simple.}
 \item{$G$ is doubly transitive  on $\Omega=\{1,\dots, q^3+1\}$. The stabilizer  $G_1$ of the point $1$ is $N_G(P)$, where $P\in Syl_3(G)$. The stabilizer $G_{1,2}$ of  the points $1$ and $2$ is a cyclic group $W$ of order $q-1$ and $N_G(P)=PW$.
 We denote by $W_{2'}$ the $2'$-Hall subgroup of $W$.
 This has order $(q-1)/2$ and $N_G(W_{2'})=N_G(W)$ is dihedral
of order $2(q-1)$.
 If $i$ is the involution in $W$, and $P'$ is the derived subgroup of $P$
  then $C_P(i)=C_{P'}(i)$ is  elementary abelian
of order $q$ and $C_P(i)\cap Z(P)=\{1\}$.
 If $w$ is a nontrivial element of $W$
 of odd order, then $C_P(w)=\{1\}$. There are  subgroups fixing exactly $3$ letters. They have  order
 $2$.}


 \item{Let $S$ be a Sylow $2$-subgroup  of $G$. Then $S$ is  elementary abelian of order $8$.
 Moreover, $C_G(S)=S$ and $|N_G(S)|=8\cdot 7\cdot 3$. The $2$-subgroups
 of the same 
order are conjugate in $G$.}
 \item{$G$ has cyclic Hall subgroups $M^j$ ($j=\pm 1$) of orders
 $q+1+j \cdot 3m$, where $m=3^n$. Each  subgroup $M^j$ is TI and it is
 the centralizer of each of its non-identity elements. The subgroups
 $B^j:=N_G(M^j)$ are Frobenius groups  with  kernel $M^{j}$ and cyclic complement of order $6$.}
 \item For each subgroup $V$ of order $4$ there  exists a cyclic Hall TI-subgroup
 $M$ of order $\frac{q+1}{4}$ and an element $t$ of order $6$ such that $N_G(V)=N_G(M)=V\rtimes (M\rtimes \langle t \rangle)\simeq (V\times (M\rtimes C_2))\rtimes C_3)$,
 $C_G(M)=V\times M$ and $C_G(V)\simeq V\times (M\rtimes C_2)$.
 \item For each nontrivial subgroup $H\leqslant A$, where $A$ is one of the subgroups $W_{2'}$, $M$, $M^{\pm1}$, the containment  $N_G(H)\leqslant N_G(A) $ holds.
 \item{The centralizer in $G$ of an involution $i$ is isomorphic to
 $\langle i\rangle\times PSL_2(q)$.}
\item{A Sylow $3$-subgroup $P$ of $G$ has order $q^3$.  It is a
TI set. Its centre $Z(P)$ is an elementary abelian subgroup of order $q$,
$P$ is of class $3$, and $P'=\Phi(P)$ is an elementary abelian subgroup
  of order
 $q^2$ containing $Z(P)$.
  The elements of $P\backslash P'$ have order $9$, their cubes are forming
$Z(P)\backslash \{1\}$.\\
    $P$ is isomorphic to the group of all triples $(x,y,z)$ with $x,y,z\in GF(q)$ and the following multiplication rule:\\
$(x_1,y_1,z_1)(x_2,y_2,z_2)=(x_1+x_2, y_1+y_2+x_1(x_2\sigma), z_1+z_2-x_1y_2+y_1x_2-x_1(x_1\sigma)x_2).$\\
Here, $\sigma$ is the automorphism of $GF(q)$ such that $x\sigma^2=x^3$ for all $x\in GF(q)$. We have}\\
$P'=\Phi(P)=\{(0,y,z)\mid y,z\in GF(q)\}\mbox{ and }Z(P)=\{(0,0,z)\mid z\in GF(q)\}$
\item The maximal subgroups of $G$ are   the following subgroups up to conjugacy:\\
    $N_G(P)$, $C_G(i)$, $N_G(M^{\pm1})$, $N_G(M)$, $G_0\simeq R(q_0),$\\
    where $q_0^a=q$ for some  prime $a$. See \cite[Theorem C]{Kleidman},
 \cite{LN},\cite[Ch 4.11.3 Thm 4.3]{RW}.

\item $|G|=2^33^{3(2n+1)}{\frac{(q-1)}{2}}{\frac{(q+1)}{4}}(q+1-3^{n+1})(q+1+3^{n+1})$, where any two of the integers $2$, $3$, $(q-1)/2$, $(q+1)/4$, $(q+1-3^{n+1})$ and  $(q+1+3^{n+1})$ are relatively prime. Each cyclic subgroup of $G$ of order $\frac{q+1}{4}$, $q+1-3^{n+1}$
and $q+1+3^{n+1}$, respectively, can be embedded into a cyclic subgroup of order
$\frac{q^3+1}{4}=\frac{q+1}{4}(q+1-3^{n+1})(q+1+3^{n+1})$. 
 \end{enumerate}
\end{tetel}
In the following we will use the notations of Theorem \ref{ree}.
 From  Theorem \ref{ree} 
 some further  properties on the structure of $G$ can be deduced.

\begin{cor}\label{Kl}\begin{enumerate}
\item  All subgroups of order $\frac{q-1}{2}$, $\frac{q+1}{4}$, $q+3^{n+1}+1$, or $q-3^{n+1}+1$ are conjugate  in $G$, respectively.
\item The subgroups of order $q-1$ are conjugate  in $G$.
\item  The involutions of $N_G(P)$ are conjugate in $N_G(P)$.
\end{enumerate}
\end{cor}

We will need the following properties on centralizers of cylic subgroups
of $G$:


\begin{lemma}\label{centralOfqpm1} 
\begin{itemize}
\item[(i)] If $C\neq \{1\}$ is a cyclic subgroup of $G$
 whose order divides $\frac{q-1}{2}$, then
 $C_G(C)\simeq C_{q-1}$ 
\item[(ii)]
If $C\neq \{1\}$ is a cyclic subgroup of $G$
 whose order divides  $\frac{q+1}{4}$, then
 $C_G(C)\simeq C_2^2\times C_{\frac{q+1}{4}}$.
\end{itemize}
\end{lemma}
\begin{proof}(i) Let $C$ be a nontrivial cyclic subgroup of $G$
 whose order divides $\frac{q-1}{2}$.
 Since the subgroups of order $\frac{q-1}{2}$  are Hall-subgroups and cyclic,
 thus
by a result of Wielandt \cite{Wie},  there is a cyclic subgroup $C_{max}$ of
 order $\frac{q-1}{2}$ such that $C\leqslant C_{max}$.
 By Theorem \ref{ree}, (b)  and (f) $C_{q-1}\simeq C_G(C_{max})\leqslant C_G(C)\leqslant N_G(C)\leqslant N_G(C_{max})\simeq D_{2(q-1)}$. Thus $C_G(C)\simeq C_{q-1}$.\\
 (ii) Similarly, if $C$ is a nontrivial cyclic subgroup of $G$ of order
 dividing $\frac{q+1}{4}$, then it is contained in a cyclic subgroup
 $C_{max}$ of order  $\frac{q+1}{4}$, and by Theorem \ref{ree} (e) and (f),
 $V\times C_{max}\simeq C_G(C_{max})\leqslant C_G(C)\leqslant N_G(C)\leqslant N_G(C_{max})\simeq (V\times (C_{max}\rtimes C_2)\rtimes C_3)$. However, by Theorem \ref{ree} (c) $C_G(S)=S$, hence no $2$-element in  $N_G(C_{max})\setminus V$ centralizes $C$. Since the Sylow
 $3$-subgroups of $G$ are TI, if a $3$-element $x$  centralized $C=\langle c \rangle$, then $c$ would normalize a Sylow $3$-subgroup, contradicting  that
$|N_G(P)|=q^3(q-1)$, which is relatively prime to $\frac{q+1}{4}$. Hence
$C_G(C)\simeq C_2^2\times C_\frac{q+1}{4}$.
\end{proof}

%\begin{lemma}\label{centraliserOfqpm3m+1} If $C$ is a nontrivial subgroup
%of order dividing $q\pm 3m+1$ then its centralizer is isomorphic to
%$C_{q\pm 3m+1}$.
%\end{lemma}
%\begin{proof}
%Similarly as above, this subgroup can be embedded to a conjugate of $M^i$
%(in Theorem \ref{ree} (d)). Since $C_G(M^i)=M^i$, which
% is also the centralizer of each of its nontrivial subgroups,
% we get the result.
%\end{proof}


Using the representation of $P\in Syl_3(G)$ in Theorem \ref{ree} (h), we get the following useful results.

\begin{lemma}\label{centralizerof3orderelement} If $p\in P'\setminus Z(P)$ for
 some  $P\in Syl_3(G)$, then $C_P(p)=P'$.
\end{lemma}
\begin{proof}Thus $p=(0,b,c)$ and $b\neq 0$. Then\\
$(0,b,c)(x,y,z)=(x, b+y, c+z+bx)\mbox{ and }
(x,y,z)(0,b,c)=(x, b+y, c+z-xb).$\\
Therefore, $(x,y,z)\in C_P(p)$ if and only if $(x,y,z)\in P'$.
\end{proof}

From this,  Theorem \ref{ree} and from the character table of $N_G(P)$, see below or in
\cite{Landrock}, we deduce the following:

\begin{cor}\label{centrp}
\begin{itemize}
\item [(i)] Let $z\in Z(P)$ be a nontrivial element, then $C_G(z)=P$.
\item [(ii)] Let $y\in P'\setminus Z(P)$. The $C_G(y)=P'\rtimes \langle j \rangle$
, where $j$ is an involution in $G$. In particular, if $T:=\langle y \rangle $
 then
$N_G(T)=C_G(T)$ is of order $2q^2$.
\item[(iii)] Let $x\in P\setminus P'$ then $C_G(x)=Z(P)\langle x\rangle$.
\qedhere
\end{itemize}
\end{cor}
\begin{proof}
\begin{itemize}
\item[(ii)] We have that  $Aut(T)$ is  of order $2$ and $N_G(T)\leq N_G(P)$
for the Sylow $p$-subgroup $P$ containing $T$. Thus $|N_G(T)/C_G(T)|$ divides
$\frac{q^3(q-1)}{2q^2}=q\frac{(q-1)}{2}$, which is odd. Hence $N_G(T)=C_G(T)$.
\end{itemize}
\end{proof}

\begin{lemma}\label{conj} Let $P\in Syl_3(G)$  and $(x,y,z), (a,b,c)\in P$.
 Then $(a,b,c)^{-1}=(-a,-b+a(a\sigma),-c)$ and $(x,y,z)^{(a,b,c)}=(x,y+x(a\sigma)-a(x\sigma), z-2xb+2ya-ax(x\sigma)+ax(a\sigma)$.
\end{lemma}
\begin{proof}
Check that $(a,b,c)^{-1}(a,b,c)=(0,0,0)$:\\
$(-a,-b+a(a\sigma),-c)(a,b,c)=(0,a(a\sigma)+(-a)(a\sigma), -(-a)b+(-b+a(a\sigma))a-(-a)(-a\sigma)a)=
(0,0,a^2(a\sigma)-a^2(a\sigma)=(0,0,0)$.\\%[0,5cm]
$(x,y,z)^{(a,b,c)}=(-a,-b+a(a\sigma),-c)(x+a, y+b+x(a\sigma), z+c-xb+ya-x(x\sigma)a)=$\\
 $=(x,y+a(a\sigma)+x(a\sigma)-a((x+a)\sigma), z-xb+ya-x(x\sigma )a+a(y+b+x(a\sigma))+(-b+a(a\sigma))(x+a)-a(a\sigma)(x+a))=\\=(x,y+x(a\sigma)-a(x\sigma), z-xb+ya-x(x\sigma)a+ ay + ab+ ax (a\sigma)-bx-ba+ax (a\sigma)+a^2(a\sigma) -ax(a\sigma)-a^2(a\sigma))=(x,y+x(a\sigma)-a(x\sigma), z-2xb+2ya-ax(x\sigma) +ax (a\sigma)).$
\end{proof}



\section{The depth of $N_G(P)$}
\begin{prop}\label{3jegystabilizator}  If $\alpha,\beta,\gamma\in\Omega=\{1,\dots, q^3+1\}$ are pairwise different, then $G_{\alpha,\beta,\gamma}$ is isomorphic to $C_2$ or $\{1\}$.
Moreover, for every two different $\alpha, \beta \in \Omega $
 there exist $\gamma, \delta\in\Omega$ such that
 $G_{\alpha,\beta,\gamma}\simeq C_2$ and $G_{\alpha,\beta,\delta}\simeq \{1\}$.
In particular, the standard primitive action on the coset space $G/N_G(P)$ has base size $3$.
\end{prop}
\begin{proof} Since $G$ acts on $\Omega $ doubly transitively,
 it is enough to show, that there exist  $\delta,\iota,\kappa,\lambda\in\Omega$
such that $G_{1,2,\delta}=\{1\}$ and $|G_{\iota,\kappa,\lambda}|=2$. The second statement follows from Theorem \ref{ree}(b).\\
If $G_{1,2,\gamma}\simeq C_2$, then there exists a symbol $\delta$ such that
$G_{1,2,\gamma, \delta}=\{1\}$.
 Otherwise $G_{1,2,\gamma,\delta}=G_{1,2,\gamma}$ for every $\delta$ and so
 $\{1\}\neq G_{1,2,\gamma}\subseteq Stab(1,\dots,q^3+1)$,
 which is a contradiction. Let us choose a $\delta$ such that
 $G_{1,2,\gamma,\delta}=\{1\}$, therefore $G_{1,2,\delta}\neq G_{1,2,\gamma}$ and both are contained in $G_{1,2}\simeq C_{q-1}$.\\
 By Theorem \ref{ree}(b) we know that $G_{1,2,\delta}\simeq C_2$ or $\{1\}$.
 Since $G_{1,2}$ contains  exactly one subgroup of order $2$,
which is $G_{1,2,\gamma}$, thus $G_{1,2,\delta}$ has to be  $\{1\}$.
\end{proof}

\begin{prop} The combinatorial depth of $N_G(P)$ is $5$. %$d_c(N_G(P),G)=5$.
 Moreover, $\mathcal{U}_{\,1}(N_G(P))$ does not contain $\{1\}$ and
$\mathcal{U}_{\,2}(N_G(P))=\mathcal{U}_3(N_G(P))=\{ N_G(P),H\leqslant N_G(P) \mid H\simeq \{1\},C_2$ or $C_{q-1}\}$.
\end{prop}
\begin{proof}
%�sszefoglalni az elej�n a jegystabiliz�torok tulajdons�gait.
 We will use  the condition (2) in Definition \ref{jellemzes}  to prove that $d_c(N_G(P),G)\leqslant 5$.
By Theorem \ref{ree} (b) we have that $G_1=N_G(P)$. Let us examine the following series of subgroups:
\[G_1\geqslant G_1^{(x_1)}\geqslant G_1^{(x_1,x_2)}\geqslant G_1^{(x_1,x_2,x_3)}.\]
Now using the knowledge on the stabilizers, the group
 $G_1^{(x_1,x_2,x_3)}=G_{1, (1)x_1, (1)x_2, (1)x_3}$ is isomorphic  either 
to $1, C_2, W$ or $G_1$.
 Let us consider the $4$ different cases. We will show that in every case we can find elements, $y_1$ and $y_2$ such that $G_1^{(x_1,x_2,x_3)}=G_1^{(y_1,y_2)}$.
\begin{enumerate}[(A)]
\item{If $G_1^{(x_1,x_2,x_3)}=G_1$, then $G_1^{x_i}=G_1$ holds for every $i$.
 Then $y_1=x_1$ and $y_2=x_2$ is a good choice.}
\item{If $|G_1^{(x_1,x_2,x_3)}|=q-1$, then two of the three containments
 in our series  are not  strict. Then let
$y_1:=x_1$ and $y_2:=x_2$, if $G_1\ne G_1^{(x_1)}$ or $G_1^{(x_1)}\neq G_1^{(x_1,x_2)}$,
 otherwise let $y_1:=x_1$ and $y_2:=x_3$.}
\item{If $|G_1^{(x_1,x_2,x_3)}|=2$, then  exactly
 one equality is  in our series.
 Thus there exists an index  $i>1$ such that
 $G_1^{(x_1,\dots,x_{i-1})}=G_1^{(x_1,\dots,x_{i})}$ or $G_1=G_1^{(x_1)}$.
Let $\{y_1,y_2\}:=\{x_1,x_2,x_3\}\setminus \{x_i\}$} in the first case and let
$\{y_1,y_2\}:=\{x_2,x_3\}$ in the second case.
\item{If $G_1^{(x_1,x_2,x_3)}=\{1\}$, then by Proposition
 \ref{3jegystabilizator} there exist $\alpha,\beta$ such
 that $G_{1,\alpha, \beta}=\{1\}$.
 Since $G$ is (doubly) transitive, there are elements
$y_1,y_2\in G$ such that $\alpha =(1)y_1$ and $\beta=(1)y_2$. Thus  $G_1^{(y_1,y_2)}=\{1\}$.}
\end{enumerate}
Now we have to check that  $x_1gx_1^{-1}=y_1gy_1^{-1}$ for any $g\in G_1^{(x_1,x_2, x_3)}$.
This is automatically true except for the case (C)
when $G_1^{x_1}=G_1$.
 Let $G_1^{(x_1,x_2,x_3)}=\langle j\rangle \backsimeq C_2$ and $G_1^{x_1}=G_1$.
 If the originally chosen $y_1,y_2$ are not suitable,
 we modify them in the
 following way. Let us choose $z\in C_G(j)\setminus G_1$, which is possible.
 Let $y_1=x_1z$. Therefore $x_1$ and $y_1$  act in the same way on
 $G_1^{(x_1,x_2,x_3)}$. Since $y_1\notin N_G(G_1)=G_1$, it follows that 
  $G_1\neq G_1^{y_1}$. The subgroup
 $G_1^{(y_1)}=G_{1,(1)y_1}$ is a stabilizer of two different points.
By Proposition $\ref{3jegystabilizator}$ we know that there is a point
 $\delta$ such that $G_{1,(1)y_1, \delta}\backsimeq C_2$.
 Furthermore, $G_{1,(1)y_1, \delta}\subseteq G_{1,(1)y_1}\simeq W$
 and $G_{1,(1)y_1}$ contains only one involution, $j$.
 If we choose an element $y_2$ such that $(1)y_2=\delta$,
 then $G_1^{(y_1,y_2)}=G_{1,(1)y_1, \delta}=\langle j \rangle=G_1^{(x_1,x_2,x_3)}$, and we are done.\\
\indent
To prove that $d_c(N_G(P),G)>4$ we will use condition  (1) in Definition \ref{jellemzes} 
 Let $\beta,\gamma$ such that $G_{1,\beta,\gamma}=1$, which is possible by
 Proposition \ref{3jegystabilizator}. Let $x_1, x_2\in G$ such that $(1)x_1=\beta$, $(1)x_2=\gamma$. Since $G$ is transitive, such $x_1$ and $x_2$ exist.
 Thus ${G_1}^{(x_1,x_2)}=G_{1,\beta,\gamma}=\{1\}$, and by Theorem \ref{ree} (b)
 there is no element $y_1$ such that ${G_1}^{(x_1,x_2)}={G_1}^{(y_1)}$.
 This implies that $d_c(N_G(P),G)=5$.
\end{proof}
Below we present the character table of $N_G(P)$ (see \cite{Landrock}). To shorten our notation, let $\zeta:=1+i \sqrt{3}m$ and $\xi:=\frac{1}{2}(m+\sqrt{3}m\ i)$.\\
The elements $X$, $Y$ and $T$ are fixed elements in $Z(P)$,  $P\setminus P'$ and $ P'\setminus Z(P)$, respectively.  $J$ is the involution in $W$ and the element $R$ is a generator of $W_{2'}$. Furthermore $\epsilon$ is a primitive root of unity of order $\frac{q-1}{2}$ and $a, b\in \mathbb{Z}_{\frac{q-3}{2}}\setminus\{0\}$.
{\small
\[
\begin{array}{l|cccccccccccc}
 &1 & X & Y & T & T^{-1} & YT & YT^{-1} & JT & JT^{-1} & R^a & JR^a & J \\\hline
\mathds{1} & 1 & 1 & 1 & 1 & 1 & 1 & 1 & 1 & 1 & 1 & 1 & 1 \\
\Delta & 1 & 1 & 1 & 1 & 1 & 1 & 1 & -1 & -1 & 1 & -1 & -1 \\
\psi_b^+& 1 & 1 & 1 & 1 & 1 & 1 & 1 & 1 & 1 & \epsilon^{ab} & \epsilon^{ab} & 1 \\
\psi_b^-&1 & 1 & 1 & 1 & 1 & 1 & 1 & -1 & -1 & \epsilon^{ab} & -\epsilon^{ab} & -1 \\
\alpha_1& q-1 & q-1 & -1 & q-1 & q-1 & -1 & -1 & 0 & 0 & 0 & 0 & 0 \\
\alpha_2& (q-1) q & -q & 0 & 0 & 0 & 0 & 0 & 0 & 0 & 0 & 0 & 0 \\

\alpha_3& m (q-1) & m (q-1) & -m & -m \overline{\zeta}  & -m\zeta &  \xi & \overline{\xi} & 0 & 0 & 0 & 0 & 0 \\

\alpha_4& m (q-1) & m (q-1) & -m & -m \zeta  & -m\overline{\zeta} & \overline{\xi } &  \xi & 0 & 0 & 0 & 0 & 0 \\

\alpha_5& \frac{m}{2} (q-1) & \frac{1}{2} m (q-1) & m &  - \frac{1}{2}m \overline{\zeta} & -\frac{1}{2} m \zeta & - \xi& - \overline{\xi} &- \frac{1}{2} \overline{\zeta} & - \frac{1}{2} \zeta & 0 & 0 & \frac{q-1}{2} \\

\alpha_6& \frac{m}{2} (q-1) & \frac{1}{2} m (q-1) & m &  - \frac{1}{2}m \overline{\zeta} & -\frac{1}{2} m \zeta  & - \xi & - \overline{\xi} &  \frac{1}{2} \overline{\zeta} &  \frac{1}{2} \zeta & 0 & 0 &-\frac{q-1}{2} \\

\alpha_7& \frac{m}{2} (q-1) & \frac{1}{2} m (q-1) & m & -\frac{1}{2} m \zeta & -\frac{1}{2} m \overline{\zeta } & -\overline{\xi} & -\xi & - \frac{1}{2} \zeta & - \frac{1}{2} \overline{\zeta} & 0 & 0 & \frac{q-1}{2} \\

\alpha_8& \frac{m}{2} (q-1) & \frac{1}{2} m (q-1) & m & -\frac{1}{2} m \zeta & -\frac{1}{2} m \overline{\zeta}  & - \overline{\xi} & - \xi & \frac{1}{2} \zeta &  \frac{1}{2} \overline{\zeta} & 0 & 0 & -\frac{q-1}{2}
\end{array}
\]}
 The centralizers of the elements can be found in \cite{Landrock}. Using  them we can  obtain that $|Cl(X)|=q-1$, $|Cl(Y)|=\frac{1}{3}q^2(q-1)$, $|Cl(T)|=|Cl(T^{-1})|=\frac{1}{2}q(q-1)$, $|Cl(YT)|=|Cl(YT^{-1})|=\frac{1}{3}q^2(q-1)$, $|Cl(JT)|=|Cl(JT^{-1})|=\frac{1}{2}q^2(q-1)$, $|Cl(R^a)|=|Cl(JR^{a})|=q^3$, and $|C(J)|=q^2$.\\

To compute the induced characters $\mathds{1}^{G},$ $\Delta^G$ and ${\psi_b^{+}}^{G}$, ${\psi_b^{-}}^{G}$ we need the following lemma.
\begin{lemma}
\begin{enumerate}[a)]
\item Let $p\in P$. For an element $x\in G$, the element  $p^{x}$ is in $N_G(P)$ if and only if $x$ is in $N_G(P)$.
\item Let $i$ be an involution in $N_G(P)$. For an element $x\in G$, the element $i^x$ is in $N_G(P)$ if and only if $x$ is in $C_G(i)N_G(P)$.
\item Let $w\in W\setminus \{i\}$. For an element $x\in G$, the element $w^x$ is in $N_G(P)$ if and only if $x$ is in $N_G(W)N_G(P)$.
\item Let $i$ be an involution in $N_G(P)$ and $p\in P$ such that $o(ip)=6$. For an element $x\in G$, the element $(ip)^x$ is in $N_G(P)$ if and only if $x$ is in $N_G(P)$.
\end{enumerate}
\end{lemma}
\begin{proof}
\begin{enumerate}[a)]
\item Let assume that $p\in P, P^{x^{-1}}$. Since the Sylow $3$-subgroups in $G$ are TI, we get that $x\in N_G(P)$. The other direction is trivial.
\item Let assume that $i,i^{x}\in N_G(P)$. Since the involutions in $N_G(P)$ are conjugate, there is an element $n\in N_G(P)$ such that $(i^{x})^{n}=i$. Thus $xn\in C_G(i)$, or equivalently $x\in C_G(i) N_G(P)$. The other direction is trivial.
\item  Without loss of generality we can suppose that $w\in W_{2'}$ and $w^x\in N_G(P)$. Otherwise, we raise $w$ into a suitable $2$-power. Since $W_{2'}$ is a Hall subgroup of order $\frac{q-1}{2}$, by a result of Wielandt \cite{Wie}, we have that $W_{2'}$, $W_{2'}^x$ are conjugate in $N_G(P)$.
    Thus, $w,(w^x)^n\in W_{2'}$ for some $n\in N_G(P)$. 
   Since $W_{2'}$ is cyclic,  by Theorem \ref{ree} (f), and (b) we get that
    $xn\in N_G(\langle w\rangle)\leqslant N_G(W_{2'})=N_G(W).$
    Thus $x\in N_G(W)N_G(P)$. The other direction is trivial.
\item By  assumption,  $(ip)^2\in P$.  On the other hand, $((ip)^2)^x\in N_G(P)$, thus by using part a), we have that $x\in N_G(P)$. The other direction is trivial.
\qedhere
\end{enumerate}
\end{proof}Using the previous Lemma we get the following:
\begin{cor}\label{4} The characters $\mathds{1}^G\mid_{N_G(P)}$, $\Delta^G\mid_{N_G(P)}$, $(\psi_b^+)^G\mid_{N_G(P)}$ and $(\psi_b^-)^G\mid_{N_G(P)}$ have the following values:
\[\begin{array}{l|cccccccccccc}
 \mathds{1^G}|_{N_G(P)}& q^3+1 & 1 & 1 & 1 & 1 & 1 & 1 & 1 & 1 & 2 & 2 & q+1\\
 {\Delta^G}|_{N_G(P)}    &q^3+1 & 1 & 1 & 1 & 1 & 1 & 1 & -1 & -1 & 2 & -2 & -q-1\\
 {(\psi_b^{+})^G}|_{N_G(P)} &q^3+1 & 1 & 1 & 1 & 1 & 1 & 1 & 1 & 1 & 2\epsilon^{ab} & 2\epsilon^{ab} & q+1\\
 {(\psi_b^{-})^G}|_{N_G(P)} &q^3+1 & 1 & 1 & 1 & 1 & 1 & 1 & -1 &- 1 & 2\epsilon^{ab} & -2\epsilon^{ab} & -q-1
\end{array}.
\]
\end{cor}
\begin{prop} The ordinary depth of $N_G(P)$ is $5$.
\end{prop}
\begin{proof}
We determined the irreducible constituents of $\mathds{1}^G|_{N_G(P)}-2(\mathds{1}|_{N_G(P)})$, $\Delta^G|_{N_G(P)}-2\Delta$, $(\psi_b^{+})^G|_{N_G(P)}-2 \psi_b^{+}$ and $(\psi_b^{-})^G|_{N_G(P)}-2 \psi_b^{-}$. Since these characters have zero values where the different $\psi_b^+$, (or $\psi_b^{-}$) differ from each other,
 we get that the  multiplicity of $\psi^+_{b'}$  (or of $\psi^-_{b'}$) in these  four 
characters is independent of $b'$. 
%To simplify calculations, let us introduce
%$\Psi^+:=\sum_{b=1}^{\frac{n-3}{2}} \psi_b^+$ and $\Psi^-:=\sum_{b=1}^{\frac{n-3}{2}} \psi_b^-.$\\
% The values of these characters are as follows
%\[\begin{array}{l|cccccccccccc}
%\Psi^+& \frac{q-3}{2} & \frac{q-3}{2} & \frac{q-3}{2} & \frac{q-3}{2} & \frac{q-3}{2}& \frac{q-3}{2} & \frac{q-3}{2} & \frac{q-3}{2} & \frac{q-3}{2} & -1 & -1 & \frac{q-3}{2} \\
%\Psi^-& \frac{q-3}{2} & \frac{q-3}{2} & \frac{q-3}{2} & \frac{q-3}{2} & \frac{q-3}{2}& \frac{q-3}{2} & \frac{q-3}{2} & -\frac{q-3}{2} & -\frac{q-3}{2} & -1 & 1 & -\frac{q-3}{2}
%\end{array}.
%\]

%The scalar product of $\psi_b^{+}$ with the  above mentioned   four characters  
% is the same as
% the scalar product of $\Psi^+$ divided by $\frac{q-3}{2}$. The minus version
%can be calculated similarly.
With the help of this, we determined the irreducible constituents of the characters
in Corollary \ref{4}.

The computed results are the following:




\begin{align*}
&\mathds{1}^{G}|_{N_G(P)}=2\, \mathds{1}|_{N_G(P)}+\alpha_1+q\alpha_2+m\alpha_3+m\alpha_4+\frac{m+1}{2}\alpha_5+\frac{m-1}{2}\alpha_2+\frac{m+1}{2}\alpha_7+\frac{m-1}{2}\alpha_8,\\
&\Delta^{G}|_{N_G(P)}=2 \, \Delta+\alpha_1+q\alpha_2+m\alpha_3+m\alpha_4+\frac{m-1}{2}\alpha_5+\frac{m+1}{2}\alpha_6+\frac{m-1}{2}\alpha_7+\frac{m+1}{2}\alpha_8,\\
&(\psi_b^+)^{G}|_{N_G(P)}=2 \, \psi_b^++\alpha_1+q\alpha_2+m\alpha_3+m\alpha_4+\frac{m+1}{2}\alpha_5+\frac{m-1}{2}\alpha_6+\frac{m+1}{2}\alpha_7+\frac{m-1}{2}\alpha_8,\\
&(\psi_b^-)^{G}|_{N_G(P)}=2 \, \psi_b^-+\alpha_1+q\alpha_2+m\alpha_3+m\alpha_4+\frac{m-1}{2}\alpha_5+\frac{m+1}{2}\alpha_6+\frac{m-1}{2}\alpha_7+\frac{m+1}{2}\alpha_8.
\end{align*}

%\[\mathds{1}^{G}|_{N_G(P)}=2 (\mathds{1}|_{N_G(P)})+\alpha_1+q\alpha_2+m\alpha_3+m\alpha_4+\frac{m+1}{2}\alpha_5+\frac{m-1}{2}\alpha_2+\frac{m+1}{2}\alpha_7+\frac{m-1}{2}\alpha_8,\]
%\[\Delta^{G}|_{N_G(P)}=2 \Delta+\alpha_1+q\alpha_2+m\alpha_3+m\alpha_4+\frac{m-1}{2}\alpha_5+\frac{m+1}{2}\alpha_6+\frac{m-1}{2}\alpha_7+\frac{m+1}{2}\alpha_8,\]
%\[(\psi_b^+)^{G}|_{N_G(P)}=2 \psi_b^++\alpha_1+q\alpha_2+m\alpha_3+m\alpha_4+\frac{m+1}{2}\alpha_5+\frac{m-1}{2}\alpha_6+\frac{m+1}{2}\alpha_7+\frac{m-1}{2}\alpha_8,\]
%\[(\psi_b^-)^{G}|_{N_G(P)}=2 \psi_b^-+\alpha_1+q\alpha_2+m\alpha_3+m\alpha_4+\frac{m-1}{2}\alpha_5+\frac{m+1}{2}\alpha_6+\frac{m-1}{2}\alpha_7+\frac{m+1}{2}\alpha_8.\]

 The distance in $G$ between $\beta,\gamma\in \Irr(N_G(P))$ is $1$
  if and only if $[\beta^G|_{N_G(P)},\gamma]\neq 0$. Hence the distance between arbitrary elements of $\{\mathds{1},\Delta,\psi_b^+,\psi_b^-\}$ and arbitrary elements of $\{\alpha_i\}_{i=1}^8$ is $1$. In particular, the distance between two irreducible characters of $N_G(P)$ is at most $2$. Thus by condition (i) in
 Definition \ref{distance} we get that $d(N_G(P),G)\leq 5$. 
Clearly $d(\mathds{1},\Delta)=2$. Moreover,\\
$m(\mathds{1}_G)=\max_{\alpha\in \Irr(N_G(P))} \min_{\chi\in \{\mathds{1|_{N_G(P)}}\}} d(\alpha,\chi)=\max_{\alpha \in \Irr(N_G(P))}  d(\alpha,\mathds{1|_{N_G(P)}})=2.$
Thus by  condition (ii) in  Defintion \ref{distance} we get that $d(N_G(P),G)=5$.
\end{proof}

\section{The depth of $N_G(M^1)$ and $N_G(M^{-1})$}
%\begin{lemma}We have, that $d_c(M^{1},G)=d_c(M^{-1},G)=3$.
%\end{lemma}
%
%\begin{proof}As we have seen in Theorem \ref{ree}(d), $M^1,M^{-1}$ are disjoint
% from their conjugates, from which we obtain the
%statement by \cite[Theorem 3.12 (b),(c)]{BDK}.
%\end{proof}

%Let $B^{i}=N_G(M^i)$ for $i=\pm 1$.


\begin{prop}\label{4.1}We have that $d_c(B^{1},G)=d_c(B^{-1},G)=4$ and $d(B^{1},G)=d(B^{-1},G)=3$. Moreover, $\mathcal{U}_{\,1}(B^{\pm 1})=\{B^{\pm 1}, \{1\}\} \cup U$, where
$U\subseteq \{H\leqslant B^{\pm 1} \mid H\simeq C_k, k=2,3,6\}$.
\end{prop}
\begin{proof}
We prove the statement for $B^1$, the proof for $B^{-1}$ is similar.\\
We use condition (1) in  Definition \ref{jellemzes}  to prove $d_c(B^1,G)\leq 4$.\\
If $x_1\in B^1$, then $(B^1)^{(x_1)}=B^1$. If $x_1\notin B^1$, then
$(B^1)^{(x_1)}$ is a subgroup of a Frobenius complement, since $M^{1}$ is TI
and $B^1=N_G(B^1)=N_G(M^1)$. Thus,  the subgroup $(B^1)^{(x_1, x_2)}$ is
 either
 $B^1$ or isomorphic to one of  $\{1\}$, $C_2$, $C_3$ and $C_6$.
\begin{enumerate}[(A)]
\item{If $(B^1)^{(x_1, x_2)}=B^1$, then let $y_1:=x_1$, and we are done.}

\item{Now we examine the case when $(B^1)^{(x_1, x_2)}=\{1\}$.
 We have to show that there is an element $y_1$ such that $(B^1)^{(y_1)}=\{1\}$.}\\
We will compute, how many  elements $y$ exist
 such that $(B^1)^{(y)}$ contains an involution or a  $3$-element.
First we prove the following:
\begin{lemma}\label{23}
If $T\subseteq B^1\cap (B^1)^{y}$ is a cyclic subgroup of order
$2$  or $3$ 
 then $y\in B^1N_G(T)$.
\end{lemma}
\begin{proof}
Assume that $T\subseteq B^1\cap (B^1)^{y}$ is a cyclic subgroup of order $2$ or $3$.
 Then there is  a subgroup $T_1\leq B^{1}$ such that $T_1^y=T$.
Since $T_1$ and $T$ are  contained in some conjugates of
 a Frobenius complement of $B^{1}$,
 there exists an element $b\in M^{1}$ such that $T_1^b=T$. Therefore $b^{-1}y\in N_G(T)$ and $y\in B^1N_G(T)$.
\end{proof}
\begin{itemize}
\item{Now we  determine at most how many $y$ exist,
 such that $(B^1)^{(y)}$ contains an involution. By Lemma \ref{23} we have:}\\
\textit{If the involution $i$ is an element of $B^1\cap (B^1)^{y}$, then $y\in B^1C_G(i)$.}\\
%To prove this: if $i\in B^1\cap (B^1)^{y}$, then there exists an element
% $i_1\in B^{1}$ such that $i_1^y=i$.
% Since $i$, $i_1$ are contained in some conjugates of the Frobenius complement of
%$B^{1}$,  there is an element
% $b\in M^1$ such that $i_1=i^{b}$.
% Thus, $by\in C_G(i)$.
\end{itemize}
Let us estimate the number of elements $y$:
\begin{align*}
|\{y\ \mid \exists i\in (B^1)^{(y)}, o(i)=2\}|&\leqslant \sum_{\{i\in B^1 \mid o(i)=2\}
}|B^1C_G(i)|\leqslant 
|M^1||B^1C_G(i)|=\\&=(q+1+3m)^2(q-1)(q+1)q.
\end{align*}
\begin{itemize}
\item{Now we will determine, how many elements $y$ exist,
 such that $(B^1)^{(y)}$  contains a $3$-element. By Lemma \ref{23} we have:}\\
\textit{If $T\subseteq B^1\cap (B^1)^{y}$ and $|T|=3$ ,
 then $y\in B^1N_G(T)$.}\\
%To prove this let us assume that $T\subseteq B^1\cap (B^1)^{y}$.
% Then there is  a subgroup $T_1\leq B^{1}$ such that $T_1^y=T$.
%Since $T_1$ and $T$ are  contained in some conjugates of
% the Frobenius complement of $B^{1}$,
% there exists an element $b\in M^{1}$ such that $T_1^b=T$. Therefore $b^{-1}y\in N_G(T)$ and $y\in B^1N_G(T)$.
\end{itemize}
%Since every element of $N_G(T)$ normalizes the Sylow $3$-subgroup
 % containing $T$, we obtain that $|N_G(T)|$ divides $|N_G(P)|=q^3(q-1)$.
  By Cor. \ref{centrp} (ii), $|N_G(T)|=|C_G(T)|=2q^2$.\\
 %Since $\frac{q-1}{2}$ is odd, we get that $|N_G(T)|=2q^2$.\\
  Therefore, \begin{align*}
|\{y \mid 3\mid  |B^1\cap(B^1)^{y}|\}|&\leqslant |\cup_{\{T\leqslant B^1 \mid  |T|=3\}}B^1N_G(T
)|\leqslant |M^1||B^1N_G(T)|\\&=2(q+1+3m)^2q^2.
\end{align*}
Now  some element $y$ must exist such that $(B^1)^{(y)}=\{1\}$, since:
\begin{align*}
|\{y \mid (B^1)^{(y)}=1\}|&\geqslant |G|-|\{y \mid \exists i:\ i\in(B^1)^{(y)},\ o(i)=2\}|-|\{y \mid \exists T:\ T\ \leqslant (B^1)^{(y)},\ |T|=3\}|\\&\geqslant 
q^3(q^3+1)(q-1)-(q+1+3m)^2(q-1)(q+1)q-2(q+1+3m)^2q^2>0 
\end{align*}
if $q\geqslant 27.$
\item{Now we can assume that $(B^1)^{(x_1, x_2)}$ is nontrivial and cyclic.}\\
Let $x_i':=\{x_1,x_2\}\backslash {x_i}$ for $i=1,2$.
If $x_i\in\{x_1,x_2\}$ and it
 satisfies  $(B^1)^{(x_i)}=B^1$, then
 $(B^{1})^{(x_1,x_2)}=(B^{1})^{(x_{i}')}$ holds.
 Otherwise, both $(B^1)^{(x_1)}$ and $(B^1)^{(x_2)}$ are subgroups
 of  some Frobenius complements of $B^1$.
 Since the Frobenius complements are TI sets, the intersection of $(B^1)^{(x_1)}$ and $(B^1)^{(x_2)}$  is either trivial, or one of them contains the other,
and  the intersection is the smaller one.
 \end{enumerate}
Since in each case of, (A)-(C), we can choose an element $y$ such
that $(B^{1})^{(x_1,x_2)}=(B^{1})^{(y)}$, hence $d_c(B^{1},G)\leqslant 4$.

Now we will show an example, where there is no convenient element $y$,
  acting  the same way as $x_1$ on the intersection.
 Let $i$ be an involution in $B^{1}$ and let us choose an element
 $x_1\in M^1\setminus C_G(i)$.
 Let $x_2\in C_G(i)\setminus B^1$. Thus
$i\in (B^1)^{(x_1,x_2)}\neq B^1.$
Let us suppose by contradiction that an element $y$ is suitable:
 $(B^1)^{(x_1,x_2)}=(B^1)^{(y)}$ and $i^{x_1}=i^y$ hold.\\
Then $i\in (B^1)^y$ and hence $i_1=i^{y^{-1}}\in B^1$.
  Since the product of two involutions of $B^1$ is in $M^1$,
thus
$(i_1i)^y\in (M^1)^y.$
On the other hand, $(i_1i)^y=i i^y=i i^{x_1}\in M^1$.
Since $M^{1}$ is TI and $y\not \in B^{1}=N_G(M^{1})$, we have that 
$(i_1i)^y=ii^y\in (M^1)^y\cap M^1=\{1\}.$
Therefore, $y\in C_G(i)$, which is a contradiction, since $x_1\notin C_G(i)$, however  $i^y=i^{x_1}$.\\
Thus, by part (2) of Definition \ref{jellemzes}, $d_c(B^1,G)=4$ and similarly $d_c(B^{-1},G)=4$.
\\


We have seen that there exist elements $x_1,x_2\in G$ such that $(B^1)^{(x_1)}=(B^{-1})^{(x_2)}=\{1\}$. Since $G$ is simple, by Theorem \ref{inter} we get that $d(B^{1},G)=d(B^{-1},G)=3$.
\end{proof}


\section{The depth of $N_G(M)$}

In this section $M$ will be a fixed cyclic subgroup of $G$
 of order $\frac{q+1}{4}$. Let  $V$ be a Klein $4$-subgroup commuting
with $M$. Then $C_G(M)=V\times M$ and
$N_G(M)=N_G(V)\simeq V\rtimes(M\rtimes C_6)\simeq (V\times (M\rtimes C_2))\rtimes C_3$, and $C_G(V)\simeq V\times (M\rtimes C_2)$


\begin{lemma}\label{nincsq+1/4es} If there is a nontrivial element $m\in N_G(M)^{(x_1)}$ whose order divides $\frac{q+1}{4}$, then $x_1\in N_G(M)$ i.e. $N_G(M)^{(x_1)}=N_G(M)$.
\end{lemma}
\begin{proof} Recall that $(|N_G(M):M|,|M|)=1$. This means that $m$ is contained in $M\cap M^{x_1}$. Therefore, by Lemma \ref{centralOfqpm1}, we have that 
 $M,M^{x_1}\leqslant C_G(m)\simeq C_{\frac{q+1}{4}}\times C_2^2$  and
hence 
$M=M^{x_1}.$
\end{proof}
\begin{prop}
\label{NGMu1e}  Let $V$ be as above.  Then
 $\mathcal{U}_{\,1}(N_G(M))=\{N_G(M), S, H\leqslant N_G(M)\ \mid  V\leqslant S\in Syl_2(G),\ [H,V]=1,\ V\neq H\simeq C_2^2 \}\cup U$, where
 $U\subseteq\{H\leqslant N_G(M) \mid H\simeq \{1\}\ ,\ H\simeq C_2, \ C_3 \mbox{ or }C_6\}$.
\end{prop}
\begin{proof}
We know from Theorem \ref{ree} (c) that the Klein subgroups of $G$ are conjugate.
 By Lemma \ref{nincsq+1/4es} we have that 
 $N_G(M)^{(x_1)}$ is isomorphic to one of the following groups:\\
 $ 1, C_2, C_3,C_2^2, C_6,C_2^3, (C_2^2)\rtimes C_3, (C_2^3)\rtimes C_3, N_G(M).$\\
It is obvious that $N_G(M)$ occurs.
  Let us examine the other cases.
\begin{itemize}
\item[(A)]{\bf  $C_2^3\lesssim N_G(M)^{(x_1)}$ :} \textbf{We  prove  the following: if
 $N_G(M)^{(x_1)}$ is a proper subgroup of $N_G(M)$ and contains a subgroup
 isomorphic to $C_2^3$, then $V$ is contained in $N_G(M)^{(x_1)}$ and
$N_G(M)^{(x_1)}$ is isomorphic to $C_2^3$.}
Let $S$ be a subgroup of $N_G(M)^{(x_1)}$ of order $8$ and assume that
$x_1\notin N_G(M)$.
 Then $S\in Syl_2(G)$ and  since $N_G(M)=N_G(V)$, we have that $V,V^{x_1}\leqslant S$.
 Thus $V$ and $V^{x_1}$ contain exactly one common involution,
 which we denote by $i$. Suppose that $h\in N_G(M)^{(x_1)}$ is
 an element of order $3$. Then obviously $h$ acts on $V$ and $V^{x_1}$ nontrivially, so the action is transitive on the involutions.
 This is a contradiction, since $i$ is the unique common involution of $V$ and $V^{x_1}$.  Thus, by Lemma \ref{nincsq+1/4es}, $N_G(M)^{(x_1)}$ can contain only $2$-elements, hence $S=N_G(M)^{(x_1)}$.
\\
\textit{Construction:} Let $S\in Syl_2(G)$  containing $V$.
We want to show that there exists an element
$x_1\notin N_G(M)=N_G(V)$ such that $N_G(M)^{(x_1)}=S$. Choose  an element
$x_1\in G$ such that $V\neq V^{x_1}\leqslant S$. This is possible by Theorem \ref{ree} (c).  Then $S\leqslant N_G(V)^{(x_1)}=N_G(M)^{(x_1)}$
and we are done due to the previous statement.

\item[(B)] $C_2^2\lesssim N_G(M)^{(x_1)}$: \textbf{We will prove
that if $ N_G(M)^{(x_1)}\ne N_G(M)$  and it contains a Klein $4$-subgroup $H$,
but it does not
  contain a Sylow $2$-subgroup of $G$, then $N_G(M)^{(x_1)}=H$ and
$[H,V]=1$, moreover $V\ne H$.
 }\\
    By  assumption and Lemma \ref{nincsq+1/4es},
$N_G(M)^{(x_1)}$ is isomorphic  either to $C_2^2$ or $C_2^2\rtimes C_3$.
    Suppose now that $N_G(M)^{(x_1)}=H\rtimes \langle h\rangle$, where $H\simeq C_2^2$
 and $o(h)=3$. Thus $h$ acts on $H, V, V^{x_1}$ nontrivially.\\
    Since $H\leqslant N_G(V)=N_G(M)$, $H=V$ or $\langle H,V\rangle\in Syl_2(G)$.
Thus $H=V$, because otherwise $h$  would not act nontrivially on  both
 $H$ and $V$. Furthermore, $H=V^{x_1}$ due to the same explanation,
  which is a contradiction, since $x_1\not\in N_G(V)$. Thus $N_G(M)^{(x_1)}\not\simeq C_2^2\rtimes C_3$.\\
    Now we prove  that $N_G(M)^{(x_1)}\ne V$.
 If  $V\leqslant N_G(M)^{(x_1)}=N_G(V)^{(x_1)}$, then
 $C_2^3\simeq\langle V,V^{x_1}\rangle\leqslant N_G(V)^{(x_1)}=N_G(M)^{(x_1)}$, which is a contradiction.\\
\textit{Construction:}
    Let $H$ be a Klein subgroup of $G$
 different from $V$,  with $[H,V]=1$. Then $\langle H,V\rangle\in Syl_2(G)$.
 Let $H\cap V=\langle i\rangle$, and let $j$ be another generator of $V$.
 Then $ V\leqslant C_G(H)= H\times D$, where $D\simeq D_{\frac{q+1}{2}}$.
We may assume that $j\in D$, otherwise we  choose  the complement
 $\langle C_{\frac{q+1}{4}},j\rangle$ instead of $D$.
 Let $k\in D$ be an involution different from $j$.
Choose  an element $x_1\in G$ such that $V^{x_1}=\langle i,k \rangle$.
 Then $[V,V^{x_1}]\ne 1$.  The subgroup $N_G(V)^{(x_1)}=N_G(M)^{(x_1)}$  does 
not contain a
Sylow $2$-subgroup  $S$ of $G$, otherwise $V,V^{x_1}\leq S$.  However,
 it contains $H$. Hence $N_G(M)^{(x_1)}=H$.
\item[(C)] $1, C_2, C_3, C_6$: $N_G(M)^{(x_1)}$ can be isomorphic to $1$, $C_2$, $C_3$ or $C_6$. To compute the depth of $N_G(M)$, however, we do not need to determine exactly which subgroups can occur. So we omit its proof.
\qedhere
\end{itemize}
\end{proof}

\begin{lemma}\label{geq5} $\mathcal{U}_{\;2}(N_G(M))=\{1,C,K,S,N_G(M) \mid C\simeq C_2, K\simeq C_2^2, S\simeq C_2^3 $ and each of them commutes with $ $V$ \}\cup \{X\cap Y \mid  X,Y\in U\}$, where $U$ is as in Prop \ref{NGMu1e}.  Thus $\mathcal{U}_{\,1}(N_G(M))\ne \mathcal{U}_{\,2}(N_G(M))$, since $V$ is in the difference set. Hence we have that
$d_c(N_G(M),G)\geqslant 5$.

\end{lemma}
\begin{proof}
If $Z\in \mathcal{U}_{\,2}(N_G(M))$,
then $Z=X\cap Y$ where $X,Y\in \mathcal{U}_{\,1}(N_G(M))$.
 Therefore we will study the intersection of subgroups of different
 structure from $\mathcal{U}_1(N_G(M))$ in a $\mathcal{U}_1-\mathcal{U}_1$ table.
 The elements of the first the row will be the possible structures
 of $X$ and similarly the elements of the first column will be the possible structures of $Y$. These can be $N_G(M)$;  a Sylow $2$-subgroup $S$ containing $V$;  a Klein subgroup $H$, which  commutes with $V$, but not equal to it; some subgroups isomorphic to $C_6$, $C_3$, $C_2$ or $1$ (if they occur at all, we denote them by $Z_6,Z_3,Z_2,Z_1$). The element in the position $(X,Y)$ in the table will show, which structure can occur as $X\cap Y$.
\begin{center}
\begin{tabular}{|c|ccccccc|}
  \hline
  % after \\: \hline or \cline{col1-col2} \cline{col3-col4} ...
   & $N_G(M)$ & $S$ & $ {H}$ & $Z_6$ & $Z_3$ & $Z_2$ & $Z_1$ \\\hline

$N_G(M)$ & $N_G(M)$ & $S$ & $ {H}$ &  $Z_6$ & $Z_3$ & $Z_2$ & $1$\\

$S$   &  $S$ & $S$, ${  V}$ & $ {H}$, ${ C_2}$ & ${ C_2}$, ${ 1}$  &${ 1}$ &  $Z_2$, $1$ & $1$ \\

$ {H}$ & $ {H}$ & $ {H}$, ${ C_2}$ & $ {H}$,${ C_2}$, ${ 1}$ & ${ C_2}$, ${ 1}$  &${ 1}$ &  $Z_2$, ${ 1}$ & $1$ \\

$Z_6$   & $Z_6$ & ${ C_2}$, ${ 1}$ & ${ C_2}$, ${ 1}$ & \multicolumn{4}{c|}{\multirow{4}{*}{$X\cap Y$, where $X,Y\in U$}}   \\
$Z_3$   & $Z_3$ & ${ 1}$ & ${ 1}$ &\multicolumn{4}{c|}{}  \\
$Z_2$   & $Z_2$ & $Z_2$, ${ 1}$ & $Z_2$,${ 1}$ & \multicolumn{4}{c|}{}  \\
$Z_1$     & $1$ & $1$ &$1$  &\multicolumn{4}{c|}{}  \\
  \hline
\end{tabular}
\end{center}
It is obvious that $V$ occurs as the intersection of two Sylow $2$-subgroups containing it. Thus $V\in \mathcal{U}_{\,2}(N_G(M))$. The other Klein subgroups, 
which commute with $V$, occur already in $\mathcal{U}_{\,1}(N_G(M))$.\\
To finish the proof, we have to show, that every
  cyclic subgroup of order $2$,  which commutes with $V$, and
also $\{1\}$  occur in $\mathcal{U}_{\,2}(N_G(M))$.\\

A subgroup of order $2$ in $V$ occurs in $\mathcal{U}_{\,2}(N_G(M))$:
% since we can choose two different Klein subgroups   which are not equal to $V$%, and commuting with it,   such that their intersection is this subgroup.
Let $\langle k\rangle$ be a subgroup of order $2$
 which commutes with $V$ but is  not contained in it.
 Let $V=\langle i,j\rangle$. Then $K_1=\langle i,k\rangle$ and $K_2=\langle j,k \rangle$ both are in $\mathcal{U}_{\,1}(N_G(M))$. Thus their intersection
$\langle k\rangle$ is in $\mathcal{U}_{\,2}(N_G(M))$.\\
To see that $\{1\}\in \mathcal{U}_{\,2}$, we take two different Sylow $2$-subgroups $S_1,S_2\leqslant N_G(M)$
containing $V$.
Let $V=\langle i,j \rangle$, and let $K_1\leq S_1 $ be of order $4$ with
 $K_1\cap V=\langle i \rangle$.
Let $K_2\leq S_2 $ be of order $4$ with $K_2\cap V=\langle j \rangle$.
Then  $K_1,K_2\in \mathcal{U}_1(N_G(M))$, thus $\{1\}=K_1\cap K_2 \in
\mathcal{U}_{\,2}(N_G(M))$.
\qedhere
\end{proof}

We  remark that every cyclic subgroup of order $6$, which is contained in
 $\mathcal{U}_{\,2}(N_G(M))$, is already  in $\mathcal{U}_{\,1}(N_G(M))$.

\begin{lemma}\label{leq6}$\mathcal{U}_{\,2}(N_G(M))=\mathcal{U}_{\,3}(N_G(M))$. Thus, $d_c(N_G(M),G)\leq 6$.
\end{lemma}
\begin{proof} Similarly to the previous case, if $Z$ in $\mathcal{U}_{\,3}(N_G(M))$, then $Z=X\cap Y$, where $X\in\mathcal{U}_{\,2}(N_G(M)), \\
Y\in\mathcal{U}_{\,1}(N_G(M))$. We demonstrate the possible intersections in a  $\mathcal{U}_1-\mathcal{U}_2$   table. The first column shows the possible values $Y$ of $\mathcal{U}_1(N_G(M))$, as above. The first row  shows the possible values $X$ of $\mathcal{U}_{\,2}(N_G(M))$. They can be $N_G(M)$,
 the $2$-subgroups which commute with $V$: $C_2^3$, $C_2^2$, $C_2$; $1$; and maybe some subgroups, which are isomorphic to $C_6$ or $C_3$,  denoted by $\zeta_6$ and $\zeta_3$.
  Note that  the $2$-subgroup in $Z_6$ commutes with $V$. We have seen above that   $\zeta_6=Z_6$.
\begin{center}
\begin{tabular}{|c|ccccccc|}
  \hline
  % after \\: \hline or \cline{col1-col2} \cline{col3-col4} ...
   & $N_G(M)$ & $C_2^3$ & $C_2^2$ & $C_2$ & $1$ & $Z_6$ & $\zeta_3$  \\\hline

$N_G(M)$ & $N_G(M)$ & $C_2^3$ & $C_2^2$ &  $C_2$ & $1$ & $Z_6$ & $\zeta_3$ \\
$S$   &  $C_2^3$ & $C_2^3$, $C_2^2$ & $C_2^2$, $C_2$ & $C_2$, $1$  &$1$ &  $C_2$, $1$ & $1$  \\

$ {H}$ & $C_2^2$ & $C_2^2$, $C_2$ & $C_2^2$,$C_2$, $1$ & $C_2$, $1$  &$1$ &  $C_2$, $1$ & $1$ \\

$Z_6$   & $Z_6$ & $C_2$, $1$ & $C_2$, $1$ &
 $C_2$, $1$ & $1$ &$Z_6$, $\zeta_3$,$C_2$, $1$ & $\zeta_3$, $1$ 
 \\
$Z_3$   & $Z_3$ & $1$ & $1$ & $1$ &$1$ &$Z_3$, $1$&$Z_3$, $1$ \\
$Z_2$   & $C_2$ & $C_2$, $1$ & $C_2$,$1$ & $C_2$,$1$ & $1$ &$C_2$,$1$ & $1$ \\
$Z_1$     & $1$ & $1$ &$1$  & $1$& $1$ & $1$ &$1$ \\
  \hline
\end{tabular}
\end{center}
The only new intersection could be isomorphic to $C_3$ if $X$ and $Y$ are isomorphic to  cyclic subgroups of order $6$. In this case both $X$ and $Y$ are in $\mathcal{U}_1(N_G(M))$ and hence the intersection is  in $\mathcal{U}_2(N_G(M))$.
\end{proof}
\begin{tetel}$d_c(N_G(M), G)=6$
\end{tetel}
\begin{proof}
By Lemmas \ref{geq5} and  \ref{leq6},  $5\leq d_c(N_G(M),G)\leq 6$. 
 We show elements $x_1,x_2,x_3$ in $G$, such that one cannot find 
  any  $y_1,y_2\in G$ such that $N_G(M)^{(x_1,x_2,x_3)}=N_G(M)^{(y_1,y_2)}$
 and  the elements $x_1,y_1$ act  on $N_G(M)^{(x_1,x_2,x_3)}$ in the same way.
Let $a,t\in C_G(V)$ such that 
$C_G(V)= V\times (M\rtimes \langle t \rangle )$,  where  $M=\langle a\rangle $, $ o(a)=\frac{q+1}{4}$ and $o(t)=2.$
Let $x_1\in N_G(M)$ and let  $x_2$, $x_3$ be such that $N_G(M)^{(x_2)}=V\times \langle t\rangle $ and $N_G(M)^{(x_3)}=V\times \langle ta \rangle $, which is  possible by Proposition \ref{NGMu1e}. Hence we get that\\
$N_G(M)^{(x_1,x_2,x_3)}=N_G(M)\cap (V\times \langle t \rangle)\cap (V\times \langle ta \rangle)=V.$\\
Let us choose elements $y_1,y_2$ such that $N_G(M)^{(y_1,y_2)}=V$.
 Since $x_1$ normalizes $V$, if $y_1$ acts in  the same way as $x_1$,  then
$y_1$  also
normalizes $V$. Therefore,
 $V=N_G(M)\cap N_G(M)^{y_1}\cap N_G(M)^{y_2}=N_G(M)^{(y_2)}$, which
 is a contradiction  by Propositon \ref{NGMu1e}. Thus $d_c(N_G(M),G)=6$.
\end{proof}
\begin{prop}\label{5.6}
There is an element $x\in G$ such that $N_G(M)^{(x)}=\{1\}$.
\end{prop}
\begin{proof}
We need  to estimate  the number of elements $x$, such that
 $N_G(M)^x$ contains special elements: elements of $M$,  $3$-elements of $N_G(M)$,  involutions of $N_G(M)$, respectively.
 First we make the following observations:

\begin{itemize}
\item{
We have seen in Lemma \ref{nincsq+1/4es} that if $N_G(M)^x$ contains some nontrivial elements of $M$ for some $x\in G$, then $x\in N_G(M)$. Thus we get that}
$A_{\frac{q+1}{4}}:=\{x\in G \mid  N_G(M)^{(x)}\cap M\neq1\}=N_G(M).$
\item{
Let $p$ be a  nontrivial $3$-element in $N_G(M)$. First we show that for an element $x\in G$ the subgroup $N_G(M)^x$ contains $\langle p\rangle$ if and only if $x\in N_G(M)N_G(\langle p\rangle)$.}\\ Let $\langle p \rangle\leq N_G(M)^x$.
Since  the subgroups of order $3$ are conjugate in $N_G(M)$, there is an element $n\in N_G(M)$ such that $\langle p\rangle^{x^{-1}}=\langle p\rangle^n$.
Hence $nx\in N_G(\langle p\rangle )$, and $x\in N_G(M)N_G(\langle p\rangle)$. The other direction is trivial.
Using that $N_G(\langle p \rangle)=C_G(\langle p\rangle)$ is of order $2q^2$ by Cor.\ref{centrp} (ii), we  have: 
    $|A_3|:=|\{x\in G \mid N_G(M)^{(x)} \mbox{ contains $3$-elements}\}|= |\cup_{\{\langle p\rangle \leqslant N_G(M) \mid o(p)=3 \}} |N_G(M) N_G(\langle p\rangle)|$ $\leqslant \sum_{\{\langle p\rangle \leqslant N_G(M) \mid o(p)=3 \}}\frac{6(q+1)2q^2}{6}=2q^2(q+1)^2.$
\item{There are $|C_G(i)|=q(q-1)(q+1)$ elements in $G$, which take a fixed involution to a fixed involution.
 Furthermore, the subgroup $N_G(M)=(V\times (M\rtimes C_2))\rtimes C_3$ contains $q+4$ involutions. Thus we have:}
    $|A_2|:=|\{x\in G\ |\ N_G(M)^{(x)} \mbox{ contains involutions}\}|\leqslant (q+4)^2q(q-1)(q+1).$
\end{itemize}
 Using all the above  results we have:
\begin{align*}
|\{x\in G \mid N_G(M)^{(x)}=1\}|&\geqslant |G| -|A_{\frac{q+1}{4}}|-|A_3|-|A_2|=\\
&=q^3(q^3+1)(q-1)-6(q+1)-2q^2(q+1)^2-(q+4)^2q(q-1)(q+1).
\end{align*}
Since this is greater than $1$, we get that there is an element $x\in G$ such that $N_G(M)^{(x)}=\{1\}$.
\end{proof}
Using Theorem \ref{inter} we have the following.
\begin{tetel} The ordinary depth of $N_G(M)$ is $3$.
\end{tetel}
\section{The depth of $C_G(i)$}

Since $C_G(i)=\langle i\rangle \times L$, where $L\simeq PSL(2,q)$, the following theorem will be useful.
\begin{tetel}(Dickson)\cite[8.27 Hauptsatz, Ch II, p. 213]{Hup1}\label{reszcsoplista}
 The complete list of all subgroups of $L=PSL(2,q)$ for $q=3^{2n+1}$ is the following:
\begin{enumerate}[a)]
\item{alternating groups $A_4$;}
\item{elementary-abelian $3$-groups of order at most $q$;}
\item{semidirect products $C_3^m\rtimes C_t$ of elementary-abelian    groups of order at most $q$ with cyclic groups of order $t$, where $t$ divides $3^m-1$ as  well as $\frac{q-1}{2}$;}
\item{groups $PSL(2, 3^m)$, where $m\mid 2n+1$;}
\item{cyclic groups of order $z$, where $z$ divides $\frac{q-1}{2}$;}
\item{dihedral subgroups of order $2z$ with $z$ as above in e);}
\item{cyclic groups of order $z$, where $z$ divides $\frac{q+1}{2}$;}
\item{dihedral subgroups of order $2z$ with $z$ as above in g) .}
\end{enumerate}
\end{tetel}
\begin{prop}\label{U1}$\mathcal{U}_{\;1}(C_G(i))=\{C_G(i), C_P(i), C_G(V), K, C, 1 \mid i\in N_G(P), P\in Syl_3(G), i\in V\simeq C_2^2, K\simeq C_2^2, [i,K]=1,i\notin K,C\simeq C_2, [i,C]=1, i\notin C\}$
\end{prop}
\begin{proof}
We will examine  what kind of  subgroups  $H\in \mathcal{U}_{\;1}(C_G(i))$ occur. If we
find a subgroup type which could occur, we will also construct it.
%$C_G(i)\cap C_G(i)^z=C_G(i)\cap C_G(i^z)$ for some $z\in G$.
%Since $C_G(i)$ is isomorphic to $\langle i \rangle \times L$, where $L$ is isomorphic to $PSL_2(q)$, we can assume, that $H\leqslant L$, if $i\notin H$.
 For this, we choose an involution $j$ such that $C_G(i)\cap C_G(j)=H$.
 Since the involutions  are conjugate in $G$, we will be done.\\
  
If $[i,i^{x_1}]=1$, then
$C_G(i)^{(x_1)}=C_G(i)\cap C_G(i^{x_1})=C_G(V)$,  where
 $V=\langle i,i^{x_1}\rangle $ is a Klein $4$-subgroup of $G$.
By Theorem \ref{ree} (e), we have that
$C_G(i)\cap C_G(i^{x_1})\simeq(C_2^2\times D_{\frac{q+1}{2}}).$
We remark that $i\in V= Z(C_G(i)^{(x_1)})$.
\begin{itemize}
\item[ $C_G(V):$] Now we show that every subgroup $H\simeq C_2^2\times D_{\frac{q+1}{2}}$, where $i\in Z(H)$, occurs in $\mathcal{U}_1{(C_G(i))}$.
 Let $j$ be another involution in $Z(H)$. Thus $\langle i,j=\rangle Z(H)$ and
$C_G(Z(H))=C_G(i)\cap C_G(j)=H$ by Theorem \ref{ree} (e).
\end{itemize}


If $[i,i^{x_1}]\neq1$, then  let $D:=\langle i, i^{x_1}\rangle$.
Since  $C_G(i)= \langle i \rangle\times L$, where  $L\simeq PSL(2,q)$,
therefore the projection to the second component $\pi_2(C_G(i)^{(x_1)})=\pi_2(C_G(D))$ is a subgroup of $L\simeq PSL_2(q)$. Now we examine,
which subgroups of $PSL_2(q)$ can occur.
 We study the  subgroups of the list in Theorem \ref{reszcsoplista}
whether they could be isomorphic to $\pi_2(C_G(D))$.\\

\begin{itemize}
\item[a)-b)] \textbf{
Let us suppose that
 $\pi_2(C_G(D))$ contains an
element $p$ of order $3$. Then $C_G(D)$ also contains an element of order $3$.
Moreover,
  $C_G(i)^{(x_1)}=C_P(i)\simeq C_3^{2n+1}$ for
 some $P\in Syl_3(G)$  with $i\in N_G(P)$. On the other hand,
 every subgroup of the form $C_P(i)$
occurs as $C_G(i)^{(x_1)}$ if $i\in N_G(P)$ for some $P\in Syl_3(G)$}.\\
 By raising to the second power if necessary, we may assume that
$p\in C_G(D)$.
 Let  $p\in P\in Syl_3(G)$. We remark that, by Cor. \ref{centrp},
$p\in P'\backslash Z(P)$. 
 Since the Sylow $3$-subgroups are TI,
    $i,i^{x_1}\in N_G(P).$
     In particular, $i,i^{x_1}\in P\rtimes \langle i\rangle$.
 Thus $i^{x_1}=p'i$ for a suitable $p'\in P$.
 Since $[p,i]=1$ and $[p,p'i]=1$,  also $[p,p']=1$.
 Hence $p'\in P'$ by Lemma \ref{centralizerof3orderelement}.
If $p'\in Z(P)$, then by  Cor. \ref{centrp} we have  that $C_G(p')=P$.
 Thus $C_G(D)=C_G(i)\cap C_G(p')=C_P(i)$.
\\If $p'\notin Z(P)$, then 
 by Cor. \ref{centrp},  $C_G(p')=P'\rtimes \langle j \rangle$, where
$j$ is an involution in $N_G(P)$. Since $[p',i]\ne 1$,  we have that $j\ne i$.
All the involutions of $C_{N_G(P)}(i)$ are contained in $P'\langle i \rangle$.
So they are conjugate by elements in $P'$. Thus none of them can be in
$P'\rtimes \langle j \rangle$, otherwise $i$ would be in $C_G(p')$, which is not the case.
Hence $C_G(D)=C_G(i)\cap C_G(p')=C_{P'}(i)=C_P(i)$, by Theorem \ref{ree} (b).  


\item[$C_P(i):$] Now we will show, that every  subgroup $C_P(i)$ occurs as $C_G(D)$, if
 $P\in Syl_3(G)$  and $i\in N_G(P)$.
 Let $P\in Syl_3(G)$ such that $i\in N_G(P)$.
 The involution $i$ acts on $P'\simeq C_3^{4n+2}$ nontrivially.
 Choose an element $p\in P'$, which is not centralized by $i$.
Let $p':=[p,i]$. Then $p'\in P'$ and it is inverted by $i$.
Moreover,  $j:=p'i=i^p$ is
 an involution
such that
    $C_G(\langle i,j\rangle )=C_G(\langle i,p'\rangle)=C_G(i)\cap C_G(p')=C_P(i)
\simeq C_3^{2n+1},$ as above.
\item[c)-f)] \textbf{  Let  $[i,i^{x_1}]\ne 1$  and let
 $D:=\langle i,i^{x_1}\rangle$. Then $\pi_2(C_G(i)^{(x_1)})=\pi_2(C_G(D))$
 cannot contain a nontrivial cyclic subgroup $C$ whose order
 divides $\frac{q-1}{2}$.}
Raising to the second power if necessary, we may suppose that $C_G(D)\geqslant C$.
 Then $D\leqslant C_G(C)\simeq C_{q-1}$ by Lemma \ref{centralOfqpm1},
 which is a contradiction.
\item[g)-h)] \textbf{Let
$[i,i^{x_1}]\ne 1$, and let $D=\langle i,i^{x_1}\rangle$.
 Then if $\pi_2(C_G(i)^{(x_1)})=\pi_2(C_G(D))$ is a dihedral subgroup of
order $2z$ or a cyclic subgroup of order $z$,
 where $z\mid \frac{q+1}{2}$, then $z\leqslant 2$.
 Only $1$,  $C_2$ and $C_2^2$ can occur as $\pi_2(C_G(D))$, then $C_G(D)\simeq 1$, $C_2$ 
or $C_2^2$, ($C_2^3$ cannot occur).
 Obviously  $i\notin C_G(D)$ and $[C_G(D),i]=1$.
    }\\
    If $z>2$, then $\pi_2(C_G(D))$ contains a cyclic subgroup $C\neq \{1\}$
 whose size divides $\frac{q+1}{4}$. Raising to the second power the elements of $C_G(D)$, if necessary, we may assume that $C\leqslant C_G(D)$.
 Thus,  by Lemma \ref{centralOfqpm1}, $D\leqslant C_G(C)\simeq C_2^2\times C_{\frac{q+1}{4}}$,
 which is a contradiction.
 Furthermore,
if $S\in Syl_2(G)$ and $C_G(D)\geqslant S$ then $i,i^{x_1}\in S$, which is a contradiction.\\

 We will show that $1$, $C_2$ and  $C_2^2$ occur as $C_G(D)$ if $i\notin C_2^2$ and   $[C_2^2,i]=1$, or $i\notin C_2$ and $[C_2,i]=1$.

\item[$C_2:$]\label{C2} Let $H=\langle k\rangle \simeq C_2$ such that $k\neq i$ and
  $[k,i]=1$. Then $i\in C_G(k)=\langle k\rangle \times U$, where $U\simeq PSL(2,q)$.
 Since  the involutions in $PSL(2,q)$ are conjugate, see 
\cite[8.5 Satz, Ch. II,  p. 193]{Hup1}, if $i\in U$, then  there exists a subgroup
 $T(\simeq D_{q-1})\leqslant U$ such that $i\in T$. Let $C\leq T$ be the unique
cyclic subgroup of order $\frac{q-1}{2}$ and
    let $j$ be  another involution in $T$.
 Then, by Lemma \ref{centralOfqpm1}, $C_G(ij)\simeq C_{q-1}$, thus 
$C_G(ij)=\langle k \rangle\times C$. Thus,
   $C_G(i)\cap C_G(j)= C_G(i)\cap C_G(ij)=\langle k\rangle $.
If $i\notin U$, then $i=ki_2$, and $i_2$ is in a dihedral subgroup $T\leq U$ of order
$q-1$.
 Let $C\leq T$ be the unique cyclic subgroup of order $\frac{q-1}{2}$  and
let $T_1:=\langle C,i\rangle$. Then  $T_1$ is also dihedral of order $q-1$. Let
$j$ be another involution in $T_1$. Then $C_G(ij)=\langle k\rangle\times C$.
As above, we have that $C_G(i)\cap C_G(j)=\langle k\rangle$.

\item[$C_2^2:$]Let $H\simeq C_2^2$ such that $i\notin H$,  and $[i,H]=1$.
 Then, by Theorem
\ref{ree} (e), $i\in C_G(H)= H\times T\simeq C_2^2 \times D_{\frac{q+1}{2}}$.
Let $C\leq T$ be the unique cyclic subgroup of order $\frac{q+1}{4}$. 
 If  $i\in T$, then  let $j$ be another   involution
in $T$. By Lemma \ref{centralOfqpm1},  $C_G(ij)\simeq C_2^2\times C_{\frac{q+1}{4}}$, thus  $C_G(ij)= H\times C $. 
   Hence, 
     $C_G(i)\cap C_G(j)= C_G(i)\cap C_G(ij)=H.$
If $i\notin T$, then $i=i_1i_2$, where $i_1\in H$ and $i_2\in T$.
Then let $C$ be as above and
let $T_1:=\langle i,C\rangle$. Then $T_1$ is also dihedral of order
 $\frac{q+1}{2}$. Let $j$ be another involution in $T_1$.
Then $C_G(ij)=H\times C$.
And hence $ C_G(i)\cap C_G(j)= C_G(i)\cap C_G(ij)=H.$
\end{itemize}
%Finally we show that the trivial subgroup also occurs as $C_G(i)^{(x_1)}$,
 % where $[i,i^{x_1}]\ne 1$.
\begin{itemize}
\item[$1:$]There is an involution $j$ such that $C_G(i)\cap C_G(j)=\{1\}$\\
   Let $M^1$  be as in Theorem \ref{ree} (d).
 Let $M^2$  be a conjugate of $M^1$,  satisfying that
 $i\in N_G(M^2)$ and let  $j$  be another involution in $N_G(M^2)$.
We have that
  $\langle i,j\rangle$ is a dihedral subgroup
 of $N_G(M^2)$ and $ij$ is an element of $M^2$.
 Using 
%Lemma \ref{centraliserOfqpm3m+1}
Theorem \ref{ree} (d) and the fact
that  $2\cdot |PSL(2,q)|$ and $q+3m+1$ are relatively prime, we get that\\
    $C_G(i)\cap C_G(j)=C_G(i)\cap C_G(ij)\simeq(\langle i\rangle \times
PSL(2,q))\cap C_{q+3m+1}=\{1\}$.
\end{itemize}
Hence $C_G(i)^{(x_1)}$ can have  exactly the values  mentioned in the Proposition.
\end{proof}
\begin{prop}\label{u2cgi}$
\mathcal{U}_{\;2}(C_G(i))=\mathcal{U}_{\;1}(C_G(i))\cupdot \{\langle i\rangle\}\cupdot \{V\ \mid \ i\in V\simeq C_2^2\}\cupdot \{S\ \mid \ i\in S\simeq C_2^3\}$
\end{prop}
\begin{proof} We have seen in  Proposition \ref{U1} that $\mathcal{U}_{\;1}(C_G(i))=\{H_2 \mid H_2\simeq C_2, i\notin H_2,\ [i,H_2]=1 \}\cupdot
\{H_4 \mid H_4\simeq C_2^2, i\notin H_4,\ [i,H_4]=1 \}
\cupdot\{ H_{2q+2} \mid H_{2q+2}\simeq C_2^2\times D_{\frac{q+1}{2}}, i\in Z(H_{2q+2})\}\cupdot \{1, C_G(i)\}\cupdot\{ C_P(i) \mid  P\in Syl_3(G), i\in N_G(P)\}$.
We will examine $C_G(i)^{(x_1,x_2)}$ as the intersection of
 $C_G(i)^{(x_1)}$ and  $C_G(i)^{(x_2)}$ similarly to the previous proof.
 Let us see the  $\mathcal{U}_1-\mathcal{U}_1$ table for $C_G(i)$.
\begin{center}
\begin{tabular}{|c|cccccc|}
\hline
& $1$ &$ H_2$& $H_4$& $H_{2q+2}$&$ C_P(i)$&$ C_G(i)$\\\hline
$1$ & $1$ & $1$ & $1$ & $1$ & $1$ & $1$ \\
$ H_2$ & $1$ & $1, H_2$ & $1, H_2$ & $1,H_2$ & $1$ & $H_2$ \\
$H_4$ & $1$ &$1, H_2$  & $1, H_2, H_4$  &$1,H_2, H_4$& $1$ & $H_4$   \\
$H_{2q+2}$ & $1$ &$1, H_2$  &$1,H_2, H_4$
&{ $\langle i \rangle $},{ $V,S$},$H_{2q+2} $ & $1$ & $H_{2q+2}$ \\
$ C_P(i)$ & $1$ & $1$ & $1$ & $1$ & $1,C_P(i)$ & $C_P(i)$ \\
$ C_G(i) $ & $1$ & $H_2$ & $H_4$ & $H_{2q+2}$  &$C_P(i)$  &  $C_G(i)$\\
  \hline
\end{tabular}
\end{center}
The only relevant case is when both $C_G(i)^{(x_1)}$ and $C_G(i)^{(x_2)}$
are isomorphic to $C_2^2\times D_\frac{q+1}{2}$. Then $[i,i^{x_1}]=[i,i^{x_2}]=1$.
 Let   $V_1:=\langle i,i^{x_1}\rangle$  and
$V_2:=\langle i,i^{x_2}\rangle$.
Then
  $C_G(V_l)=C_G(i)^{(x_l)}$ for $l=1,2$. Furthermore, we introduce the following notation:  $C_G(i)\geqslant C_G(V_l)=V_l\times (M_l\rtimes \langle i_l\rangle)\simeq V_l\times  D_{\frac{q+1}{2}}$ for $l=1,2$, i.e. $M_l$ denotes the cyclic
subgroups of order $\frac{q+1}{4}$ in $C_G(V_l)$.
Since  $C_G(i)\simeq \langle i \rangle \times L$, where $L\simeq PSL(2,q)$, both
$M_1$ and $M_2$ are subgroups of the second component in this
direct product.  According to Theorem \ref{ree} (e),
 these two subgroups either coincide or intersect trivially.
\begin{itemize}
\item[$H_{2q+2}:$]
If $M_1=M_2$ then since $C_G(M_l)=V_l\times M_l$, we have that
$V_1=V_2$ and then $C_G(i)^{(x_1,x_2)} =C_G(V_1)\simeq  H_{2q+2}$.
\end{itemize}
If they intersect trivially, then $V_1\ne V_2$ and the intersection
 $C_G(V_1)\cap C_G(V_2)=C_G(i)^{(x_1,x_2)}$ is a
$2$-subgroup. We will show that for a suitable choice of $x_1$, $x_2$, the intersection can be $S$, $V$ or $\langle i\rangle$.
\begin{itemize}
\item[$S:$] Let us choose the elements $x_1,x_2$ in such a way that
 $S:=\langle i,i^{x_1},i^{x_2} \rangle \in Syl_2(G)$.
This  is possible since all involutions of $G$ are conjugate.  Let $V_1, V_2$ be as above. Then $V_1\ne V_2$, thus
$C_G(V_1)\cap C_G(V_2)$ is a $2$-group. Since it contains $S$, it must be $S$.\\
\item[$V:$] Let $V=\langle i, j \rangle $, where $i$ and $j$ are involutions in $G$ with
 $[i,j]=1$. We want to show that there are two involutions $i^{x_1}, i^{x_2}$
such that $V=C_G(i)^{(x_1,x_2)}$.
 We know that $C_G(i)\cap C_G(j)\simeq\langle i, j\rangle\times D_{\frac{q+1}{2}}$.
Let us choose two different involutions $k_1,k_2$ from $D_{\frac{q+1}{2}}$.
Let $V_1:=\langle i, k_1\rangle, V_2:=\langle i,k_2\rangle$.
Then $V_1\ne V_2$ thus $C_G(V_1)\cap C_G(V_2)$ is a $2$-group.
However, each Sylow $2$-subgroup of $C_G(V_1)$ contains $V_1$ and
each Sylow $2$-subgroup of $C_G(V_2)$ contains $V_2$, thus the intersection cannot be a Sylow $2$-subgroup since $[k_1,k_2]\ne 1$. So the intersection is $V$. Since   the involutions are all conjugate in $G$, thus $k_1=i^{x_1},k_2=i^{x_2}$ for suitable elements $x_1,x_2$. Hence, $V=C_G(i)^{(x_1,x_2)}$.
\item[$\langle i\rangle:$] Now we construct $V_1,V_2$ in such a way that $C_G(V_1)\cap C_G(V_2)=\langle i \rangle$.
Let $[i,i^{x_1}]=1$. Then $C_G(i)\cap C_G(i^{x_1})=\langle i \rangle\times (L\cap C_G(i^{x_1}))=\langle i\rangle \times D$, where $D\leqslant L\simeq PSL(2,q)$ is a dihedral subgroup of order $q+1$. However,
by Lemma 2.2 in \cite{F}, for each dihedral subgroup $D$
 of $L\simeq PSL(2,q)$ there is an element
 $g\in L$ such that $D^g\cap D=\{1\}$. Thus
if we choose $g$ in this way then $g\in C_G(i)$ and
$C_G(i)\cap C_G(i^{x_1g})=\langle i \rangle  \times {D}^g$ and hence
$C_G(i)^{(x_1,x_1g)}=\langle i \rangle $.
\end{itemize}
Since $H_{2q+2}$ always contains $i$, no other intersection is possible.
\end{proof}


%:Thus new contents of $\mathcal{U}_2(C_G(i))$ are the Sylow subgroups, which contain $i$, and those Klein groups, which contains $i$ and maybe $\langle i \rangle$a Now we will show, that there is $x_1,x_2$ such that $C_G(i)^{(x_1,x_2)}$ does not contain any involution unless $i$, therefore  $C_G(i)^{(x_1,x_2)}=\langle i \rangle $.\\
%The question is, whether exist involutions $i^{x_1},i^{x_2}$ such that there is no involution, which commutes with, $i,i^{x_1},i^{x_2}$. If there is an involution, then it is in $C_{C_G(i)}(i^{x_1})\cap C_{C_G(i)}(i^{x_2})$. Since this involution can not be $i$, we have to examine $C_{Psl_2(q)}(i^{x_1})\cap C_{Psl_2(q)}(i^{x_2})$. We know from Theorem \ref{reszcsoplista}, that $C_{Psl_2(q)}(i^{x_j})\simeq D_{q+1}$. Furthermore we know from \cite{} Fritzsche, that there is $x_1$, $x_2$ such that $C_{Psl_2(q)}(i^{x_1})\cap C_{Psl_2(q)}(i^{x_2})=1$.
\begin{tetel}We have that $d_c(C_G(i),G)=6$ and $d(C_G(i),G)=3$.
\end{tetel}
\begin{proof}
Now we  set up a $(\mathcal{U}_2\setminus \mathcal{U}_1)-\mathcal{U}_1$ table to study
the types of subgroups $Z\in \mathcal{U}_{\;3}(C_G(i))$.
 We know that  $Z$ is the intersection of some $X$ and $Y$,
 where $X\in \mathcal{U}_{\;1}(C_G(i))$
 and $Y\in \mathcal{U}_{\;2}(C_G(i))$. Obviously the intersection of
 two elements of $\mathcal{U}_{\;1}(C_G(i))$ is in $\mathcal{U}_{\;2}(C_G(i))$,
 therefore we  only need to examine   the case when Y is a new
 content of $\mathcal{U}_{\;2}(C_G(i))$. We will use the same notation
 for the subgroup types of $\mathcal{U}_{\;1}(C_G(i))$ as in the proof of
 Proposition \ref{u2cgi}. Denote the  subgroup types in 
  $\mathcal{U}_{\;2}(C_G(i))\setminus \mathcal{U}_{\;1}(C_G(i))$ by $\langle i\rangle$, $V$ and $S$, respectively,
 where $V$ is a Klein $4$-subgroup containing $i$ and $S$ is a Sylow $2$-subgroup containing $i$.
\begin{center}
\begin{tabular}{|c|cccccc|}
\hline
& $1$ &$ H_2$& $H_4$& $H_{2q+2}$&$ C_P(i)$&$ C_G(i)$\\\hline
$\langle i\rangle$ & $1$ & $1$ & $1$ & $\langle i\rangle$ & $1$ & $\langle i\rangle$ \\
$ V$ & $1$ & $1, H_2$& $1,H_2$ & $\langle i\rangle, V$  & $1$ & $V$ \\
$S$ & $1$ &$1, H_2$  & $1, H_2, H_4$  &$\langle i\rangle, V,S$& $1$ & $S$   \\
  \hline
\end{tabular}
\end{center}
Now we have that $\mathcal{U}_{\;2}(C_G(i))=\mathcal{U}_{\;3}(C_G(i))$ and thus, $d_c(C_G(i), G)$ is $5$ or $6$ by condition (1) in Definition \ref{jellemzes}.
Using this,  we will show that the combinatorial depth is $6$.\\
Let $x_1\in C_G(i)$ and let $x_2, x_3$  be elements in $G$
such that
 $C_G(i)^{(x_2)}=C_G(\langle i,k_1\rangle )$, $C_G(i)^{(x_3)}=C_G(\langle i,k_2\rangle )$ where $i,k_1,k_2$ are involutions  as in Proposition \ref{u2cgi},
at the construction of $V$. Then $[i,k_1]=[i,k_2]=1\neq [k_1,k_2]$. 
We also have that
$i\in C_G(i)^{(x_1,x_2,x_3)}=V.$
Let us suppose that we have elements $y_1,y_2\in G$ such that
$C_G(i)^{(y_1,y_2)}=C_G(i)^{(x_1,x_2,x_3)}$ and  $x_1\in C_G(i)$ acts in the same way as $y_1$. Then $y_1\in C_G(i)$. Therefore $V=C_G(i)^{(y_1,y_2)}=C_G(i)^{(y_2)}\in \mathcal{U}_1(C_G(i))$, which is a contradiction.
Thus, $d_c(C_G(i),G)=6$. 
Finally, since $1$ is the intersection $2$ conjugates of $C_G(i)$, by Theorem \ref{inter} we get that $d(C_G(i),G)=3$.
\end{proof}

\section{The depth of $G_0$ if $G_0\not\simeq R(3)$}
In this section we consider  a maximal subgroup $G_0\simeq R(q_0)$ of $G$, where
 $q_0=3^{2n_0+1}=3m_0^2$ for $n_0\geqslant 1$. We will study  which subgroups
of $G$ can occur in the form $G_0^{(g)}\ne G_0$, by considering the maximal subgroups of $G_0$ that can contain them. 
Let $H$ be a maximal subgroup of $G_0$. Let us introduce the notation
  $U_{H}=\{G_0^{(g)} \mid g\in G\setminus G_0 \mbox{ and } G_0^{(g)}\leqslant H\}$. Then we have that
$\mathcal{U}_{\;1}(G_0)=\{G_0\} \cup  \bigcup_{H\in Max(G_0)}\ U_{H}.$


The following Lemma shows that  in several cases $G_0^{(g)}$ is already in $\mathcal{U}_{\;1}(H)$, for some $H<G_0$. 
 


\begin{lemma}{\label{abcd}}
\begin{enumerate}[a)]
\item Let $H_0,T_0\leqslant G$ and  $g\in G$. If $H_0\leqslant T_0^{(g)}$,
 and the subgroups of $T_0$ of order $|H_0|$ are conjugate in $T_0$, then $g\in T_0N_G(H_0)$.
\item  Let $T_0,H\leqslant G$ and  $g\in G$. If there is an element $t\in T_0$ such that $t^{-1}g\in N_G(H)$ and $T_0^{(g)}\leqslant H$, then $T_0^{(g)}=(T_0\cap H)^{(t^{-1}g)}$.
\item Let $H\leqslant G$ and $g\in G$.
 If there exists an element $r\in G_0$ such that  $r^{-1}g\in N_G(H)$, then $G_0^{(g)}\cap N_G(H)=(N_{G_0}(H))^{(r^{-1}g)}$.
\item Let $H_0\leqslant G_0$ such that $N_{G_0}(H_0)$ is the only maximal subgroup in $ G_0$ containing $H_0$. If there exists an element $r\in G_0$ such that for a fixed $g\in G\backslash G_0$, $r^{-1}g\in N_{G}(H_0)$, 
then $G_0^{(g)}=(N_{G_0}(H_0))^{(r^{-1}g)}$.
\end{enumerate}
\end{lemma}
\begin{proof}
\begin{enumerate}[a)]
\item Since $H_0^{g^{-1}}\leqslant T_0$, there is an element $t\in T_0$ such that $H_0^{g^{-1}t}=H_0$ and  hence $g^{-1}t\in N_G(H_0)$, so we are done.
\item $T_0^{(g)}=(T_0\cap H)\cap (T_0^g\cap H)=(T_0\cap H)\cap (T_0^{t^{-1}g}\cap H^{t^{-1}g})=(T_0\cap H)^{(t^{-1}g)}$. 
\item $G_0^{(g)}\cap N_G(H)=G_0^{(r^{-1}g)}\cap N_G(H)=
N_{G_0}(H)\cap N_{G_0^{r^{-1}g}}(H)=N_{G_0}(H)\cap (N_{G_0}(H))^{r^{-1}g}$
\item By part c)  $G_0^{(g)}\geqslant (N_{G_0}(H_0))^{(r^{-1}g)}$,
thus  $H_0\leqslant G_0^{(g)}$. By assumption, $G_0^{(g)}\neq G_0$.
 Since both $G_0$ and $G_0^{r^{-1}g}$ have  only one maximal subgroup containing $H_0$,
we have that $G_0^{(g)}\leqslant N_{G_0}(H_0)\cap  N_{G_0^{r^{-1}g}}(H_0)=(N_{G_0}(H_0))^{(r^{-1}g)}$.
\qedhere
\end{enumerate}
\end{proof}
%Accordingly to determine $G_0^{(g)}$ we can consider this intersection as the intersection of smaller subgroups.
 The following Lemma shows  that  if we can find elements of certain type   in
 $G_0^{(g)}$, then   their centralizer in $G_0$ also lies in $G_0^{(g)}$.


 \begin{lemma}\label{6tolrelpr}
\begin{enumerate}[a)]
\item Let $x$ be a  non-trivial element  of $G$ whose order divides $\frac{q_0+1}{4}$. 
If     $x\in G_0^{(g)}$, then $C_{G_0}(x)(\simeq C_2^2\times C_{\frac{q_0+1}{4}})\leqslant G_0^{(g)}$.
\item  Let $x$ be  a non-trivial  element  of $G$ whose order divides $\frac{q_0-1}{2}$. If  $x\in G_0^{(g)}$,  then $C_{G_0}(x)(\simeq C_{q_0-1}) \leqslant G_0^{(g)}$.
\item Let $x$ be  a non-trivial element of $G$ whose order divides either $q_0+ 3m_0+1$ or $q_0- 3m_0+1$. If $x\in G_0^{(g)}$, then $C_{G_0}(x)\leqslant G_0^{(g)}$, and 
  is isomorphic to $ C_{q_0+ 3m_0+1}$ or $C_{q_0- 3m_0+1}$, respectively.
\item If  a Klein $4$-subgroup $V$  is contained in $G_0^{(g)}$,
then the unique subgroup  $M_0$ of order $\frac{q_0+1}{4}$ of
$C_{G}(V)= V\times (M\rtimes C_2)$  is a subgroup of $G_0^{(g)}$.
 Thus $C_{G_0}(M_0)=V\times M_0\leqslant G_0^{(g)}$.
\qedhere
\end{enumerate}
\end{lemma}
\begin{proof}
\begin{enumerate}[a)]
\item By Lemma \ref{centralOfqpm1}, $C_{G_0}(x)$ and $C_{G_0^g}(x)$
are isomorphic to $C_2^2\times C_{\frac{q_0+1}{4}}$ and\\
 $C_{G}(x)\simeq C_2^2\times C_{\frac{q+1}{4}}$. Since $C_G(x)$ contains  a unique subgroup of order $q_0+1$, we have that $C_{G_0}(x)= C_{G_0^g}(x)\leqslant G_0^{(g)}$. 
\item By Lemma \ref{centralOfqpm1},  $C_{G_0}(x)$ and
 $C_{G_0^g}(x)$ are isomorphic to $C_{q_0-1}$ and $C_{G}(x)\simeq C_{q-1}$.
 Since $C_G(x)$ contains a unique subgroup of order $q_0-1$, we have that
 $C_{G_0}(x)=C_{G_0^g}(x)\leqslant G_0^{(g)}$. 
\item We consider the case, when the order of $x$ divides $q_0+3m_0+1$, the other case is similar. By  Theorem \ref{ree} (d),
%Lemma \ref{centraliserOfqpm3m+1}
  $C_{G_0}(x)$ and $C_{G_0^g}(x)$ are isomorphic to $C_{q_0+3m_0+1}$ and $C_G(x)\simeq C_2^2\times C_{\frac{q+1}{4}}$ or $C_{q\pm 3m+1}$ depending on, whether $\frac{q+1}{4}$ or $q\pm3m+1$ is divisible by $q_0+3m_0+1$. In each  case there is  a unique subgroup of order $q_0+3m_0+1$ in $C_G(x)$ and  hence $C_{G_0}(x)=C_{G_0^g}(x)\leqslant G_0^{(g)}.$
\item Let us suppose that $G_0^{(g)}$ contains a Klein $4$-subgroup $V$.
  Observe that the subgroup $M_0$ of order $\frac{q_0+1}{4}$ of $C_G(V)=V\times (M\rtimes C_2)$ is unique. However,
$C_{G_0}(V)$ and $C_{G_0^g}(V)$ are both isomorphic to
 $V\times (C_{\frac{q_0+1}{4}}\rtimes C_2)$ and they are contained in $C_G(V)$.
Thus $M_0\leqslant G_0^{(g)}$, and hence $C_{G_0}(M_0)=V\times M_0\leqslant G_0^{(g)}$.
\qedhere
\end{enumerate}
\end{proof}

Now we determine $U_{N_{G_0}(P_0)}$ for a fixed Sylow
$3$-subgroup $P_0$ of $G_0$.

\begin{lemma}\label{hat}
 Let $P$ be a Sylow $3$-subgroup of the Ree group $G$ and let $W$ be an arbitrary complement of $P$ in $N_G(P)$. Let $W_{2'}$ be the $2'$ part of $W$. Then:
\begin{enumerate}[a)]
\item Every nontrivial element of $W$ acts regularly on $Z(P)\setminus \{1\}$ via the conjugation action.
\item The  conjugation action of every nontrivial element $w\in W_{2'}$ on $P'/ Z(P)$ has only one fixed point, namely $Z(P)$. Furthermore, the conjugation action of $W_{2'}$ has 3 orbits on $P'/Z(P)$, where $O_1=Z(P)$ and $O_2^{-1}=O_3$.
\item The  conjugation action of every nontrivial element $w\in W_{2'}$ on $P/ P'$ has only one fixed point, $P'$.
\end{enumerate}
\end{lemma}
\begin{proof} Obviously $Z(P)$ and $P'$ are $W$-invariant.
\begin{enumerate}[a)]
\item By Theorem \ref{ree} (b), for every nontrivial element
$w\in W_{2'}$, 
we have that $C_{Z(P)}(w)=\{1\}$. 
  The same holds for  elements of order $2$ in $W$.
 Hence, $C_{Z(P)}(w)=\{1\}$ for every  nontrivial element $w\in W$.
 Since $|W|=|Z(P)\setminus \{1\}|$, the action is regular.
\item By Theorem \ref{ree} (b), for every
 nontrivial element $w\in W_{2'}$, we have that  $C_{P'}(\langle w\rangle)=\{1\}$.
 
  Using  \cite[Theorem 3.15, Ch. 5, p. 187]{G},
 we have that  $C_{P'/Z(P)}(\langle w\rangle )=\{\overline 1\}$.  Obviously,
 an element and its inverse cannot belong to the same $W_{2'}$-orbit.
Since $|W_{2'}|=\frac{|P'/Z(P)|-1}{2}$, we are done.
\item We use again \cite[Theorem 3.15, Ch. 5, p.187]{G} to deduce that
$C_{P/P'}(\langle w \rangle)=\{\overline 1\}$.
\qedhere
\end{enumerate}
\end{proof}
\begin{prop}\label{3Sly}
Let $G$ be a Ree group, $G_0$ a Ree-subgroup of $G$ and $P$ a Sylow
 $3$-subgroup of $G$. Then $P\cap G_0$ is either trivial or a Sylow $3$-subgroup of $G_0$.
\end{prop}
\begin{proof}
Assume that the intersection of $G_0$ and $P$ is not trivial.
 In this case there is a Sylow $3$-subgroup $P_0$ of $G_0$ which  contains
 the intersection. Since the Sylow $3$-subgroups in $G_0$ are TI,
 $P_0$ is unique. If $P$ did not contain $P_0$, then there would be
 another Sylow $3$-subgroup $R$ of $G$, containing it. However,
 in  this  contradicts
 the fact, that the Sylow $3$-subgroups of $G$ are TI.
Therefore, we have that $G_0\cap P$ contains $P_0$ and hence they are equal.
\end{proof}
\begin{notation}\label{3jel} We fix the following notation.
 Let $G_0$ be a  fixed  maximal subgroup of $G$ isomorphic to $R(q_0)$. Let
$P_0$ be a Sylow $3$-subgroup of $G_0$. Let $P\in Syl_3(G)$  containing $P_0$.
Let us suppose that  in the representation of $P$ of Theorem \ref{ree} 
(h), the
 elements
of $P_0$ correspond to triples $\{(a,b,c) | a,b,c\in GF(q_0) \}$.
 Let $W_0$ be a complement of $P_0$ in $N_{G_0}(P_0)$  let $W$
be  the complement of $P$ in $N_G(P)$ containing $W_0$. Let $i$ be the unique
involution in $W_0$.
\end{notation}
\begin{lemma}\label{*} We use  Notation \ref{3jel}.
\begin{enumerate}[a)]
\item Let $x\in G$
 and  let $j$ be an involution in $P_0W_0$. Then  $j\in (P_0W_0)^{(x)}$ if and only if $x\in P_0 C_G(j)$.

\item  Let $W_1\leqslant P_0W_0$ of order $q_0-1$. Then $W_1\leqslant (P_0W_0)^{(x)}$
if and only if $x\in P_0N_G(W_1)$.
\end{enumerate}
\end{lemma}
\begin{proof} a) $\Rightarrow$ By assumption $j,j^{x^{-1}}\in P_0W_0$. Since the involutions are conjugate in $P_0W_0$,
 we have that
 there exists an element $p_0\in P_0$ such that $j^{p_0}=j^{x^{-1}}$, thus
 $p_0x\in C_G(j)$ and $x\in P_0C_G(j)$.\\
    $\Leftarrow$ By assumption there is an element $p_0\in P_0$ such that $p_0x\in C_G(j)$, thus $(P_0W_0)^{(x)}=(P_0W_0)^{(p_0x)}$, which  contains $j$.

b) The proof is similar to that of a).
\end{proof}
%{\bf We remark that this Lemma also true for $q_0=3$. }
\begin{prop}\label{ungp} Using Notation \ref{3jel} we have that
\[U_{N_{G_0}(P_0)}=\{ Z(P_0),P_ 0', P_0'\langle i^{p_0}\rangle \mid p_0\in P_0\}\cup U\cup V,\]
where
\begin{center}
$U\subseteq \{Z(P_0)\langle h\rangle, P_0,P_0\langle i\rangle, P_0W_0 \mid h\in P_0\setminus P_0'\}$
and
$V\subseteq \{ 1, \langle i^{p_0}\rangle , W_0^{p_0} \mid  p_0\in P_0\}$.
\end{center}
\end{prop}
\begin{proof}It directly follows from:
\end{proof}
\begin{lemma}\label{ungpcomp}Let $g\in G$ be an element
 such that $G_0^{(g)}\ne G_0$.  Then
\begin{enumerate}[(A)]
\item If $G_0^{(g)}$ contains a $3$-element $p\in P_0$,
then there exists an element $r\in G_0$ such that $r^{-1}g\in N_G(P)$.
 Moreover, this element $r$ can be chosen so that $r^{-1}g=p_1\in P\setminus P_0= (Z(P)\setminus Z(P_0))\cup (P'\setminus P_0'Z(P))\cup (P\setminus  P_0P')$ and $G_0^{(g)}=(P_0W_0)^{(r^{-1}g)}=(P_0W_0)^{(p_1)}$. In particular, $Z(P_0)\leqslant G_0^{(g)}$
and $G_0^{(g)}\leq N_{G_0}(P_0)$.
    \begin{enumerate}[I.]
    \item{ If $r^{-1}g(=p_1)\in Z(P)\setminus Z(P_0)$ then $G_0^{(g)}=G_0^{(p_1)}=P_0,P_0\langle i \rangle$ or $P_0W_0$.}
\item{ If $r^{-1}g(=p_1)\in P'\setminus P_0'Z(P)$ then $G_0^{(g)}=G_0^{(p_1)}$
 is isomorphic to $P_0'$ or $P_0'\langle i \rangle$.
 Moreover,  one can find an element $g_1  \in  P'\setminus P_0'Z(P)$  such that $G_0^{(g_1)}=P_0'$ and for every involution $j\in P_0W_0$ there exists
an element $g_2 \in  P'\setminus P_0'Z(P)$ such that
   $G_0^{(g_2)}=P_0'\langle j\rangle$. The case $P_0'W_0$ does not occur. Hence
 $U_{N_{G_0}(P_0)}$ contains $P_0'$ and $P_0'\langle j\rangle$, however  it does not contain $P_0'W_0$. }
\item{ If $r^{-1}g(=p_1)\in P\setminus P_0P'$ then $G_0^{(g)}=G_0^{(p_1)}$ is $Z(P_0)$ or $ Z(P_0)\langle h\rangle$, where $h$ is an element of order $9$ in $P_0$.
 Moreover, there exists an element $g_3\in P\setminus P_0P' $ such
 that $G_0^{(g_3)}=Z(P_0)$. Hence, $Z(P_0)$ belongs to $U_{N_{G_0}(P_0)}$.}


\end{enumerate}
\item If $G_0^{(g)}\leq N_{G_0}(P_0)$ and  $G_0^{(g)}$ does not contain any nontrivial $3$-elements from $P_0$, then $G_0^{(g)}$ is isomorphic  to $1,C_2$ or  $C_{q_0-1}$.
\end{enumerate}
\end{lemma}
\begin{proof}
{\it Case (A):} Let $p\in G_0^{(g)}$ be a nontrivial $3$-element belonging to $P_0$.
 Let $S_0\in Syl_3(G_0)$ containing $p^{g^{-1}}$.
Then there exists an element $r\in G_0$ such that ${S_0}^r=P_0$. Since $p,p^{g^{-1}r}\in P_0\leq P$, and $P$ is TI, we have that
$r^{-1}g\in N_G(P)$. Thus,  $g\in G_0N_G(P)\setminus G_0$.
 Then, since
 $G_0WP=G_0W_0P\cup G_0(W\setminus W_0)P= G_0(P\setminus P_0)\cup G_0\cup G_0(W\setminus W_0)P$,
 we have that
$G_0N_G(P)\setminus G_0= G_0(P\setminus P_0) \cup G_0(W\setminus W_0)P.$
\indent
We will show that if $g\in G_0(W\setminus W_0)P$ then $G_0^{(g)}$ does not contain
any non-trivial $3$-elements of $P_0$. Thus, in  the Case (A), $g\in G_0(P\setminus P_0)$
 holds.

Let $g=rw_1 p_1$, where $r\in G_0$, $w_1\in W\setminus W_0$ and $p_1\in P $.
We may suppose that $w_1\in W_{2'}$, otherwise, since $g=(ri)(iw_1)p_1$,
we substitute $r$  and $w_1$ by $ri$ and $iw_1$, respectively.
Using the fact that $N_{G_0}(P_0)\leq N_{G_0}(P)$ and
  Lemma \ref{abcd} c), with  $H:=P$ and $r:=r$,
 we obtain that $G_0^{(g)}\cap N_G(P)=
 N_{G_0}(P)^{(w_1p_1)}=N_{G_0}(P_0)^{(w_1p_1)}=(P_0W_0)^{(w_1p_1)}$.\\
Let $1\ne z\in Z(P_0)\cap (P_0W_0)^{w_1p_1}$. Then $z\in (P_0W_0)^{w_1}$
and $z^{w_1^{-1}}\in P_0\cap Z(P)=Z(P_0)$. Let $z_0=z^{w_1^{-1}}$. Then there exists an element $w_0\in W_0$ such that $z^{w_0}=z_0$. Hence $z^{w_0w_1}=z$.
Using Lemma \ref{hat} a) we have a contradiction, as $w_0\ne w_1^{-1}$.
 Hence $(P_0W_0)^{(w_1p_1)}$  does not
 contain  any nontrivial elements from $Z(P_0)$.\\
    Hence $(P_0W_0)^{(w_1p_1)}$   contains no elements of order $9$
 of $P_0$. We show  that $(P_0W_0)^{(w_1p_1)}$ does not contain any  nontrivial $3$-elements.  Let $h$ be an element of order $3$ in
 $(P_0W_0)^{(w_1p_1)}$. Then $h \in P_0'\backslash Z(P_0)$.
    Using the representation of the Sylow $3$-subgroups $P_0\in Syl_3(G_0)$ and $P\in Syl_3(G)$, we have that
  $h=(0,y_0,z_0)$ and $p_1^{-1}=(a,b,c)$, where
 $y_0\neq 0$, $y_0,z_0\in GF(q_0)$ and $a,b,c\in GF(q)$. Denote $w_1^{-1}$ by $w$. Then by Theorem  \ref{ree} (h) and (by Lemma \ref{conj}) we have that
    $h^{p_1^{-1}w}=(0,y_0,z_0)^{w}(0,0,2y_0a)^{w}\in P_0\cap P'=P_0'.$
 Denote the element  $(0,0,2y_0a)^w$ by $\zeta$. Since $(0,y_0,z_0)^w\zeta\in P_0'\setminus Z(P_0)$, applying Lemma \ref{hat} b) for $P_0'$ we have that there are elements $w_0\in (W_0)_{2'}$ and $\zeta_0\in Z(P_0)$ such that $(0,y_0,z_0)^w\zeta=(0,y_0,z_0)^{w_0}\zeta_0$ or $(0,y_0,z_0)^w\zeta=((0,y_0,z_0)^{-1})^{w_0}\zeta_0$.
    Using Lemma \ref{hat} b) again for $P^{'}$, we have in both cases a contradiction. In the first case $ww_0^{-1}\in W_{2'}$ fixes $(0,y_0,z_0)Z(P)$, and in the second case $ww_0^{-1}$ takes $(0,y_0,z_0)Z(P)$ to its inverse.
    Thus $G_0^{(g)}\cap P_0=\{1\}$,  which is a contradiction. Therefore 
$g\not\in G_0(W\backslash W_0)P$.
\\

 Hence if $G_0^{(g)}$ contains  a nontrivial
 $3$-element $p\in P_0$, then $g\in G_0(P\setminus P_0)$, i. e. $g=rp_1$ where $r\in G_0$ and $p_1\in P\setminus P_0$. Thus, by Lemma \ref{abcd} d) with $Z(P_0)$ as $H_0$, we have
 that $G_0^{(g)}=(N_{G_0}(Z(P_0)))^{(p_1)}=(P_0W_0)^{(p_1)}$. In particular,
$Z(P_0)\leqslant {G_0}^{(g)}\leqslant N_{G_0}(P_0)$.\\
Obviously, $P\setminus P_0$ is a disjoint union of $Z(P)\setminus Z(P_0)$, $P'\setminus P_0'Z(P)$ and $P\setminus P_0 P'$. Hence $p_1=r^{-1}g$ belongs to one of them.
\begin{enumerate}[I.]
\item{If $r^{-1}g=p_1\in Z(P)\setminus Z(P_0)$ then as $[P_0,p_1]=1$,
$G_0^{(g)}=(P_0W_0)^{(p_1)}\geqslant P_0$. Using  Lemma \ref{6tolrelpr} b) we have that $G_0^{(g)}$ is  $P_0,P_0\langle i \rangle $ or $P_0W_0$.
}


\item{If $r^{-1}g=p_1\in P'\setminus P_0' Z(P)$, then $G_0^{(g)}=(P_0W_0)^{(p_1)}\geqslant P_0'$, since $P'$ is elementary abelian.
If $G_0^{(g)}=(P_0W_0)^{(p_1)}$  contained an element $h$
of order $9$, then $h^{{p_1}^{-1}}\in P_0$ would hold. we will use
 the representation
of the elements of a Sylow $3$-subgroup $P$ of $G$ in Theorem \ref{ree} (h) and
Lemma \ref{conj}. Since $p_1\in P'\setminus P_0'Z(P)$, we have that
 ${p_1}^{-1}=(0,b,c)$ and $h=(x_0,y_0,z_0)$ with $b\not\in GF(q_0)$, 
 $x_0\ne 0$ and $x_0,y_0,z_0 \in GF(q_0)$. Hence,
 $h^{{{p_1}}^{-1}}=(x_0,y_0,z_0-2x_0b)\in P_0$, which is a contradiction.
Hence, again by Lemma \ref{6tolrelpr} b), $G_0^{(g)}$ can only be isomorphic
 to  $P_0',P_0'\rtimes C_2 $ or  $P_0'\rtimes C_{q_0-1}$.}

We prove that $G_0^{(g)}=(P_0W_0)^{(p_1)}$ is not isomorphic to $P_0'\rtimes C_{q_0-1}$. Assume that $W_1=W_0^{p_0}\leqslant (P_0W_0)^{(p_1)}$ for some $p_0\in P_0$. By Lemma \ref{*} b),
% $W_1\leqslant (P_0W_0)^{(p_1)}$ if and only if $p_1\in P_0N_G(W_1)$. Thus  we have that
 $p_1\in P_0N_G(W_1)$, so $p_1\in P_0N_P(W_1)$. However, by Theorem \ref{ree} (b)  we have that
 $N_G(W_1)\simeq W_1\rtimes C_2$,  hence $N_P(W_1)=1$. Thus $p_1\in P_0$,  which contradicts our assumption that $p_1\in P'\setminus P_0'Z(P)$.\\

We show, that for every involution $j\in P_0W_0$ and $p_1\in C_{P\setminus P_0}(j)$,    $G_0^{(p_1)}=P_0'\langle j\rangle$. By Theorem \ref{ree} (b), we have that
 $p_1\in P'\setminus Z(P)$ and $p_1\in P'\setminus P_0'$. We show that $p_1\in P'\setminus P_0'Z(P)$. For this, consider the representation of the Sylow $3$-subgroup of $G$ as in Theorem \ref{ree} (h).
By Theorem \ref{ree} (b) we have that $C_P(j)\cap Z(P)=\{1\}$. We prove that
 for every $b\in GF(q)$ there is at most one $c\in GF(q)$ such that $(0,b,c)\in C_P(j)$.
Let us suppose that  $(0,b,c_1),(0,b,c_2)\in C_P(j)$.  Then their quotient, $(0,0,c_1-c_2)\in C_P(j)\cap Z(P)=\{1\}$. Hence  $c_1=c_2$.
 Since $|C_P(j)|=q$,   for every $b\in GF(q)$ there is exactly
 one $c\in GF(q)$  with $(0,b,c)\in C_P(j)$. A  similar statement holds for
$C_{P_0}(j)$.  Hence, if $b\in GF(q_0)$, then the
unique $c$ must be  contained in $GF(q_0)$ and so $C_{P_0'Z(P)}(j)=C_{P_0'}(j)$.
  Thus, $p_1\in (P'\setminus P_0')\cap C_{P}(j)$ implies that
 $p_1\in P'\setminus P_0'Z(P)$. Since $p_1\in N_G(P)$,  we have seen at the beginning of the proof
that
 $G_0^{(p_1)}=(P_0W_0)^{(p_1)}$. Using that $p_1\in P'\setminus P_0'Z(P)$, we
have  already proved  that $(P_0W_0)^{(p_1)}$ is isomorphic to $P_0'$ or
$P_0'\langle i \rangle$. By Lemma \ref{*}, since $p_1\in C_G(j)$, we have that $j\in (P_0W_0)^{(p_1)}$.
Hence
$G_0^{(p_1)}=P_0'\langle j\rangle$.\\


Finally we prove, that there is an element $p_1\in P'\setminus P_0' Z(P)$ such that $(P_0W_0)^{(p_1)}$ does not contain any involutions.
 Let $j\in (P_0W_0)^{(p_1)}$ be a fixed involution.
 Using Lemma \ref{*}  for $x=p_1\in P'\setminus P_0'Z(P)$,
 by Theorem \ref{ree} (b) and (h) we have that\\
    $|\{p_1\in P'\setminus P_0' Z(P)  \mid  j\in(P_0W_0)^{(p_1)} \}|= |(P'\setminus P_0' Z(P))\cap (P_0C_P(j))|$ $\leq|P_0'C_P(j)\setminus P_0'|=(q-q_0)q_0$.

    If for some element $p_1\in P'\setminus P_0' Z(P)$, the subgroup
$(P_0W_0)^{(p_1)}$ contains an involution $j$, then
 $(P_0W_0)^{(p_1)}=P_0'\langle j \rangle $.
%, since we have proved that
%it cannot be $P_0'W_0$. %The subgroup $P_0'\langle j \rangle $
% contains exactly $q_0$ involutions. These are conjugate by elements of $P_0'$.
 Since the involutions are conjugate in $P_0W_0$ by elements of $P_0$,
 for every involution $j\in P_0W_0$ the same number of $p_1\in P'\setminus P_0'Z(P)$ occurs such that $(P_0W_0)^{(p_1)}$ contains $j$. Since by Lemma \ref{centralizerof3orderelement} we  have that $C_P(p_1)=P'$, the cosets of $P_0'$
in $P_0$ move $p_1$ to $q_0$ different places.
Thus, an upper bound for the number of elements $p_1\in P'\setminus P_0'Z(P)$  such that $(P_0W_0)^{(p1)}$ contains an involution can be obtained from the number of elements $p_1$, such that $(P_0W_0)^{(p1)}$ contains $j$, by multiplying it by $q_0$. 
%    Since $P_0W_0$ contains $q_0^2$ involutions, there are $q_0$ subgroups
%of $P_0W_0$ which are isomorphic to $P_0'\rtimes C_2$.
Therefore,
    $|\{p_1\in P'\setminus P_0' Z(P) \mid (P_0W_0)^{(p_1)}\mbox{ contains involutions} \}|\leq (q-q_0)q_0^2.$
    Since $|P'\setminus P_0' Z(P) |=q(q-q_0)$ is bigger than
 $q_0^2 (q-q_0)$, there is an element $p_1\in P'\setminus P_0' Z(P)$
 such that $(P_0W_0)^{(p_1)}$ does not contain involutions.
Then, by the beginning of the proof of II.,  we have that $(P_0W_0)^{(p_1)} =P_0'$  and hence
$G_0^{(p_1)}=(P_0W_0)^{(p_1)}=P_0'$.\\



\item{If $r^{-1}g=p_1\in P\setminus {P_0P'}$ then $G_0^{(g)}=(P_0W_0)^{(p_1)}\geqslant Z(P_0)$. We show that $(P_0W_0)^{(p_1)}$ cannot contain noncentral elements of order $3$ in $P_0$.}
As before, we use the representation in  Theorem 1 h) of $P$ and $P_0$.
Let us suppose that $h\in (P_0W_0)^{(p_1)}$ is a noncentral element of order $3$
in $P_0$. Let $p_1^{-1}=(a,b,c)$ and $h=(0,y_0,z_0)\in P_0$. Then
$a\not\in GF(q_0)$, otherwise $(a,b,c)=(a,0,0)(0,b,c+ab)\in P_0P'$, which is not the case. On the other hand,   $y_0\ne 0$, since otherwise  $h\in Z(P_0)$.
Then by Lemma \ref{conj}, $h^{{p_1}^{-1}}=(0,y_0,z_0+2y_0a)\not\in P_0$.
This is a contradiction, since $h^{{p_1}^{-1}}\in P_0$.\\

Now we prove that $G_0^{(g)}=(P_0W_0)^{(p_1)}$ does not contain any elements outside $P_0$. As before, using Lemma \ref{6tolrelpr} b), it is enough to show that $(P_0W_0)^{(p_1)}$ does not contain involutions.\\
Suppose this is not true and let $j$ be an involution in $(P_0W_0)^{(p_1)}$.
By Lemma \ref{*} we have  that $p_1\in P_0C_P(j)$ and by Theorem \ref{ree} (b),
   $p_1\in P_0P'$, which  contradicts  our assumption.\\

Now we show that if  there is an element $h\in P_0$ of order $9$
in $(P_0W_0)^{(p_1)}$, then $(P_0W_0)^{(p_1)}= Z(P_0)\langle h\rangle$.
 We use the representation of the Sylow $3$-subgroups $P_0$ and $P$, as before.
Let ${p_1}^{-1}=(a,b,c)$ and $h=(x_0,y_0,z_0)$. Clearly, $a\not\in GF(q_0)$, $x_0,y_0,z_0\in GF(q_0)$ and $x_0\ne 0$. Then by Lemma \ref{conj},
 we have that\\
 $h^{{p_1}^{-1}}=(x_0,y_0 +x_0(a\sigma)-a(x_0\sigma),z_0-2x_0b+2y_0a-ax_0(x_0\sigma)+ax_0(a\sigma))\in P_0.$
Consider the solutions  $x\in GF(q_0)\setminus \{0\}$ of the relation
 $x(a\sigma)-a(x\sigma)\in GF(q_0)$. Applying $\sigma $ to this relation, as $\sigma $ is also an automorphism of $GF(q_0)$, we have that
$(x\sigma)a^3-(a\sigma)x^3\in GF(q_0)$. Multiplying by $x^2$  the
 left hand side of the original relation, adding the left hand side of the second to it and dividing by $(x\sigma)$, we have that
$a^3-ax^2\in GF(q_0)$. Recall that $x_0$ and $-x_0$ both satisfy this relation.
If some $x\in GF(q_0)\setminus \{0\}$ also satisfies it, then
 $a^3-a(x_0)^2-(a^3-ax^2)=a(x_0-x)(x_0+x)\in GF(q_0)$. Since $a\not\in GF(q_0)$,  either $x=x_0$ or $x=-x_0$.
Let $h'=(x',y',z')\in (P_0W_0)^{(p_1)}$ be another element. As we have seen,
 $h'\in P_0$.
 If $h'\notin Z(P_0)$, then it is of order $9$. 
Then
$h'^{{p_1}^{-1}}=(x',y'+x'(a\sigma)-a(x'\sigma),z'-2x'b+2y'a-ax'(x'\sigma)+ax'(a\sigma))\in P_0.$

Then $x'$ also satisfies $a^3-ax'\in GF(q_0)$, thus $x'=x_0$ or $x'=-x_0$ holds.

We may suppose that $x'=x_0$, otherwise we consider $h'^{-1}$, instead
of $h'$. 
By subtracting the $3$-rd
component of $h'^{{p_1}^{-1}}$ from that of $h^{{p_1}^{-1}}$ we obtain that
$2a(y_0-y')\in GF(q_0)$. Since $a\not\in GF(q_0)$, 
$y_0-y'=0$ follows. Hence
$h'\in hZ(P_0)$.\\
% If $x'=-x_0$ then by adding
 %the third component
% of $h'^{{p_1}^{-1}}$ to that of $h^{{p_1}^{-1}}$, we have that
%$2a(y_0+y'-x_0(x_0\sigma)\in GF(q_0)$.
%Since $a\not\in GF(q_0)$, we have that $y'=-y_0+x_0(x_0\sigma )$, thus by Lemma \ref{conj}, we have that $h'\in h^{-1}Z(P_0)$ and hence
 Thus, $(P_0W_0)^{(p_1)}$ is  equal to  $ Z(P_0)\langle h \rangle$.\\

We have seen that $p_1\in P\setminus P_0P'$ implies that
 $(P_0W_0)^{(p_1)}$ is $Z(P_0)\langle h\rangle$ for
 some element $h\in P_0$ of order $9$ or $(P_0W_0)^{(p_1)}=Z(P_0)$.


We show that there is an element $p_1\in P\setminus {P_0P'}$ such that $G_0^{(p_1)}=(P_0W_0)^{(p_1)}=Z(P_0)$.\\
Let $p_1^{-1}=(a,b,c)\in P\setminus P_0P'$.
Let $h=(x_0,y_0,z_0)\in P_0$ be an element such that $x_0\ne 0$, i.e. it is of order $9$. Assume that $h\in (P_0W_0)^{(p_1)}$. Then $h^{p_1^{-1}}\in P_0$ hence
$x_0,y_0,z_0$ satisfies

\begin{gather}x_0(a\sigma )-a(x_0\sigma )\in GF(q_0)\tag{1}
\end{gather}
and
\begin{gather}\tag{2}
-2x_0b+2y_0a-ax_0(x_0\sigma )+ax_0(a\sigma )\in GF(q_0).
\end{gather}

Moreover,
\begin{align*}
|\{p_1\in P\setminus P_0P' \mid (x_0,y_0,z_0)\in (P_0W_0)^{(p_1)}\}|&\leqslant
|\{(a,b,c)\in GF(q)^3 \mid  a\notin GF(q_0), \mbox{  $b$ satisfies (2)}\}|\\
&\leqslant (q-q_0)q_0q.
\end{align*}
If $p_1\in P\setminus P_0P'$ and $(P_0W_0)^{(p_1)}$ contains an element $h$ of order $9$, then $(P_0W_0)^{(p_1)}= Z(P_0)\langle h \rangle $. It
 contains exactly $2q_0$ elements of order $9$, since $h^3\in Z(P_0)$.

So there are $\frac{q_0^3-q_0^2}{2q_0}$  elements of order $9$ in $P_0$ giving
different subgroups in $P_0$  of the form $ Z(P_0)\langle h \rangle$. Thus,\\
$|\{p_1\in P\setminus P_0P' \mid (P_0W_0)^{(p_1)}\mbox{ contains elements of order $9$}\}| \leq \frac{q_0^3-q_0^2}{2q_0} q_0q(q-q_0).$\\
Since $2q>q_0^2(q_0-1)$,
 we obtain that $|P\setminus P_0P'|=q^2(q-q_0)$ is greater than $\frac{1}{2} q (q_0^3-q_0^2)(q-q_0)$.
Therefore,
  there is an element $p_1$ such that $G_0^{(p_1)}=(P_0W_0)^{(p_1)}=Z(P_0).$

\end{enumerate}
{\it Case (B):}
Since $G_0^{(g)}$ does not contain $3$-elements and $G_0^{(g)}\leqslant N_{G_0}(P_0)=P_0W_0$, by Lemma \ref{6tolrelpr} b) we have that $G_0^{(g)}$ is isomorphic  to $1,C_2$ or  $C_{q_0-1}$.
\end{proof}

Now we introduce the following notation to determine $U_{N_G(M^{+1})}$, $U_{N_G(M^{-1})}$ and $U_{N_G(M)}$.


\begin{notation}\label{Hjel}
Let $G_0$ be a maximal subgroup of $G$ isomorphic to $R(q_0)$. Let
$M_0\in Hall_{\frac{q_0+1}{4}}(G_0)$.
 Since $q=q_0^a$, where $a$ is an odd prime,   $q_0+1|q+1$ follows. Hence
 $M_0\leqslant M$ for some $M\in Hall_{\frac{q+1}{4}}(G)$.
By Theorem \ref{ree} (e)  we have that $C_{G_0}(M_0)=M_0\times V_0$
and $C_G(M)=M\times V$, for some Klein-subgroups $V_0$ and $V$.
Since $V\triangleleft N_G(M)=V\times (M\rtimes C_2)\rtimes C_3$, we have that $V$ is contained in every Sylow $2$-subgroup of $N_G(M)$. Since by Theorem \ref{ree} (f), $N_{G_0}(M_0)\leqslant N_G(M)$, at least one of these Sylow $2$-subgroups is inside $N_{G_0}(M_0)$. Hence $V\leqslant N_{G_0}(M_0)$.
Using    $[V,M_0]=1$,  we have that $V\leqslant C_{G_0}(M_0)$ and thus $V_0=V$.
 We have that $N_{G_0}(M_0)=(M_0\times V_0)\rtimes \langle t\rangle$, where
 $o(t)=6$. Since by Theorem \ref{ree} (f),
 $N_G(M_0)\leq N_G(M)$,  we have equality here.
Hence,   $N_{G_0}(M_0)\leqslant N_G(M_0)=N_G(M)=(M\times V_0)\rtimes \langle t\rangle$.\\
 Let  $M_0^{j}\leq G_0$ be a  Hall subgroup of order $q_0+1+3jm_0$, where
$j=\pm 1$. It can be embedded into a Hall subgroup $\tilde M^{j'}$ of order $q+1+j'3m$ in $G$ or to a Hall subgroup $\tilde M$ of order $\frac{q+1}{4}$
 depending on which factor of $\frac{q^3+1}{4}=\frac{q+1}{4}(q+1+3m)(q+1-3m)$ is divisible by $q_0+1+j3m_0$. We note that  each case is possible, however $\frac{q+1}{4}$ divides only one of the factors since the nontrivial elements of
$\tilde M$, $\tilde M^{1}$ and $\tilde M^{-1}$ do  not centralize each other. 
 \\
Furthermore, similarly to the case of  $M_0$, by
Theorem \ref{ree} (d) (e) and (f), $N_{G_0}(M_0^j)=M_0^j\rtimes \langle t \rangle$  for
 $j=\pm 1$ and for an element $t$ of order $6$, and $N_G(M_0^{j})=N_G(\tilde M^{j'})$, where $j'=\pm j$
 or $N_G(M_0^{j})=N_G(\tilde M)$, depending on if $M_0^j\leqslant \tilde M^{\pm j}$ or
$M_0^j\leqslant \tilde M$. Thus\\
$N_G(M_0^j)\in \{\tilde M^{+ 1}\rtimes \langle t \rangle \mbox{ , } \tilde M^{-1}\rtimes \langle t \rangle\mbox{ , }(\tilde M\times V)\rtimes \langle t\rangle\}.$ 
%Further $N_{G_0}(M_0)=(M_0\times V)\rtimes \langle t\rangle$ and $N_G(M_0)=(M\times V)\rtimes \langle t\rangle$, where $V$ is the Klein group from $C_{G_0}(V)$ and $t$ is an element of order $6$.
%Recall that $N_G(M_0^0)=N_G(M_0)$ and $N_G(M^0)=N_G(M)$, we can use Theorem \ref{ree} (f) and  we get that $N_G(M_0^{j})\leqslant N_G(M^{j'})\simeq M^{j'}\rtimes C_6$, thus $N_G(M_0^j)=M^{j'}\rtimes \langle t \rangle$.
 \end{notation}

\begin{prop}\label{ungm+} We use Notation \ref{Hjel}.
If  $G_0^{(g)}$ contains a nontrivial element from $M_0^{+1}$ then $g\in G_0  N_G(M_0^{+1})$. Furthermore,
\[U_{N_{G_0}(M_0^{+ 1})}=\begin{cases}
\{ M_0^{+ 1}, M_0^{+ 1}\rtimes C_2\}\cup U  &\text{if } q_0+1+3m_0 | q+1 \\
\{ M_0^{+ 1}\}\cup U & \text{otherwise} 
\end{cases}\]
where $U$   may only contain  cyclic subgroups of order $2$ or $1$.
\end{prop}
\begin{proof}
Assume that $g\in G$ such that $G_0^{(g)}$ contains a nontrivial element
 $m\in M_0^{+1}$. Then by Lemma \ref{6tolrelpr} c), $M_0^{+1}\leq G_0^{(g)}$,
 thus $(M_0^{+1})^{g^{-1}}\leq G_0$.
 Using  the fact that the
Hall subgroups of order $q_0+1+3m_0$  are conjugate in $G_0$, we have that
there is an element $r\in  G_0$ such that
$ (M_0^{+ 1})^{g^{-1}r}=M_0^{+1}$ and thus $r^{-1}g\in N_G(M_0^{+ 1})$. Hence $g\in G_0N_G(M_0^{+1})$.\\

To prove  the second part, suppose that 
  $G_0^{(g)}\leqslant N_{G_0}(M_0^{+1})$ for some $g\in G$.  By Lemma \ref{ungpcomp}  if $G_0^{(g)}$ contains a nontrivial  $3$-element,   then it
 contains the center of a Sylow $3$-subgroup of $G_0$, thus it also
 contains a subgroup isomorphic to
  $C_3^2$. This  contradicts  our assumption $G_0^{(g)}\leqslant N_{G_0}(M_0^{+1})$.  Hence in this case
 $G_0^{(g)}$  does not contain nontrivial  $3$-elements.\\
 If $G_0^{(g)}$ does not contain nontrivial elements whose order divides $q_0+1+3m_0$, this subgroup is isomorphic  to either $C_2$ or  $\{1\}$. Assume that $G_0^{(g)}$ contains  nontrivial elements from $M_0^{+1}$. By the first part of the proof,
 we have that there is an element $r\in G_0$ such that $r^{-1}g\in N_G(M_0^{+1})$. Using Lemma \ref{abcd} b) with $H=N_G(M_0^{+1}),t=r$  and $T_0=G_0$ we obtain \\
$G_0^{(g)}=(G_0\cap N_G(M_0^{+1}))^{(r^{-1}g)}=(N_{G_0}(M_0^{+1}))^{(r^{-1}g)},$
where $r^{-1}g\in N_G(M_0^{+1})\setminus G_0$.
Then $M_0^{+1}\leqslant N_{G_0}(M_0^{+1})^{(r^{-1}g)}$ by the choice of $r^{-1}g$.
Recall that $N_{G_0}(M_0^{+1})=M_0^{+1}\rtimes \langle t_1 \rangle$, where $t_1$ is an element of order $6$. Observe that both $N_{G_0}(M_0^{+1})$ and $N_{G_0}(M_0^{+1})^{r^{-1}g}$ are Frobenius groups with the same Frobenius kernel. Thus $N_{G_0}(M_0^{+1})^{(r^{-1}g)}\neq M_0^{+1}$, if and only if there is an element $r_1\in M_0^{+1}$ such that
\begin{gather}\tag{*}\label{*2}
\langle t_1\rangle^{r^{-1}gr_1}\cap \langle t_1\rangle\neq 1.
\end{gather}
Here $r^{-1}gr_1\in N_{G}(M_0^{+1})\setminus G_0$. Depending on the relation of $q_0$ and $q$, the subgroup $M_0^{+1}$ is contained in one of
  $\tilde M^{+ 1}$,  $\tilde M^{-1}$ or $\tilde M$.\\\indent
First assume that $M_0^{+1}\leqslant \tilde M^{+1}$.
Then $N_G(M_0^{+1})=\tilde M^{+1}\rtimes \langle t_1\rangle$ is also a Frobenius group with Frobenius complement $\langle t_1\rangle$. Equation
 (\ref{*2}) implies that $r^{-1}gr_1\in\langle t_1\rangle$, which is a contradiction. Thus $G_0^{(g)}=M_0^{+1}$ in this case. This can really occur, e.g. if we choose $g\in \tilde M^{+1}\setminus G_0$.

%?Le kell �rni, hogy ha g olyan, hogy $G_0^{(g)}$ tartalmaz $M_0^{+1}$-beli elemet, de nem egyenl� $G_0$-lal, akkor $G_0^{(g)}$ a bizony�tottak alapj�n $N_{G_0}(M_0)$ �s term�szetesen $g\in N_G{M_0^{+1}}\setminus G_0$ ilyen elem.\\\indent

If $M_0^{+1}\leqslant \tilde M^{-1}$, then the proof is similar. \\\indent

Finally let us assume that $M_0^{+1}\leqslant \tilde M$.
 Thus $N_{G}(M_0^{+1})=(\tilde M\times V)\rtimes\langle t_1\rangle$.
 Let $m$ be a generator of $\tilde M$ and $r^{-1}gr_1=t_1^am^bv_1$, where
$a,b\in \mathbb{Z}$,  and $v_1\in V\setminus \{1\}$. Suppose that the $3$-element $t_1^2$ acts on $V=\{1,v_1,v_2,v_3\}$ as $v_1^{t_1^2}=v_2$, $v_2^{t_1^2}=v_3$ and $v_3^{t_1^2}=v_1$. The involution $t_1^3$ is centralizes $V$, thus $v_1^{t_1}=v_3$, $v_3^{t_1}=v_2$, $v_2^{t_1}=v_1$. 
Then $t_1^{t_1^am^bv_1}=t_1^{m^bv_1}=(t_1[t_1,m^b])^{v_1}=t_1^{v_1}[t_1,m^b]=t_1[t_1,v_1][t_1,m^b]$.
By Theorem \ref{ree} (d), 
%Lemma \ref{centraliserOfqpm3m+1}
 $[t_1,m^b]\ne 1$ if and only if $\frac{q+1}{4}\not|b$.
Thus $\langle t_1 \rangle ^{t_1^am^bv_1}\cap \langle t_1 \rangle = \{1\}$ if
$\frac{q+1}{4}\not|b$.
However, $\langle t_1 \rangle^{t_1^av_1}=\langle t_1 \rangle ^{v_1}$ and since
$t_1^{v_1}=v_3t_1$, $(t_1^{v_1})^2=v_2t_1^2$, $(t_1^{v_1})^3=t_1^3$,
we have that $\langle t_1 \rangle^{t_1^av_1}\cap \langle t_1 \rangle=\langle t_1^3 \rangle$.

Thus if  $G_0^{(g)}$  contains nontrivial elements from $M_0^{+1}$
and $M_0^{+1}\leq \tilde M$, then $G^{(g)}$ is either $M_0^{+1}$ or $M_0^{+1}\rtimes \langle t_1^3\rangle$. These cases in fact occur.
 If $g\in \tilde M\setminus M_0$, then $G_0^{(g)}=M_0^{+1}$ and if $g\in V\setminus \{1\}$, then $G_0^{(g)}=M_0^{+1}\rtimes \langle t_1^3\rangle$.
\end{proof}



\begin{prop}\label{ungm-} We use Notation \ref{Hjel}. 
If  $G_0^{(g)}$ contains a nontrivial element from $M_0^{-1}$, then $g\in G_0  N_G(M_0^{-1})$.  Furthermore, 
\[U_{N_{G_0}(M_0^{- 1})}=\begin{cases}\{ M_0^{ -1}, M_0^{- 1}\rtimes C_2\}\cup U&\text{if } q_0+1-3m_0|q+1 \\
\{ M_0^{- 1}\}\cup U &\text{otherwise,}
\end{cases}\]
where $U$ may only contain  cyclic subgroups of order $2$ or $1$.
\end{prop}
\begin{proof}
The proof is  similar to the previous one.
\end{proof}


\begin{prop}\label{ungm} We use Notation \ref{Hjel}.
If  $G_0^{(g)}$ contains a nontrivial element from $M_0$, then $g\in G_0  N_G(M_0)$. Furthermore,
\begin{center}$U_{N_{G_0}(M_0)}=\{M_0\times V\}\cup U,$\end{center}
where $U$ may only contain  cyclic subgroups of order $2$ or $1$.
\end{prop}
\begin{proof}
Assume that $g\in G$ such that $G_0^{(g)}$ contains a nontrivial element
 $m\in M_0$.
Then by Lemma \ref{6tolrelpr} a) we have that $M_0 \leq G_0^{(g)}$, hence
$M_0,M_0^{g^{-1}}\leq G_0$.
 Since the Hall subgroups of order $\frac{q_0+1}{4}$  are conjugate in $G_0$,
 there exists an element $r\in G_0$ such that $M_0^{g^{-1}r}= M_0$ and thus
 $r^{-1}g\in N_G(M_0)$. Hence $g\in G_0N_G(M_0)$.\\
To prove  the second part, suppose that  $G_0^{(g)}\leqslant N_{G_0}(M_0)$  for some $g\in G$.
 By Lemma \ref{ungpcomp},  if a  nontrivial $3$-element lies in $G_0^{(g)}$,
 then $G_0^{(g)}$ contains the center of a Sylow $3$-subgroup of $G_0$, and hence it also contains a subgroup isomorphic to
$C_3^2$. This contradicts our assumption that $G_0^{(g)}\leq N_{G_0}(M_0)$. Thus $G_0^{(g)}$ does not contain  nontrivial
$3$-elements.\\
 If $G_0^{(g)}$ does not contain elements whose order divides $\frac{q_0+1}{4}$, then it   is  isomorphic  to $C_2^3$, $C_2^2$,  $C_2$ or $\{1\}$.
 By Lemma \ref{6tolrelpr} d) we know that $C_2^2$ and $C_2^3$ cannot occur.
 Assume that $G_0^{(g)}$ contains a nontrivial  element of $M_0$.
 By the first part of the proof,  there is an element $r\in G_0$
 such that $r^{-1}g\in N_G(M_0)$.
Using Lemma \ref{abcd} b) with $T_0=G_0$, $H=N_G(M_0)$  and $t=r$, we obtain
 that\\
$G_0^{(g)}=(G_0\cap N_G(M_0))^{(r^{-1}g)}=(N_{G_0}(M_0))^{(r^{-1}g)},$
where $r^{-1}g\in N_G(M_0)\setminus G_0$.
By Lemma \ref{6tolrelpr} a) we have that
 $M_0\times V\leqslant (N_{G_0}(M_0))^{(r^{-1}g)}$. We want to prove that
equality holds here.
 Recall that $N_{G_0}(M_0)=(M_0\times V)\rtimes \langle t \rangle$, where $t$ is an element of order $6$ and $N_G(M_0)=(M\times V)\rtimes \langle t\rangle$.\\
If $N_{G_0}(M_0)\cap N_{G_0}(M_0)^{r^{-1}g}> M_0\times V$  holds, then this intersection contains $\langle t^k \rangle ^x$ for some integer $0<k<6$ and $x\in M_0\times V$. Hence it also contains $\langle t^k\rangle$. But then
$\langle t^k \rangle\leq \langle t \rangle ^{r^{-1}gy}$, for some $y\in M_0\times V$.
 Since $r^{-1}g\in N_G(M_0)$, $r^{-1}gy=t^smv$ for some integer $s$,  $v\in V$ and $m\in M\setminus M_0$.
Thus $\langle t^k\rangle\leq \langle t \rangle^{mv}$ and hence
$\langle t^k\rangle= \langle t^k \rangle^{mv}$.
Then $[mv,t^k]\in (M\times V)\cap \langle t^k \rangle =1$.
Thus ${(t^k)}^{m}={(t^k)}^{v}$ and hence ${(t^k)}^{m^2}={t^k}$.
By Lemma  \ref{6tolrelpr} a),  $C_G(m^2)=M\times V$ holds.
This contradicts the fact that  $0<k<6$. Hence
$G_0^{(g)}=N_{G_0}(M_0)^{(r^{-1}g)}= M_0\times V$.

This case in fact occurs.
Let $g\in M\setminus M_0$. Then 
 $G_0^{(g)}\ne G_0$ and  $M_0\times V\leqslant G_0^{(g)}$.
Thus, $G_0^{(g)}\leq N_{G_0}(M_0)$, or $G_0^{(g)}\leq C_{G_0}(i)$, for some involution $i\in G_0$. By the results above,
 in the first case $G_0^{(g)}=M_0\times V$.
 We will show that the second case alone does not occur. Let us suppose by contradiction that
 $M_0\times V\leq G_0^{(g)}\leq C_{G_0}(i)$. Since $g\in M\leq C_G(i)$, 
 $i^{g}=i$.
By Lemma \ref{abcd} b) with $T_0=G_0$, $H=C_{G}(i)$  and $t=1$, we have that
$G_0^{(g)}=(G_0\cap C_G(i))^{(g)}= C_{G_{0}}(i)^{(g)}$.
 Let $C_{G_0}(i)=\langle i \rangle \times L$, where $L\simeq PSL(2,q_0)$.
 
By  \cite[Lemmas 5 and 6]{zll},
  if ${G_0}^{(g)}$ is solvable, then in our case
it is contained in $N_{G_0}(M_0)$, and we are done. If it is nonsolvable, then since it contains $M_0\times V$
it can only be $C_{G_0}(i)=\langle i \rangle \times L$ from the 
list of subgroups.
Thus $C_{G_{0}}(i)^{(g)}=C_{G_0}(i)$. However, by Proposition \ref{ungpcomp}, we know that $G_0^{(g)}=C_G(i)^{(g)}$  contains the center of a Sylow $3$-subgroup, which by Theorem \ref{ree} (b) cannot occur. 
\end{proof}



\begin{prop}\label{urq0}  Let $R_0$  be a maximal subgroup in $G_0$ isomorphic
to $R(q_1)$. If $H\in U_{R_0}$ then  $H\simeq \{1\}$ or  $C_2$.
\end{prop}
\begin{proof}Let $g\in G$ be such that $G_0^{(g)}\leqslant R_0$.
 If $G_0^{(g)}$  contained nontrivial
 elements whose orders divide either $\frac{q_0+1}{4}$,
$\frac{q_0-1}{2}$ or $q_0\pm3m_0+1$, then by
 Lemma \ref{6tolrelpr} a), b) and c) it would contain
 a subgroup isomorphic to $C_2^2\times C_{\frac{q_0+1}{4}}$,
 $C_{q_0-1}$ or $C_{q_0\pm3m_0+1}$, respectively.
Hence $G_0^{(g)}$ could not be inside $R_0$.
 Thus in $G_0^{(g)}$  a  nontrivial element must have
 order divisible by $2$ or $3$.
 If $G_0^{(g)}$ contains a nontrivial   $3$-element, then by Lemma \ref{ungpcomp} a) $Z(P_0)\simeq C_3^{2n_0+1}\leqslant G_0^{(g)}$ and so $G_0^{(g)} $  cannot be
 inside  $R_0$.
 Furthermore, $G_0^{(g)}$ cannot contain more than one involution.
 Otherwise, if it contains two commuting involutions, then  by Lemma \ref{6tolrelpr} d) $C_2^2\times C_{\frac{q_0+1}{4}}\leqslant G_0^{(g)}$, which is a contradiction.
If $G_0^{(g)}$ contains two non-commuting involutions,
then they generate a dihedral group.
It either contains a Klein $4$-subgroup, or an element of odd order.
None of them can occur.
  Thus $G_0^{(g)}$ can only be $\{1\}$ or a cyclic subgroup of order $2$.\end{proof}

Now we determine $U_{C_{G_0}(i)}$.

\begin{prop}\label{ucgi}
Let $i$ be an involution in $G_0$. Then\\
$U_{C_{G_0}(i)}=\{1, \langle k\rangle, W_0, M_0\times V \mid \langle k\rangle \simeq C_2,\ [i,k]=1,\ i\in W_0\simeq C_{q_0-1}, \ i\in V, M_0\times V\simeq C_{\frac{q_0+1}{4}}\times C_2^2\}.$\\
Morover, every  cyclic subgroup  of order $2$ and $q_0-1$ in $G_0$ occurs
as $G_0^{(g)}$ for some suitable $g\in G$.
\end{prop}

\begin{proof}
Let $g\in G$ be such that $G_0^{(g)}\leqslant C_{G_0}(i)$.
Let us suppose that $p\in G_0^{(g)}$ is an element of order $3$ and denote by $P_0$ the Sylow 3-subgroup of $G_0$ which contains $p$. Then by Lemma \ref{ungpcomp} we have that $Z(P_0)\leq G_0^{(g)}\leqslant C_{G_0}(i)$.  
However, by Theorem \ref{ree} (b), $Z(P_0)\cap C_{G_0}(i)=\{1\}$, which is a contradiction.\\
Hence, since $C_{G_0}(i)\simeq \langle i \rangle \times PSL(2,q_0)$,
  the only prime order elements in $G_0^{(g)}$ are  of order $2$
 and of orders that are
 divisors of $\frac{q_0-1}{2}$ or $\frac{q_0+1}{4}$.


 By Lemma \ref{6tolrelpr} a), b) and d) we know that $G_0^{(g)}$ is a cyclic subgroup of order $2$ or contains either a cyclic subgroup of order $q_0-1$ or a subgroup isomorphic to ${C_{\frac{q_0+1}{4}}\times C_2^2}$.

 By Theorem \ref{reszcsoplista},  the subgroups of $C_{G_0}(i)$ which  contain $C_{q_0-1}$ or $C_{\frac{q_0+1}{4}}\times C_2^2$ and do not contain nontrivial $3$-elements
 are isomorphic to $C_{q_0-1}$, $D_{q_0-1}\times C_2$,  $C_{\frac{q_0+1}{4}}\times C_2^2$ or $(C_{\frac{q_0+1}{4}}\times C_2^2)\rtimes C_2$.

 On the other hand, if $G_0^{(g)}\leqslant N_{G_0}(M_0)$, then by Proposition \ref{ungm}, the subgroup $G_0^{(g)}$ can  only be $\{1\}$, $C_2$ or $M_0\times V$.
Hence we can exclude $(C_{\frac{q_0+1}{4}}\times C_2^2)\rtimes C_2$. 

Moreover, according to Lemma \ref{6tolrelpr} d) we can also exclude
 that $G_0^{(g)}\simeq D_{q_0-1}\times C_2$.\\
 Summarizing: $G_0^{(g)}$ is isomorphic to one of the following groups: $1, C_2, C_{q_0-1}, C_{\frac{q_0+1}{4}}\times C_2^2$.

By Proposition \ref{ungm},  the subgroups of $C_{G_0}(i)$
isomorphic to   $C_{\frac{q_0+1}{4}}\times C_2^2\simeq M_0\times V$
  occur as $G_0^{(g)}$ (we may suppose that $g\in M\setminus M_0$ and $i\in V$).
   Now we give a construction for the remaining three cases.\\

Denote by $L$ the factor group $C_G(i)/\langle i\rangle$. Let
$L_0$  be the image of $C_{G_0}(i)$
 under the natural homomorphism $^-:C_G(i)\rightarrow L$.  
By \cite[Lemma 2.16]{F2}, there exists an element
 $\overline{l}_1\in L$ such that $L_0^{(\overline{l}_1)}=\{ 1\}$, since
$2(2n_0+1)<2n+1$.
We will need that more than $\frac{1}{2}q_0^2(q_0-1)^2(q_0+1)^2$ such elements exist. It is  shown in the proof of \cite[Lemma 2.16]{F2} that the number of them is at least
$A=\frac{q(q-1)(q+1)- qq_0^4-2qq_0^3-3qq_0^2+2qq_0+2q+2q_0^5+3q_0^4-2q_0^3-q_0^2}{2}.$
 Observe  that $q_0\leq q/9$, $q_0^2\leq q/3$,  $q_0^3\leq q$ and $(2qq_0+)2q_0^5+5q_0^4-2q_0^3-2q_0^2\geq 0$. Then 
$A-\frac{1}{2}q_0^2(q_0-1)^2(q_0+1)^2\geq\frac{q^3-q- q^3/9-2q^2-3q^2/3+2q-q^2}{2}=\frac{8/9\, q^3 -4\,q^2+q}{2}.$
Since it is  greater than $0$, if $q\geq 27$,  there are more than $\frac{1}{2}q_0^2(q_0-1)^2(q_0+1)^2$ elements $\overline{l}_1\in L$ such that 
$L_0^{(\overline{l}_1)}=\{1\}$. Hence, by taking inverse images, we have that
 there are more than $q_0^2(q_0-1)^2(q_0+1)^2$ elements $l_1\in C_G(i)$ such that $C_{G_0}(i)^{(l_1)}=\langle i \rangle $.\\



We want to prove that there exists an element $\overline l_2\in  L$ such that $L_0^{(\overline l_2)}\simeq C_{\frac{q_0-1}{2}}$. Let $U_0\leq L_0$
be a subgroup isomorphic to $C_{\frac{q_0-1}{2}}$.
By \cite[Theorem 1.2 (ii)]{F2}  we have that $N_L(U_0)\simeq D_{q-1}$ and
$N_{L_0}(U_0)\simeq D_{q_0-1}$. Let us denote  the maximal cyclic subgroup of $N_L(U_0)$ by $U$.  Since $L_0$ is selfnormalizing in $L$, by 
 the proof of \cite[Lemma 2.16]{F2}, thus if $\overline l_2\in
U\setminus U_0$ then $U_0\leq L_0^{(\overline l_2)}\ne L_0$. 
 Hence, $U_0=L_0^{(\overline l_2)}$, by
\cite[Lemma 2.15]{F2}.
Taking an inverse image $l_2$ of   $\overline l_2$, 
 we have that
  $(C_{G_0}(i))^{(l_2)}\simeq \langle i\rangle \times C_{\frac{q_0-1}{2}}$.\\
We show that  for the above $l_1$ and $l_2$, 
if $G_0^{(l_1)}\leqslant C_G(i)$ then $G_0^{(l_1)}\simeq C_2$
and  if  $G_0^{(l_2)}\leqslant C_{G_0}(i)$, then  
 $G_0^{(l_2)}\simeq C_{q_0-1}$.
 Applying  Lemma \ref{abcd} c) with 
$H=\langle i \rangle$, $r=1$ and $g=l_1$ or $g=l_2$, respectively,
 we have that
   $G_0^{(l_1)}\cap C_{G_0}( i )=G_0^{(l_1)}\cap N_{G}(\langle i \rangle)=N_{G_0}(\langle i \rangle )^{(l_1)}=C_{G_0}(i )^{(l_1)}=\langle i \rangle$ and $G_0^{(l_2)}\cap C_{G_0}(i)=C_{G_0}(i)^{(l_2)}\simeq C_{q_0-1}$.
 Hence  if $G_0^{(l_1)}\leqslant C_G(i)$, then $G_0^{(l_1)}=\langle i \rangle $,
 and if $G_0^{(l_2)}\leqslant C_{G_0}(i)$, then
 $G_0^{(l_2)}\simeq C_{q_0-1}$, containing
 $i$. 
For another involution $k\in C_{G_0}(i)$ we may choose an element $x\in G_0$
such that $i^x=k$, then $\langle k\rangle=G_0^{(l_1x)}$. So we are done with the first statement of the Proposition in this case.\\
 Let us suppose that for some of  the above $l_1,l_2$, 
$G_0^{(l_1)}$ and $G_0^{(l_2)}$ are  not  contained $C_{G_0}(i)$. 
  Since both $G_0^{(l_1)}$ and $G_0^{(l_2)}$ contain $ i$, by Propositions \ref{ungp}, \ref{ungm+}, \ref{ungm-}, \ref{ungm} and \ref{urq0}, depending on which maximal subgroup of $G_0$ contains them, they are isomorphic to one of the
subgroups of the first row of the table below.
The intersection  $G_0^{(l_j)}\cap C_{G_0}(i)$  is isomorphic to the
to the corresponding subgroups in the second  row:\\  
\begin{center}
\begin{tabular}{l||l|l|l|l|l|l}
$G_0^{(l_j)}$ & $C_2$ & $C_{q_0-1}$& $P_0'\rtimes C_2$ & $P_0\rtimes C_2$
 &  $P_0\rtimes C_{q_0-1}$& $C_2^2\times C_\frac{q_0+1}{4}$\\
% $C_{q_0+3m_0+1}\rtimes C_2$ &$C_{q_0-3m_0+1}\rtimes C_2$  & $G_0$\\
\hline
$G_0^{(l_j)}\cap C_{G_0}(i)$ & $C_2$ & $C_{q_0-1}$&  $C_ 3^{2n_0+1}\rtimes C_2$ & $C_3^{2n_0+1}\rtimes C_2$ &
$C_3^{2n_0+1}\rtimes C_{q_0-1}$ & $C_2^2\times C_{\frac{q_0+1}{4}}$\\
%  $C_2$ & $C_2$ & $C_{G_0}(i)$.
\end{tabular} 
\end{center}
\begin{center} 
\begin{tabular}{l|l|l}
$C_{q_0+3m_0+1}\rtimes C_2$ &$C_{q_0-3m_0+1}\rtimes C_2$  & $G_0$\\
\hline
$C_2$ & $C_2$ & $C_{G_0}(i)$.
\end{tabular}
\end{center}
%The intersection of one of these subgroups with the centralizer $C_{G_0}(i)$ of one of their involutions $i$, is isomorphic to the following: $C_2, C_{q_0-1}, C_ 3^{2n_0+1}\rtimes C_2,C_3^{2n_0+1}\rtimes C_2, C_3^{2n_0+1}\rtimes C_{q_0-1}, C_2^2\times C_{\frac{q_0+1}{4}}, C_2,C_2, C_{G_0}(i)$.\\

 Since this intersection must be $C_2$ or $C_{q_0-1}$,
 this implies that the only  possibilities are: ${G_0}^{(l_2)}\simeq C_{q_0-1}$, 
 $G_0^{(l_1)}=\langle i\rangle$  or $G_0^{(l_1)}=M_0^{\pm1}\rtimes \langle i\rangle$ for some $M_0^{\pm1}\in Hall_{q_0\pm3m+1}(G_0)$.
In Propositions \ref{ungm+}
 and \ref{ungm-} we have seen that if $G_0^{(g)}\simeq C_{q_0\pm3m_0+1}\rtimes C_2$, then $(M_0)^{\pm1}\leqslant \tilde M\in Hall_\frac{q+1}{4}(G)$ and $g\in G_0(V\setminus \{1\})$, where $C_2^2\simeq V\leqslant C_G(\tilde M)$.
 There are $$\frac{|C_{G_0}(i)|}{|N_{C_{G_0}(i)}(M_0^{\pm1})|}=\frac{q_0(q_0-1)(q_0+1)}{6}$$ subgroups in  $G_0$,  
 isomorphic to $C_{q_0\pm3m_0+1}$ whose 
 normalizer contains $i$.  Since $V\leqslant C_G(i)$, there are at most $2\frac{q_0(q_0-1)(q_0+1)}{6}|C_{G_0}(i)(V\setminus{1})|=q_0^2(q_0-1)^2(q_0+1)^2$
 elements $g\in C_G(i)$ such that $G_0^{(g)}\simeq C_{q_0\pm3m_0+1}\rtimes C_2$.
 However, we have seen that there are more elements $l_1\in C_G(i)$ such that $C_{G_0}(i)^{(l_1)}=\langle i\rangle$.
This implies that there is an element $l_1\in C_G(i)$ such that  $G_0^{(l_1)}=\langle i\rangle$. We also have seen that  $G_0^{(l_2)}\simeq C_{q_0-1}$ holds for some $l_2\in C_G(i)$. 

 Since both the involutions and cyclic subgroups of order $q_0-1$ are conjugate in $G_0$, every cyclic subgroup of order $2$ and $q_0-1$ occurs as $G_0^{(g)}$ for some $g\in G$.\\

Finally, we will show that there exists an element $g\in G$ such that $G_0^{(g)}=1$. Looking at the order of $G_0$, by Lemma \ref{6tolrelpr} it is enough to
 show that there is an element $g$ in $G$ such that $G_0^{(g)}$  contains
 neither an involution $i\in G_0$, nor elements from $P_0$ and $M_0^{\pm1}$ for every $P_0\in Syl_3(G_0)$ and $M_0^{\pm 1}\in Hall_{q_0\pm3m_0+1}(G_0)$.
 If $G_0^{(g)}$ contains elements from $P_0$ or $M_0^{\pm1}$, then by Lemma \ref{ungpcomp},  Proposition \ref{ungm+} and Proposition \ref{ungm-},
 $g\in G_0 N_G(P)$, (where $P$ is the unique Sylow $p$-subgroup of $G$ containing $P_0$) or $g\in G_0 N_G(M_0^{\pm1})$, respectively. If the involution $i$ is in $G_0^{(g)}$, then using Lemma \ref{abcd} a) with $H_0=\langle i\rangle $ and $T_0=G_0$ we have that, $g\in G_0 C_G(i)$. Thus, we can
 give an upper bound to the cardinality\\
$\bigcup_{P_0\in Syl_3(G_0)} G_0 N_G(P)\cup\bigcup_{M_0^{\pm1}\in Hall_{q_0\pm3m_0+1}(G_0)}G_0 N_G(M_0^{\pm1})\cup \bigcup_{i\in G_0, \ o(i)=2}G_0 C_G(i)|\leqslant $ 
\\$
|Syl_3(G_0)|\cdot |G_0 ||N_G(P)|/|N_{G_0}(P)|+$ $\sum_{M_0^{+1}\in Hall_{q_0+3m0+1}(G_0)}|G_0|| N_G(M_0^{+1})|/|N_{G_0}(M_0^{+1})|+$ 
\\
 $\sum_{M_0^{-1}\in Hall_{q_0-3m0+1}(G_0)}|G_0|| N_G(M_0^{-1})|/|N_{G_0}(M_0^{-1})|+$ 
$\sum_{i\in G_0, \ o(i)=2}|G_0|| C_G(i)|/|C_{G_0}(i)|=$\\
 $|G_0 | ([G_0:N_{G_0}(P_0)]\frac{|N_G(P)|}{|N_{G_0}(P_0)|}|+$ $\sum_{i=\pm1}|
G_0:N_{G_0}(M_0^i)|\frac{|N_G(M_0^{i})|}{|N_{G_0}(M_0^{i})|}+
 [G_0:C_{G_0}(i)]\frac{|C_G(i)|}{|C_{G_0}(i)|} )$

 By Theorem \ref{ree} (d) and (j),\\ 
$|G_0:N_{G_0}(M_0^{\pm})|\leqslant \frac{q_0^3(q_0^2-1)(q_0+3m_0+1)}{6},
 |N_G(M_0^{\pm1})|\leqslant |N_G(M^{+1})|=6(q+3m+1),$\\
and $|G_0 \cap N_G(M_0^{\pm1})|\geqslant |N_{G_0}(M_0^{-1})|=6(q_0-3m_0+1).$\\
 Thus the cardinality of the above set is at most:\\
$q_0^3(q_0-1)(q_0^3+1)\Big((q_0^3+1)\cdot\frac{q^3(q-1)}{q_0^3(q_0-1)}+2\frac{q_0^3(q_0^2-1)(q_0+3m_0+1)}{6}\cdot\frac{ 6(q+3m+1)}{6(q_0-3m_0+1)}+$$q_0^2(q_0^2-q_0+1)\frac{ q (q-1)(q+1)}{q_0 (q_0-1)(q_0+1)}\Big)=$
$(q_0^3+1)^2q^3(q-1)+\frac{1}{3}q_0^6(q_0^2-1)^2(q_0+3m_0+1)^2(q+3m+1)+q_0^4(q_0^2-q_0+1)^2q(q-1)(q+1)$.\\
A naive upper bound for this is:\\
$(2q_0^3)^2q^3(q-1)+\frac{1}{3}q_0^6(q_0^2)^2(3q_0)^2(3q)+q_0^4(q_0^2)^2q(q-1)(2q)=$
$(q-1)(4q_0^6q^3+\frac{1}{3}q_0^{12}\frac{q}{q-1}27+2q_0^8q^2).$
Using that $q>3$ (actually we have $q\geqslant 27$), we have that
$\frac{q}{q-1}\leqslant \frac{3}{2}$ and $q_0^3\leq q$, and so the cardinality of the above set is at most
$(q-1)(4q^5 + 27/2q^4+2q^5)\leqslant 20(q-1)q^5$. For $q\geqslant 27$
this is obviously smaller than $|G|=(q-1)q^3(q^3+1)$, and as we have seen before, this implies that there is an element $g\in G$ such that $G_0^{(g)}=\{1\}$.
\end{proof}

\begin{tetel}\label{depthr(q0)} For maximal subgroups $G_0\simeq R(q_0)$ in $G=R(q)$, where $q_0>3$ we have that   $\mathcal{U}_1(G_0)=\mathcal{U}_2(G_0)$,  $d_c(G_0,G)=4$ and $d(G_0,G)=3$.
\end{tetel}
\begin{proof}
We use Notations \ref{3jel} and  \ref{Hjel}.
 By Proposition \ref{ungp}, \ref{ungm+}, \ref{ungm-} \ref{urq0} and \ref{ucgi} we know that
  either $\mathcal{U}_{\;1}(G_0)=\bigcup_{P_0\in Syl_3(G_0)}U_{N_{G_0}(P_0)}\cup\{H \mid H\simeq 1, C_2, C_{q_0-1}, C_{q_0+1\pm3m_0}, C_2^2\times C_{\frac{q_0+1}{4}}\}$  or 
 $C_{q_0+1\pm3m_0}\rtimes C_2$ also belong to it.
    Since $G_0^{(g_1,g_2)}=G_0^{(g_1)}\cap G_0^{(g_2)}$, the set $\mathcal {U}_{\;1}(G_0)$ is closed under intersection if and only if $\mathcal{U}_{\;1}(G_0)=\mathcal{U}_{\;2}(G_0)$.
 It is easy to see, that $\mathcal{U}_{\;1}(G_0)$  will be intersection closed, if $U_{N_{G_0}}(P_0)$ is intersection closed for every $P_0\in Syl_3(G_0)$. To show this, we display the subgroups  according to Proposition \ref{ungp} and \ref{ucgi}, boxing those subgroups which are  proved to be contained in
 $\mathcal{U}_{\;1}(G_0)$. In Section \ref{subd} we will see, that $P_0$ also can be in a  box.\\
{\footnotesize{ \small \xymatrix{
 & & & &   & P_0W_0      &\\
 & & && P_0\langle i\rangle\ar@{-}[ur]\\
 & & & *+[F]{P_0}\ar@{-}[ur] & *+[F]{P_0'\langle i\rangle }\ar@{-}[u]& \dots & *+[F]{P_0'\langle i^{p_0}\rangle}\ar@{-}[ull]\\
 \langle Z(P_0),h_1\rangle\ar@{-}[urrr]&\dots &\langle Z(P_0),h_k\rangle\ar@{-}[ur] &*+[F]{ P_0'}\ar@{-}[u]\ar@{-}[ur]\ar@{-}[urrr]   &   & *+[F]{W_0}\ar@{-}[uuu]&      &*+[F]{W_0^{p_0}}\ar@{-}[uuull]\\
  & & & *+[F]{Z(P_0)}\ar@{-}[u]\ar@{-}[ul]\ar@{-}[ulll]& *+[F]{\langle i\rangle}\ar@{-}[uu]\ar@{-}[ur] & \dots & *+[F]{\langle i^{p_0}\rangle}\ar@{-}[uu]\ar@{-}[ur]\\
   & & &  *+[F]{1}\ar@{-}[u] \ar@{-}[ur]\ar@{-}[urrr]  &   &       &
}}} \\

It can be checked  that if the intersection of any two subgroups in the above picture is different from any of the two subgroups,  then  the intersection  is a boxed subgroup. Hence it is enough to prove
 that  the  set of boxed subgroups is 
 closed under intersection. Using  condition (1) in Definition \ref{jellemzes},  we have that $d_c(G_0,G)\leqslant 4$. Using condition (2) in Definition \ref{jellemzes},  we show that $d_c(G_0, G)>3$.

  Let $x_2\in G$ be such that $G_0^{(x_2)}=P_0'$ (actually we can choose any proper subgroup of $G_0$ from $\mathcal{U}_{\;1}(G_0)$ instead of $P_0'$) and let $x_1\in G_0\setminus N_{G_0}(P_0')$. Then $G_0^{(x_1,x_2)}=P_0'$. Assume by contradiction
 that there exists an element
 $y_1\in G$ such that $G_0^{(x_1,x_2)}=G_0^{(y_1)}$ and $h^{x_1}=h^{y_1}$ for all $h\in P_0'$.
Since $P_0'^{x_1}=P_0'^{y_1}$, we have that $P_0'^{x_1}\leqslant G_0^{y_1}$ and so $P_0'^{x_1}\leqslant G_0^{(y_1)}=P_0'$. By the choice of $x_1$, i. e. $x_1 \notin N_G(P_0')$,  we have a contradiction. Thus $d_c(G_0, G)>3$ and hence
$d_c(G_0, G)=4$.
Using Theorem \ref{inter}, we  obtain $d(G_0,G)=3$.
\end{proof}

\begin{rem} With similar arguments we  proved that if $G_0\simeq R(3)$
then $d_c(G_0,G)=4, d(G_0,G)=3$ also holds. Moreover, there exists an element $x\in G$ with $G_0\cap G_0^x=\{1\}$. 
  We omit the proof. For more details, see \cite{HHP2016}.
\end{rem}

\section{Proof of Theorem \ref{tsubd}}
\label{subd}

In this section we prove Theorem \ref{tsubd} where we determine the so called subdegrees, the sizes of orbits of  point stabilizers, in primitive actions of $G=R(q)$.
If $M$ is a maximal subgroup of $G$ then ${\mathcal U}_1(M)=\{ M\cap M^x \mid x\in G \}$. The subdegrees of primitive actions of $G$ can be produced as the set of  numbers $|M:M\cap M^x|$, where $M$ is maximal in $G$ and $x\in G$ arbitrary.
\\
To prove Theorem \ref{tsubd} it is enough to detemine  ${\mathcal U}_1(M)$ for
each maximal subgroup $M\leq G$. This was almost done in the previous sections, in this section we extend the results with the missing cases.

%\begin{center}
%\begin{tabular}{c|p{12cm}}
% $M$ & subdegrees \\
%\hline
%$N_G(P)$ &  $1$,   $q^3$\\
%\hline
%$N_G(M^{\pm 1})$ & $1$, $6(q+1\pm 3^n)$, $3(q+1\pm 3^n)$, $2(q+1\pm 3^n)$, $(q+1\pm 3^n)$\\
%\hline
%$N_G(M)$ & $1$, $6(q+1)$, $3(q+1)$, $2(q+1)$, $(q+1)$\\
%\hline
%$C_G(i)$ & $1$, $(q-1)(q+1)$, $\frac{q(q-1)}{2}$, $\frac{q(q-1)(q+1)}{4}$,
%$\frac{q(q-1)(q+1)}{2}$, $q(q-1)(q+1)$\\
%\hline
%$G_0$ & $1$, $(q_0^3+1)q_0^3(q_0-1)$, $(q_0^3+1)(q_0-1)$,
%        $(q_0^3+1)q_0(q_0-1)$,  $\frac{(q_0^3+1)q_0(q_0-1)}{2}$, \\
%      & $(q_0^3+1)q_0^2(q_0-1)$, $*\frac{(q_0^3+1)q_0^2(q_0-1)}{3}$,
%        $\frac{(q_0^3+1)q_0^3(q_0-1)}{2}$,
%        $(q_0^3+1)q_0^3$, \\
%      & $(q_0+1-3^{n_0+1})q_0^3(q_0-1)$,
%        $**\frac{(q_0+1-3^{n_0+1})q_0^3(q_0-1)}{2}$, \\
%      & $(q_0+1+3^{n_0+1})q_0^3(q_0-1)$, $***\frac{(q_0+1+3^{n_0+1})q_0^3(q_0-1)}{2}$, $(q_0^2-q_0+1)q_0^3(q_0-1)$  
%\end{tabular}
%\end{center}
% Here the value at  $*$ can occur only if $2n+1=3(2n_0+1)$ and certain equations  over  $GF(q_0)$ have solutions in $GF(q)\setminus GF(q_0)$.
%The value at $**$ occurs  if and only if $q_0+1+3^{n_0+1}|q+1$ and the value at% $***$ occurs  if and only if  $q_0+1-3^{n_0+1}|q+1$.\\
%\bigskip


%\centerline{\it Explanations}
\bigskip\noindent
$\mathcal{U}_1(N_G(P))= \{N_G(P), W\leqslant N_G(P)\ \mid \ W \simeq C_{q-1}\ \}$, since 
$N_G(P)^{(x)}= G_1\cap G_{(1)x}=G_{1,(1)x}\simeq C_{q-1} \mbox{ or }N_G(P)$ (see Theorem \ref{ree} (b))\\[1cm] 
$\mathcal{U}_1(N_G(M^{\pm 1}))=\{N_G(M^{\pm 1}), 1\}\cup \{H \leqslant N_G(M^{\pm 1}) \mid H\simeq C_2,C_3, C_6\}$ (see Theorem \ref{ree} (d) and Prop. \ref{4.1})\\[0.2cm]
%$\mathcal{U}_1(B^{-1})=\{B^{-1}, 1\}\cup\{H \leqslant B^{-1} \mid H\simeq C_2,C_3, C_6\}$ (see Prop. i\ref{4.1})\\[0.2cm]
We deal only with $B^1=N_G(M^1)$,  for $B^{-1}=N_G(M^{-1})$ the proof is similar.\\
We prove that the missing cases $C_2$, $C_3$ and $C_6$ all occur:\\
\begin{itemize}
\item[$C_2$:]
{\bf Let $i$ be an involution in $B^1$ and $x$ be an element in $C_G(i)$, whose order is $q-1$. We show that $(B^1)^{(x)}=\langle i\rangle$.}\\
Since $x\notin B^1$, $(B^1)^{(x)}< B^1$. Hence  by the proof of
 Proposition \ref{4.1} 
%(valszeg szebb strukt\'ur\'at k\'ene kital\'alni)
 $(B^1)^{(x)}\lesssim C_6$. Assume that the intersection is a subgroup $U$ which
is isomorphic to $C_6$.
Then $U, U^{x^{-1}} \leq B^1$, hence for a suitable element $m_1\in M^1$ we have that
$U^{m_1}=U^{x^{-1}}$, hence $U^{m_1x}=U$, in other words $m_1x\in N_G(U)$. Since $i\in U$, we have that $m_1x\in C_G(i)$, thus $i^{m_1x}=i$. Since $i^x=i$, we have that $i^{m_1}=i$, which implies that $m_1=1$, since $B_1$ is a Frobenius group
with kernel $M^1$. Thus $x\in N_G(U)$. Let $T\leq U$ be the subgroup of order $3$. Then $x\in N_G(U)\leq N_G(T)=C_G(T)$ (by  Cor. \ref{centrp} (ii)). This is a contradiction since $|C_G(T)|=2q^2$, and $o(x)=q-1$.
As $i\in {(B^1)}^{(x)}$, we have that ${(B^1)}^{(x)}=\langle i \rangle$.

% Denote by $T$ the subgroup of order 3 of $(B^1)^{(x)}\simeq C_6$, then $[T,i]=1$. By Proposition 4.2 $x\in B^1N_G(T)=M^1N_G(T)$, i.e. exists $m\in M^1$ and $n\in N_G(T)$, with $x=mn$.
%\[i=i^x=i^{mn}\] %\leftrightarrow
%thus $i^m=i^{n^{-1}}=i$.
%Since $m\in C_G(i)$, $m$ can be just $1$, thus $x=n\in N_G(T)$. $N_G(T)$ does not contain elements with order $q-1$, thus we got a contradiction.\\[0.2cm]
\item[$C_3$:]
{\bf Let $T$ be a subgroup of order 3 in $B^1$ and $x$ be a nontrivial element in $Z(P)$,  such that $T\leqslant P\in Syl_3(G)$.
We show that $(B^1)^{(x)}=T$.}\\
Since $x\notin B^1$, $(B^1)^{(x)}< B^1$. Hence as above, we have  that
 $(B^1)^{(x)}\lesssim C_6$. Assume that the intersection is  a subgroup $U$ 
isomorphic to $C_6$. Let $i$ be the involution in $U$.
As in the proof above we have that  there exists an element $m_1\in M^1$ such that
$m_1x\in N_G(U)\leq N_G(T)=C_G(T)$. Since $x\in C_G(T)$, we have that
$m_1\in C_G(T)$, which implies that $m_1=1$ and $x\in N_G(U)\leq C_G(i)$. This is a contradiction, since $x\in  Z(P)\setminus \{1\}$.
As $T\leq {(B^1)}^{(x)}$, we have that  ${(B^1)}^{(x)}=T$.

 %$Denote by $i$ the involution of $(B^1)^{(x)}\simeq C_6$, then $[T,i]=1$. We have seen that $x\in B^1N_G(T)$, i.e. exists $m\in M^1$ and $c\in C_G(i)$, where $x=mc$.
%\[T=T^x=T^{mc} \leftrightarrow T^m=T^{c^{-1}}=T\]
%Since $m\in N_G(T)$, $m$ can be just $1$, thus $x=c\in C_G(i)$. $C_G(i)$ does not contain elements from a center of a Sylow $3$-subgroup, thus we got a contradiction.\\[0.2cm]
\item[$C_6$:] {\bf  Let $\langle h \rangle$ be a subgroup of order 6 in $B^1$. Denote by $P$ the Sylow $3$-subgroup, which contains $h^2$. Since $B^1\cap P= \langle h^2\rangle$, and by Cor. \ref{centrp} (ii) $N_G(\langle h^2\rangle)\cap P= P'$, furthermore,   by Theorem \ref{ree} (b), $C_G(\langle h^3\rangle)\cap P\simeq C_3^{2n+1}\leqslant P'\setminus Z(P)$, we can choose $x\in (N_G(\langle h^2\rangle)\cap C_G(\langle h^3\rangle))\setminus B^1 $. Thus $h^2,h^3\in (B^1)^{(x)}< B^1$, it means that $(B^1)^{(x)}=\langle h \rangle$. }
\end{itemize}
 \bigskip

\noindent
$\mathcal{U}_{\,1}(N_G(M))=\{N_G(M)=N_G(V), S, 1,  H\leqslant N_G(M) \mid  V\leqslant S\in Syl_2(G),\ [H,V]=1,\ V\neq H\simeq C_2^2 \}\cup 
\{H \leqslant N_G(M) \mid H\simeq C_2,C_3, C_6\}$ (see  Theorem \ref{ree} (e), Prop. \ref{NGMu1e} and  \ref{5.6})\\
 We prove that the  missing cases $C_2$, $C_3$ and $C_6$ all occur:\\



%$N_G(M)=V\rtimes (M\rtimes C_6)$. For $M\rtimes C_6$ similar thing can be done like for $B^1$. One has to take care about $V$. How can it be done???


%We know that  $N_G(M)=(V\times (M\rtimes \langle t_2 \rangle ))\rtimes \langle t_3 \rangle $, where $o(t)=2$, $o(t_3)=3$, $[t_2,t_3]=1$.
% Let us denote $t_2$ by $i$.
%Let $x\in C_G(i)$ of order $\frac{q-1}{2}$. Since $N_G(V)=N_G(M)$  and its order is relatively prime to $\frac{q-1}{2}$, we have that $V^x\ne V$.
%We will prove that for a suitable  power $x^k$ of $x$ even $V^{x^k}\cap V=\{ 1\}$ holds.
%Let $V=\{1,i_1,i_2,i_3\}$. If the conjugtion by $x$ does not leave any nontrivial element of $V$
%inside $V$, then $V^x\cap V=\{1\}$.  The element $x$ cannot centralize any nontrivial element in $V$, since then $x\in C_G(H)$, for a Klein four group $H$ containing $i$ and the centralizer of a Klein four subgroup in $G$ has order relatively prime to $\frac{q-1}{2}$. 
% So we may assume that $i_1^x=i_2$.
%Then   $i_2^x\not\in V$ and $i_3^x\not\in V$, otherwise $x\in N_G(V)$, which is not the case.
%Then $i_1^{x^2}=i_2^x\not\in V$
%\begin{itemize}
%\item{(i)}Let us suppose that $i_2^{x^2}=i_3$.
% Then $i_2^{x^3}=i_3^x \not\in V $ 
%and  $i_1^{x^3}=i_2^{x^2}=i_3$.
%Hence $i_2^{x^3},i_3^{x^3}\not\in V$.
%Then $i_1^{x^4}=i_2^{x^3}\not\in V$.
%Furthermore, $i_1^{x^5}=i_2^{x^4}=i_3^{x^2}\ne i_3$, it also cannot be
%$i_1$ and $i_2$. Thus $i_2^{x^4}\not\in V$.
%Moreover,  $i_3\ne i_3^{x^4}=i_2^{x^6}=i_1^{x^7}$, which cannot be $i_1$ and $i_2$. Hence $i_3^{x^4}\not\in V$ and we have that $V\cap V^{x^4}=\{1\}$.
%\item{(ii)} If $i_2^{x^2}\not\in V$. Then $i_1^{x^2}=i_2^x\not\in V$.
%\begin{itemize}
%\item{a}  If $i_3^{x^2}\not\in V$ then $V^{x^2}\cap V=\{1\}$.
%\item{b} If $i_3^{x^2}=i_1$ then $i_3^{x^3}=i_1^x=i_2$ and $i_3^{x^4}=i_2^x\not\in V$. Moreover, $i_3\ne i_3^{x^6}=i_1^{x^4}=i_2^{x^3}$ which cannot be $i_1$ and $i_2$. Hence $i_1^{x^4}\not\in V$. And $i_2^{x^4}=i_1^{x^5}=i_3^{x^7}$ which cannot be any of $i_1, i_2, i_3$. Thus $V\cap V^{x^4}=\{1\}$.
%\item{c} If $i_3^{x^2}=i_2$ then ${i_1}^{x^2}\not\in V$ and $i_2^{x^2}\not\in V$. Also $i_3^{x^3}=i_2^x\not\in V$. Moreover, $i_1^{x^4}=i_2^{x^3}=i_3^{x^5}\not\in V$. Furthermore, $i_1^{x^3}=i_2^{x^2}=i_3^{x^4}\not\in V$.
%Hence, $V\cap V^{x^3}=\{1\}$.
%\end{itemize}
%\end{itemize}

%Let us exchange $x$ if necessary with a power so that $V\cap V^x=\{1\}$.
%Then $o(x)|\frac{q-1}{2}$. If $q>3$ then $x^5$ and $x^7$ cannot be $1$, so we can construct such an $x$ as above.\\

% Now we show that  $N_G(M)^{(x)}$ cannot be a Sylow $2$-subgroup of $G$.
%Otherwise we have that  $S\leq N_G(V)$ and $S\leq N_G(V)^x$, and hence $V\leq S$ and $V^x\leq S$, which is impossible since $V\cap V^x=\{1\}$.
%\\

\begin{itemize}
\item[$C_2:$]

We use from  page \pageref{C2} that every order $2$ subgroup $\langle k \rangle $ is the intersection of $C_G(i)\cap C_G(j)$ for suitable involutions $i,j\in G$.
Let $k\in S_1\in Syl_2(C_G(i))$ containing $i$.
Let $k\in S_2\in Syl_2(C_G(j))$ containing $j$.
Then $S_1\cap S_2 =\langle k \rangle $.
Let $\langle i,k \rangle :=V_1$, $\langle j,k\rangle :=V_2$.
We show that $N_G(V_1)\cap N_G(V_2)=\langle k \rangle $.
Let us suppose by contradiction that there is another involution $i'$ in this intersection.  Then $\langle i',V_1\rangle=:\tilde {S_1}\leq C_G(i)$
and $\langle i',V_2 \rangle =:\tilde {S_2}\leq C_G(j)$. But then $i' \in\tilde {S_1}\cap \tilde {S_2}\leq C_G(i)\cap C_G(j)=\langle k \rangle $, which is a contradiction.
Thus the intersection cannot contain another involution. If it would contain an order $3$ element then it would permute the involutions of $V_1$ and $V_2$, and again we get more than $1$ involution in the intersection.
Since $V_1\ne V_2$ the intersection cannot be $N_G(M)$. Hence it can only be of order $2$.
\\
%Question: If $V:=V_1$ and $V^x=V_2$ then we have that $V\cap V^x\ne 1$.
%What kind of element can be $x$? 
% In the proof of page 11
%$i$ and $j$ are two involutions in a dihadral group $ D_{q-1}$. So they are conjugate by an element from $C_{\frac{q-1}{2}}$ and they are in the centralizer of $k$. So $C_G(k)$ is of order  $56$. This contradict to the calculations
%where if the intercection of $V$ and $V^x$ was nontrivial then the normalizers were intersection in a Klein four group.
%What is wrong?
\item[$C_3:$]

We know that $N_G(M)=V\times (M\rtimes C_2)\rtimes C_3=V\rtimes (M\rtimes C_6)$.
Since subgroups of order $3$ are all conjugate in $N_G(M)$, their centralizers in $N_G(M)$ are all conjugate. Hence
all subgroups of order $6$ in $N_G(M)$ are conjugate. We can do similar thing like in the case of $B^1$. Let $T\leq N_G(M)$ be a subgroup of order $3$. Let $T\leq P\in Syl_3(G)$. Let $x\in Z(P)$ a nontrivial element. By Cor \ref{centrp} $x\not\in N_G(M)$, hence $N_G(M)^{(x)}<N_G(M)$. We also have that $T\leq N_G(M)^{(x)}$. Then $N_G(M)^{(x)}$ can only be of order $3$ or of order $6$.
Let us suppose by contradiction that  $N_G(M)^{(x)}=U$ of order $6$. Then
$U,U^{x^{-1}}\leq N_G(M)$. Hence there exists an element $n\in N_G(M)$ such taht $U^{xn}=U$. Then $xn\in N_G(U)\leq N_G(T)=C_G(T)$ by  Cor. \ref{centrp} (ii).
 Since $x\in C_G(T)$ we have that $n\in C_{N_G(M)}(T)=U$ and hence $x\in N_G(U)\leq C_G(i)$ for the involution  $i$ of $U$. This contradicts to the fact that
$x\in Z(P)\setminus\{1\}$ by Cor. \ref{centrp}. 
Hence $N_G(M)^{(x)}=T$.

\item[$C_6:$] Let $\langle  h \rangle $ be a subgroup of order $6$ in $N_G(M)$. Denote $P$ the Sylow $3$-subgroup containing $h^2$. Then $N_G(M)\cap P=\langle h^2 \rangle $. By  Cor.\ref{centrp} (ii)  we have that
 $N_G(\langle h^2 \rangle )\cap P=P'$. By Theorem \ref{ree} (b) and Cor. \ref{centrp}  $C_G(h^3)\cap P\simeq C_3^{2n+1}\leq P'\setminus Z(P)$. Thus we can choose an element
 $x\in N_G(\langle h^2 \rangle )\cap C_G(\langle h^3 \rangle)\setminus N_G(M)$.
 Thus $h^2,h^3\in N_G(M)^{(x)}$, hence $N_G(M)^{(x)}=\langle h \rangle $.
\end{itemize}











%aiow we show that $N_G(M)^{(x)}$ cannot be a Klein four subgroup of $G$.
%We know that only $H\simeq C_2^2$, $H\ne V$ and  $[H,V]=1$ might occur.
%\\
%Suppose that $N_G(M)\cap N_G(M)^x=H$. Then $i\in H$ and $H\cap V=\langle u \rangle $ of order $2$. Then $u^{x^-1}\not in V$, since $V\cap V^x=\{1\}$.
%Furthermore, $[u^{x^-1},i]=1$. Thus $u^{x^-1}=i_1^ai_2^bi$.
%Thus $u=(i_1^a)^x(i_2^b)^xi$. Here $i_1^x=i_1^ri_2^sk$, $i_2^x=i_1^mi_2^nl$,where
%$k,l\in M\rtimes C_2$. Hence $u=(i_1^ri_2^sk)^a(i_1^mi_2^nl)^bi=
%i_1^{ra+mb}i_2^{sa+nb)(k^a)(l^b)i\in V$. Thus $k^al^bi=1$. However $k\ne 1 $,  $l\ne 1$ and $kl\ne 1$, since $i_1^x,i_2^x, (i_1i_2)^x\not in V$.
%But




\bigskip



\noindent
$\mathcal{U}_1(C_G(i))=\{C_G(i),C_P(i),C_G(V),K,C,1 \mid i\in N_G(P), P\in Syl_3(G), i\in V\simeq C_2^2, K\simeq C_2^2, [i,K]=1, i\not\in K, C\simeq C_2, [ i,C]=1, i\not\in C\}$. (see Prop.  \ref{U1} )\\



\noindent
$\mathcal{U}_1(G_0)=\{G_0, 1\}\cup \{U_{N_{G_0}(P_0)}\cup U_{N_{G_0}(M_0^{+1})}\cup U_{N_{G_0}(M_0^{-1})}\cup U_{ N_{G_0}(M_0)}\cup U_{R_0}\cup U_{C_{G_0}(i)}\cup \{C_2, C_{q_0-1} \}$
(see Theorem \ref{ree} (i), Prop.  \ref{ungp}, \ref{ungm+}, \ref{ungm-}, \ref{ungm}, \ref{urq0}, \ref{ucgi})\\
These are up to isomorphism:\\
$\{G_0, 1, P_0, P_0',P_0'\langle i \rangle , Z(P_0), *Z(P_0)\langle h \rangle ,
C_2, C_{q_0-1} , M_0^{+1}, **M_0^{+1}\rtimes C_2, M_0^{-1}, *** M_0^{-1}\rtimes C_2, M_0\times V \}$

Those with stars occur under certain conditions, see  Theorem \ref{tsubd}.\\

Missing cases: to determine if the subgroups $Z(P_0)\langle h \rangle$, where $h\in P_0$ and  $o(h)=9$, $P_0$, $P_0\langle i \rangle $, $P_0W_0$ occur as $G_0^{(x)}$.
{\bf  We prove that $P_0$ occurs, $P_0\langle i \rangle $  and  $P_0W_0$ do not occur, and
$Z(P_0)\langle h \rangle$  just in special cases  might occur, depending on the values of $q$ and $q_0$.\\}


{\bf We prove that $Z(P_0)\langle h \rangle$ does not occur always:}\\

According to Lemma \ref{ungpcomp} a), III (1), there have to exist $a\in GF(q)\setminus GF(q_0)$ such that $x_0(a\sigma) - a(x_0\sigma) \in GF(q_0)$, where $h=(x_0,y_0,z_0)\in P_0$.\\
\begin{equation}
\begin{split}
x_0(a\sigma) - a(x_0\sigma)  & =: d_0  \hspace*{1.55cm}\mbox{  apply  } \sigma\\
(x_0\sigma) a^3 - (a\sigma) x_0^3 & = (d_0\sigma)  \hspace*{1cm}\mbox{  add $x_0^2$-times  the first row}\\
(x_0\sigma) a^3 - x_0^2(x_0\sigma) a - (d_0\sigma) - d_0x_0^2 & = 0
\end{split}
\end{equation}
Since $ (x_0\sigma) X^3 - x_0^2(x_0\sigma) X - (d_0\sigma) - d_0x_0^2 $ is a polynomial over $GF(q_0)$, to have a root in $GF(q)\setminus GF(q_0)$ is possible only if $2n+1 = 3 (2n_0+1)$ ($q=3^{2n+1}$ and $q_0=3^{2n_0+1}$). \\


{\bf We show  that some cases this intersection really occurs:}\\

Let $h:=(1,0,0)\in P_0\setminus {P_0}'$ and let ${p_1}^{-1}:=(a,b,c)\in P\setminus P'P_0$.
Then  by Lemme 2.6 we have that $(1,0,0)^{(a,b,c)}=(1,a\sigma-a,-2b,-a+a(a\sigma ))$.
Let us suppose that $a\sigma -a=d_0\in GF(q_0)$. Then the third component of 
$(1,0,0)^{(a,b,c)}$ is $-2b-a+a^2+ad_0\in GF(q_0)$. If this is zero, then
$a^2+a(1-d_0)-2b=0$ and $b=a^2+a(1-d_0)$.
Hence we have that $(1,0,0)^{(a,a^2+a(1-d_0),0)}=(1,d_0,0)\in P_0$.
We have to look for some $a$ with $a\sigma -a\in GF(q_0)$ and $a\in GF(q)\setminus GF(q_0)$. Since $\sigma $ is raising to the $n+1$-st power, we have to find 
$a$ with $a^{n+1}-a\in GF(q_0)$ and $a\in GF(q)\setminus GF(q_0)$.
Let us suppose that $n_0=1$ and $n=4$. Then $9=2n+1=3\cdot 3=(2n_0+1)\cdot 3$
and there are $36$ elements $a$ in $GF(3^9)\setminus GF(3^3)$ with $a^5-a\in GF(3^3)$. Let $h:=(1,0,0)$, let $a$ be one of these $36$ values, $d_0:=a^5-a\in GF(3^3)$. Then for $p_1^{-1}:=(a,a^2+a(1-d_0),0)$ we have that $G_0^{(p_1)}=Z(P_0)\langle h \rangle .$
\\

By Lemma \ref{ungpcomp} if $G_0^{(g)}$ contains a nontrivial $3$-element then there exists
an $r\in G_0$ such that $r^{-1}g=p_1\in P\setminus P_0$ such that $G_0^{(g)}=
(P_0W_0)^{(p_1)}$. If $P_0,P_0\langle i \rangle$ or $P_0W_0$ would occur then $p_1\in Z(P)\setminus Z(P_0)$.
For such $p_1$ we have that $P_0\leq (P_0W_0)^{(p_1)}=G_0^{(g)}$.
{\bf Let $g=z\in Z(P)\setminus Z(P_0)$, we show that only $G_0^{(z)}=P_0$ can occur.}\\
\noindent

{\bf First we show that $(P_0W_0)^z\cap P_0W_0=P_0W_0$ cannot occur:}\\
Suppose by contradiction that it occurs. It means that  there exists an element $p_0\in P_0$ such that $W_0^z=W_0^{p_0}$. Hence $zp_0^{-1}\in N_{PW}(W_0)\leq N_{PW}((W_0)_{2'})\leq N_{PW}(W_{2'})=N_{PW}(W)\leq N_G(W)\simeq D_{2(q-1)}$, see Theorem \ref{ree} (b).
Hence $z=p_0\in Z(P)\cap P_0=Z(P_0)$, contradicting to the choice of $z$.
Thus $P_0W_0$ does not occur.\\

{\bf Now we show that $P_0\langle i \rangle $ cannot occur, either.}\\

%Let $z\in Z(P)\setminus Z(P_0)$. Let us suppose that $(P_0W_0)^z\cap P_0W_0\supseteq P_0\langle i \rangle $. 
%Then $i^z=i^{p_0}$ for some $p_0\in P_0$, hence $zp_0^{-1}\in C_P(i)\leq P'$ and so $p_0\in P_0'$. Hence $i^z\in Z(P)\langle i \rangle \cap P'\langle i \rangle =
%\langle i\rangle (P_0\cap Z(P))=\langle i \rangle Z(P_0)$.
%However each element of $Z(P)$ conjugates $i$ to different elements, so this is impossible that $i^z=i^{z_0}$ for some $z_0\in Z(P_0)$. Hence if $z\in Z(P)\setminus Z(P_0)$ then $G_0^{(z)}=P_0$.\\
%$G_0$ legkisebb Ree-re is jo-e ez a biz.???\\[1cm]



Let $z\in Z(P)\setminus Z(P_0)$. Let us suppose that $(P_0W_0)^z\cap P_0W_0=P_0\langle i \rangle $.
Then $i^z=i^{p_0}$ for some $p_0\in P_0$, hence $zp_0^{-1}\in C_P(i)\leq P'$ and so $p_0\in P_0'$. Hence $i^z\in Z(P)\langle i \rangle \cap P_0'\langle i \rangle =
\langle i\rangle (P_0'\cap Z(P))=\langle i \rangle Z(P_0)$.
However each element of $Z(P)$ conjugates $i$ to different elements, so it is impossible that $i^z=i^{z_0}$ for some $z_0\in Z(P_0)$.\\

{\bf  Hence if $z\in Z(P)\setminus Z(P_0)$ then $G_0^{(z)}=P_0$.}\\
%$G_0$ legkisebb Ree-re is jo-e ez a biz.???\\[1cm]

Let now $H:=R(3)$. It can be maximal in $G$ if and only if $2n+1$  is a prime.

In \cite{HHP2016} we  determined $\mathcal{U}_1(H)$ up to certain cases, which we will describe now.\\

$\mathcal{U}_1(H)=\{H, 1, C_2, C_2^2 \}\cup U_{N_H(P_0)}\cup U_{N_H(M_0^{+1})}$. (see \cite[Prop. 8.1, Cor 8.3, Prop 8.4, Prop 8.5, Prop 8.6]{HHP2016})\\

These  subgroups are up to isomorphism:\\

$\{H, 1, C_2, C_2^2, Z(P_0)\simeq C_3, P_0^1\simeq C_3^2, P_0^1\rtimes C_2\simeq C_3^2\rtimes C_2, Z(P_0)\langle h \rangle \simeq C_9, P_0, P_0\rtimes C_2 \}.$ 

The missing cases are: $C_2^2$, $Z(P_0)\langle h \rangle$, $P_0$, $P_0\rtimes C_2$.\\

{\bf Let $V\leq H $ be a  Klein four subgroup of $H$.} Then $C_H(V)=S$ the unique Sylow
$2$-subgroup containing $V$.
We know that $C_G(V)=V\times (M\rtimes C_2)$, where $|M|=\frac{q+1}{4}$.
 Let $x\in M\setminus H$. 
Then $H^x\cap H\geq V$, but it does not contain $S$, otherwise $V\leq S,S^{x^{-1}}\leq H$. Thus $S^x=S$ and $x\in N_G(S)=N_H(S)$, which is a contradiction  since $x\not\in H$.
  Hence by our list of possible intersections, $H\cap H^x$ can  only be  $V$.
{\bf Thus $C_2^2$ occurs}.\\


With analogous methods as in the case of  $G_0$, {\bf we can exclude $P_0\rtimes C_2$ and we can prove that $P_0$ occurs.
$Z(P_0)\langle h \rangle $ occurs only if $q=27$.}\\



%We show that in $N_G(M)$ all subgroup of order 6 are conjugate, and so the proof for $N_G(M^1)$ will work here also.\\
%Fist we will check that the order of $tm^ai$ is always 6, where $\langle m\rangle = M$, $i\in V=\langle i,j \rangle$ and $a \in \mathbb{Z}$. Before thar let us investigate how the cojugation by $t$ acts on the elements of $V$ or $M$. We can assume that $i^t=j$m and then $j^t=ij$. We know that there exists $\tau\in\mathbb{Z}$ (maybe $\tau$ not a good letter), where $(m^a)^t=m^{a\tau}$ for arbitrary $a$. Since $m^{a\tau^6}=m^a$, $(q-1)/4$ divides $\tau^6-1$. According to Lemma 2.3 (ii) $(m^a)^t=(m^a)^\tau\neq m^a$, i. e. the gcd of $(q-1)/4$ and $\tau-1$ is 1. Thus $(q-1)/4$ divides $\tau^5+\tau^4+\tau^3+\tau^2+\tau+1$. We are ready to compute the order of $tm^ai$:\\
%$(tm^ai)^2=t^2 (m^a)^t i^tm^ai =t^2m^{a(\tau+1)}ij\neq 1$\\
%$(tm^ai)^3=t^2m^{a(\tau+1)}ij t m^ai=t^3 (m^{a(\tau+1})^t( ij)^tm^ai=t^3m^{a(\tau^2+\tau+1)}\neq 1$\\
%$(tm^ai)^6=t^3m^{a(\tau^2+\tau+1)}t^3m^{a(\tau^2+\tau+1)}=t^6(m^{a(\tau^2+\tau+1)})^{t^3}m^{a(\tau^2+\tau+1)}=
%t^6m^{a(\tau^5+\tau^4+\tau^3+\tau^2+\tau+1)}=1$.\\

%In $MV$ does not contain involution, which commutes to an element of order 3 in $N_G(M)$.  Thus all subgroups of order $6$ in $N_G(M)$ is part of $N_G(M)\setminus MV$. Since in $N_G(M)\setminus MV$ only the elements in form $tm^ai$ and $t^5m^ai$ can have order $6$, we get that there are $q+1$ subgroup of order $6$.\\
%Na most ezt m\'eg ki k\'ene tal\'alni hogy hogyan lehet j\' ol beillszteni az ujabb verzi\'okba.\\
%Using kicsiree22 Lemma 13, we know that the elements of $MV$ never centralize $t$ or any elements of order $6$. Thus at least $q+1$ subgroup of order $6$ is conjugate and so all of them is conjugate.

%\end{document}



\noindent
{\bf Acknowledgements:} The authors thank for the referee for  suggesting
to include  the information which is contained in Theorem\ref{tsubd}  and also
for pointing out that our results improve some bound on base sizes. The first and the second author were supported
by the National Research, Development and Innovation Office -NKFIH Grant No. 115288 and 115799. 


\begin{thebibliography}{40}
\bibitem{BDK} R.~Boltje, S.~Danz and B.~K\"ulshammer,
\newblock On the depth of subgroups and
group algebra extensions,
\newblock {\em Journal of Algebra} {\bf 335} {\bf 1} (2011) 258--281.

\bibitem{BKK} S.~Burciu, L.~Kadison, B.~K\"ulshammer,
\newblock On subgroup depth,
\newblock {\em Int. Electron. J. Algebra} {\bf 9} (2011) 133--166.

\bibitem{BLSh} T.~C.~Burness, M.~W.~Liebeck and $A.~Shalev$,
\newblock Base sizes for simple groups and a conjecture of Cameron,
\newblock {\em  Proc. London Math. Soc.} (3) {\bf 98} {\bf 1} (2009) 116--162.

\bibitem{F} T.~Fritzsche,
\newblock The depth of subgroups of $PSL(2,q)$,
\newblock {\em Journal of Algebra} {\bf 349 } (2012) 217--233.

\bibitem{F2} T.~Fritzsche,
\newblock The depth of subgroups of $PSL(2,q)$ $II.$,
\newblock {\em Journal of Algebra} {\bf 381} (2013) 37--53.

\bibitem{FKR} T.~Fritzsche, B.~K\"ulshammer,~C.~Reiche,
\newblock The depth of Young subgroups of symmetric groups,
\newblock {\em Journal of Algebra} {\bf 381} (2013) 96--109.

\bibitem {GHJ} F.~Goodman, P.~de~la~Harpe, V.F.R.~Jones. (1989).
\newblock Coxeter graphs and towers of algebras.
\newblock Springer, New York.

\bibitem{G} D.~Gorenstein,
\newblock Finite groups,
\newblock  Chelsea Publishing Co., New York, 1980

\bibitem{HHP} L. ~H\'ethelyi, E. ~Horv\'ath, F. ~Pet\'enyi,
\newblock The depth of subgroups of Suzuki groups,
\newblock {\em Communications in Algebra} {\bf  43} (2015) 4553--4569.


\bibitem{HHP2016}L. ~H\'ethelyi, E. ~Horv\'ath, F. ~Pet\'enyi,
\newblock The depth of maximal subgroups of Ree groups,
\newblock Preprint 2016. arXiv:1608.06774 

\bibitem {Hup1} B. ~Huppert,
\newblock Endliche {G}ruppen,
\newblock Springer-Verlag, Berlin, Heidelberg, New York 1963.

\bibitem {Hup2} B. ~Huppert and N. ~Blackburn,
\newblock Finite groups II,
\newblock Springer-Verlag, Berlin, Heidelberg, New York, 1982.

\bibitem{Hup} B.~Huppert and N.~Blackburn,
\newblock Finite groups III,
\newblock Springer-Verlag, Berlin, Heidelberg, New York, 1982.

\bibitem{K} L. ~Kadison,
\newblock Algebra depth in tensor categories,
\newblock arXiv:1511.02349 2015.
\newblock {\em Communications in Algebra} {\bf 34} (2006) 3103--3122.

\bibitem{KHSZ} L. ~Kadison, A. ~Hernandez, M. ~Szamotulski,
\newblock Subgroup depth and twisted coefficients,
\newblock {\em Communications in Algebra } {\bf 44} (2016) 3570--3591.

\bibitem{KK} L. ~Kadison, B. ~K\"ulshammer,
\newblock Depth two, normality and a trace ideal condition for Frobenius extensions
\newblock {\em Communications in Algebra} {\bf 34} (2006) 3103--3122.



\bibitem{KN} L. ~Kadison, D. ~Nikshych,
\newblock Hopf algebra actions on strongly separable extensions of depth two,
\newblock {\em Advances in Mathematics} {\bf 163} (2001) 258--286.

\bibitem{Kleidman} P.~B.~Kleidman,
\newblock The maximal subgroups of the Chevalley groups $G_2(q)$ with $q$ odd, the Ree groups $^2G_2(q)$, and their automorphism groups,
\newblock {\em Journal of Algebra} {\bf 117} (1988) 30--71.

\bibitem{Landrock} P..~Landrock and G.~O.~Michler,
\newblock Principal $2$-blocks of the simple groups of Ree type,
\newblock {\em Transactions of the American Mathematical Society} {\bf 260} (1980) 83--111.


\bibitem{LN} V.~M.~Levchuk and Ya.~N.~Nuzhin,
\newblock Structure of Ree groups,
\newblock {\em Algebra and Logic} {\bf 24} (1985) 16--24.

%\bibitem{PP} F.~Pet\'enyi, Programs. www.math.bme.hu/~pfranci/programs.html 2016.

\bibitem{zll} S.~Zhou, H.~Li and W.~Liu,
\newblock The Ree groups $^2G_2(q)$ and $2-(v,k,1)$ block designs,
\newblock {\em Discrete Mathematics} {\bf 224} (2000) 251--258.

%\bibitem{WM} Wolfram Mathematica. http://www.wolfram.com/mathematica.

\bibitem{W} H.~N.~Ward,
\newblock On Ree's series of simple groups,
\newblock {\em Transactions of the American Mathematical Society} {\bf 121} (1966) 62-89.

\bibitem{Wie} H.~Wielandt,
\newblock Zum Satz von Sylow,
\newblock {\em Math. Z.} {\bf 60} (1954) 407--409.


\bibitem{RW} R. A. Wilson,
\newblock The Finite Simple Groups,
\newblock  Springer, Berlin, Heidelberg, New York 2007.
\end{thebibliography}

\end{document}




                                             








